The Project Gutenberg eBook of En sommarsaga från Finland

This ebook is for the use of anyone anywhere in the United States and
most other parts of the world at no cost and with almost no restrictions
whatsoever. You may copy it, give it away or re-use it under the terms
of the Project Gutenberg License included with this ebook or online at
www.gutenberg.org. If you are not located in the United States, you will
have to check the laws of the country where you are located before using
this eBook.

Title: En sommarsaga från Finland

Author: Johannes Alfthan

Release date: July 4, 2017 {[}eBook \#55044{]}

Language: Swedish

*** START OF THE PROJECT GUTENBERG EBOOK EN SOMMARSAGA FRÅN FINLAND ***

Produced by Helvi Ollikainen and Tapio Riikonen

EN SOMMARSAGA FRÅN FINLAND

Berättad af

Johannes Alfthan

Stockholm, Albert Bonniers förlag, 1872.

INNEHÅLL:

\begin{enumerate}
\def\labelenumi{\arabic{enumi}.}
\tightlist
\item
  En skillsmässa.
\item
  En frånvarande presenteras för läsaren.
\item
  På förhoppningarnes kyrkogård.
\item
  Litet politik.
\item
  Finska paralleler.
\item
  En helsning från Finland.
\item
  Om fennomanskor och fennomaner.
\item
  Ödemarkens lif.
\item
  Inkognito resande.
\item
  En ``administrativ tjensteman''.
\item
  Öfverraskningar.
\item
  Vid Imatra.
\item
  ``Ett smultron, vuxet i skuggan.''
\item
  Huru Erik tillbragte qvällen.
\item
  Hvad ett månsken får se.
\item
  Två drömmar på en natt.
\item
  Mellanspel.
\item
  Idylliskt lif.
\item
  Nattligt äfventyr.
\item
  Nya öfverraskningar.
\item
  Vid kaffebordet.
\item
  Länsmans-diplomati.
\item
  Polisundersökning i skogen.
\item
  Hvarför Jenny kom till skogen.
\item
  Gäster på Ojala.
\item
  Skuggan viker.
\item
  Den vackra adoptivdottern. Efterskrift.
\end{enumerate}

En yngling, som har mycket allvarliga minnen från sin skol- och
barndomstid, blir merendels sjelf allvarlig till sitt lynne, men blommor
växa det oaktadt äfven på hans stig och dessa äro, för sällsynthetens
skull, så mycket kärare budskap då han sänder någon af dem såsom en
helsning till sin förre lärare. Den äldre vännens hjerta deltager ju så
innerligt i den yngres förhoppningar och gläder sig åt de framsteg han
gör på egen hand. I detta förhållande står den unga finska nationen till
det äldre svenska folket.

Ehuru vi tyvärr äro mera vane att från andra sidan Bottenhafvet förnimma
dystra vintersagor om hunger och nöd, om svikna förhoppningar och moln
på landets framtidshimmel, en himmel som verkligen i många afseenden
ännu är oklar, så skönjer dock det öga, som vill se, i flera af
samhällslifvets yttringar omisskänneliga spår af ett nytt och friskt lif
hos detta vårt kära fosterbrödrafolk. De der våra vänner, finnarne, ha
duktigt ruskat på sig, och den hand de lagt vid sitt nya nationella verk
saknar ingalunda styrka, ehuru den någon gång påminner om björnen såsom
handtlangare. De slunga nämligen ett och annat stenblock utöfver målet,
men det är dock ett godt stycke arbete de ha förelagt sig och de försöka
med all kraft att foga stenarne i sin byggnads grundval så fast och
hårdt till hvarandra, att det hela må bli ett stadigt verk, som kan
trotsa såväl den frostbringande nordan som ock den andligt döfvande
östanvinden. Sådan är tvifvelsutan deras grundafsigt och som de nu tagit
litet hett och hårdhändt itu med saken, så måste åskådaren hålla till
godo med arbetarnes något bistra uppsyn. Men det är dock värme i deras
allvar, och har man blott ett välvilligt undseende med några
öfverdrifter och ensidigheter, hvilka nog skola afslipas ju längre
verket framskrider, så skall man på bottnen af det nya finska
sträfvandet icke allenast varsna mycket lefnadsmod och kraft, utan äfven
i dess alster på andens område icke så sällan skönja de anslående
grundfärgerna af en frisk egendomlighet och en viss folkhumor.

Den mest ensidiga och svensk-fiendtliga yttring af det ``finsk-finska''
partiets (ty så måste man väl kalla det) verksamhet har erhållit titeln
af ``fennomani'', som ordagrannt; återgifvet på svenska betyder
``finnraseri''. Tillämpad på det finska nationalitetsarbetet i sin
helhet är denna benämning högst orättvis och innebär en djup kränkning
af ett friboret folks naturliga sjelfbestämmelserätt. I Finland ha
uttrycken ``fennomani'' och ``fennomaner'' nästan helt och hållet ingått
i det allmänna språkbruket. -- Men oaktadt de många svåra tiderna, de
inre slitningarne och de ofta allvarliga anletena, så trifves dock
åtminstone tillsvidare ännu äfven ``glädjens blomster i Finlands mull''.
Det ligger derföre intet hån och ingen hädelse deruti att försöka
berätta en solvarm sommarsaga från de ``tusen sjöars land''.

\begin{enumerate}
\def\labelenumi{\arabic{enumi}.}
\tightlist
\item
\end{enumerate}

En skillsmässa.

Det var år 1864, och det var sommar. Den första finska landtdagen (sedan
1809) hade nyss blifvit aftackad och dess medlemmar hade åter dragit sig
tillbaka inom privatlifvet för att hvila ut efter sin hedrande
offentliga verksamhet. Det var i allmänhet en någorlunda glad tidpunkt i
Finlands nyaste historia och ganska väl egnad för studier öfver detta
lands senaste utveckling och de qvasikonstitutionella förhållanden, i
hvilka det inträdt, omständigheter, som icke lemnades obegagnade af den
unge svenske litteratören, ingeniören vid bergsstaten, Erik Stenrot,
hvilken vi härmed presentera för våra läsare. Öppen för och intresserad
af alla nya företeelser på den andliga utvecklingens område, beslöt
denne vår vän att göra ett besök i fosterbrödralandet för att erhålla
ett omedelbart intryck deraf, huru en ung nation går tillväga då den
arbetar uppå att grundlägga sin framtid. För den oberoende mannen voro
beslut och handling ett och hans kappsäck var snart packad samt pass och
ångbåtsbiljett i ordning. ``Aura'' skulle följande morgon klockan två
afgå till den gamla goda staden Åbo samt derifrån längre österut. På
ångbåten med detta namn ville Erik Stenrot göra sitt inträde i Finland.

``Och du vill verkligen lemna oss allena för hela sommaren, min käre
kusin? Säg, är det ditt fullkomliga allvar att resa till Finland?''

Detta yttrades af en ung flicka med dunkelblå ögon som vid den unge
mannens sida promenerade uppför Drottninggatan i Stockholm.

``Ja, söta Jenny'', blef svaret, ``det är det. Och du skall minsann inte
sakna mig i det muntra, af intressanta resande uppfyllda
sommar-Stockholm.''

``Det kommer jag visserligen inte att göra'', återtog Jenny och såg upp
till sin kusin med en af de der underliga blickarne som, när det så
behagade, stodo till den lilla elfvans oinskränkta, ibland kanske litet
sjelfsvåldiga förfogande, ``ty tant Agatha och jag, vi resa också.''

``Hvart då?'' sporde Erik, något öfverraskad och synbarligen nyfiken.

``Det får du väl höra någon gång, förrädiske kavaljer.'' Jenny helsade
med solfjädern och försvann leende i uppgången till den våning hon
bebodde tillsammans med tanten, den femtiåriga fröken Agatha Stråle.

Den sistnämda damens broder, häradshöfding Jakob Stråle, var de två
kusinernas morbror samt den fader- och moderlösa Jenny Bertrams
förmyndare. Han var en af Sverges berömdaste sakförare samt en ganska
välmående och jovialisk gammal ungkarl. Då man frågade honom hvarför han
icke velat gifta sig, svarade han skämtsamt: ``att han alltför mycket
älskade rättvisan för att, genom ett äktenskap enligt svensk lag och
sed, vilja beröfva en qvinna hennes frihet.'' Till denne sin morbror
begaf sig nu Erik för att taga afsked och tillika möjligtvis få veta
något om Jennys resplaner, hvilka intresserade honom mera än han ville
erkänna inför sig sjelf. Han hade nämligen hittills på de flesta af
kusinens och tant Agathas talrika resor utrikes varit deras manliga
följeslagare och kände sig liksom litet stött deraf att denna gång vara
alldeles förbigången samt icke ens ha fått del af målet för den
tillämnade resan. Men morbror Stråle var i denna punkt förbehållsam. Han
tog ett hjertligt afsked af sin systerson; om Jennys resplaner gaf han
leende blott den förklaring att hon denna gång ville resa ``inkognito'',
tilläggande derjemte i ganska allvarlig ton:

``Jag kan inte ogilla Jennys beslut och jag önskar den präktiga flickan
all framgång i utförandet af sin mycket egendomliga idé, men jag har
lofvat tystlåtenhet och inte ens tant Agatha känner Jennys afsigt i hela
dess vidd; det är imellertid frukten af ett moget öfvertänkt beslut.''

Detta meddelande kunde naturligtvis blott stegra ung Eriks nyfikenhet.
Han trodde sig känna Jenny och hennes egendomligheter och förutsåg i
andanom något besynnerligt påhitt. Men hvad var hennes plan, hvart
skulle nu resan gälla? Till Schweitz's, Skotlands eller Norges fjell?
Dem hade Jenny ju besökt förut; nej, det var bestämdt någonting alldeles
splitter nytt. Att det var allvar med saken hade Erik hört på
förmyndarens ton och han kände dessutom alltför väl den i hans tycke
något fantastiska kusinens raska beslutsamhet, för att ett ögonblick
kunna tvifla på utförandet af hennes en gång fattade föresats. Han ville
dock något närmare söka utforska tant Agatha. Klockan sex på
eftermiddagen gjorde han sin afskedsvisit hos de två damerna. Jenny var
kanske litet mindre munter än vanligt och hon beklagade att Erik icke
hade medtagit sin flöjt, ty hon hade så gerna velat höra honom blåsa
någon svensk folkmelodi. Erik förklarade skrattande att flöjten var
inpackad och föreslog att Jenny i dess ställe skulle vid sitt piano
sjunga en liten afskedssång för honom.

``Min sång är inpackad, kusin flöjtblåsare'', sade den besynnerliga
flickan, ``och den klingar inte åter förr än jag uppnått målet för min
resa.''

``Nå, och hvart i Guds namn gäller då resan?'' utropade Stenrot något
otålig.

``Det får du väl höra någon gång, du pligtförgätne kavaljer'', svarade
Jenny leende, ``du, som lofvat att troget åtfölja mig på alla mina
resor, utom den så kallade resan genom lifvet, du förtjenar i sanning
att jag skiljes från dig med den virgilianska versen, som den öfvergifna
Dido tillropar den flyende Aeneas. Den der din plötsligt beslutna resa
till Finland var ett streck i mina beräkningar, men res du allena, jag
reser också. O, hvad det skall bli skönt att åter få andas ny, frisk
luft och i nya förhållanden röra sig så ledig och fri, alldeles fri, --
ty tant Agatha är intet band, hon är bara en sköld.''

``Ja så'', inföll Erik tankfull, ``och jag var ett band för dig?''

``Ja visst, gode Erik'', återtog Jenny med mycken vänlighet, ``och
säkerligen någon gång ett ganska nyttigt band.''

``Var jag då intet annat för dig på våra gemensamma resor, Jenny?''
frågade ynglingen förargad.

``Visserligen'', inföll Jenny muntert, ``du var en ypperlig följeslagare
och jag är dig tack skyldig för en mängd historiska upplysningar och
förklaringar. Jag tackar dig också ännu en gång för stödet af din starka
arm vid våra bergvandringar i Alperna och isynnerhet för den vackra
skotska melodien som du blåste för mig och doktor Ros på Loch-Lomonds
strand. Och tack för din regnkappa som skyddade mig för stänket vid
Rjukan fors och för lexorna och varningarne då jag så lätt gjorde nya
bekantskaper på ångbåtar och i jemvägskupeer\ldots{}''

``Och på bergvandringarne i Skottland sedan!'' afbröt Erik ironiskt.

``O ja'', svarade Jenny hastigt och rodnade litet, ``tack för din
broderliga svartsjuka, min cicerone, min allvetande resehandbok i denna
gamla verlden\ldots{}''

``Jenny!'' utbrast den unge mannen, ``nu är du stygg. Hvarigenom har jag
förtjent detta? Men du sade `gamla verlden', ämnar du måhända företaga
en resa till Amerika?''

``Förlåt, om jag sårat dig, kusin Erik'', bad den unga flickan mildt och
tillade sedan med ett allvar, som klädde det älskliga ansigtet
förträffligt: ``dit jag ämnar resa, der kommer en ny verld att öppna sig
för mig.''

Nu inträdde tant Agatha och detta var ganska bra, ty situationen hotade
med att bli något plågsam för de två i öfrigt med hvarandra så förtrogna
kusinerna. Eriks förhoppning att af den fryntliga, gamla damen få mera
tillfredsställande upplysningar om deras resplan strandade dock helt och
hållet emot ett från detta håll alldeles oförväntadt: ``Käre Erik, Jenny
vill ändtligen att vi skola resa `inkognito'.''

``Men hvart, hvart reser ni?'' sporde Erik, ``det kan väl tant säga?''

``Nej, se det går inte heller an, vi resa alldeles `inkognito', gode
Erik.''

``Men'', invände denne smått stött, ``hittills har jag alltid hedrats
med edert förtroende och jag tycker att\ldots{}''

``Du skall få veta allt bara först någonting händt'', afbröt Jenny i
bevekande ton, ``du skall få veta allt bara det finns någonting att
berätta. Gif dig nu till tåls, snälle Erik, annars tror jag att du är
nyfiken.''

Den bedjande, veka tonen i Jennys stämma förmådde, mera än hennes ord,
Erik att upphöra med sina spörjsmål, men han varnade dock tanten för att
låta Jenny begå något pojkstreck, såsom han, ännu icke glömmande sin
förtrytelse, behagade yttra sig och hvarvid han tyckte att lilla kusin
åter rodnade. Imellertid måste han, på enträgen begäran, lofva att
skrifva till tant Agatha och Jenny, genast efter sin ankomst till
Helsingfors. Derefter tog Erik Stenrot afsked, innerligen missbelåten
med en sådan skilsmessa från sin vackra kusin.

\begin{enumerate}
\def\labelenumi{\arabic{enumi}.}
\setcounter{enumi}{1}
\tightlist
\item
\end{enumerate}

En frånvarande presenteras för läsaren.

Då unge Erik Stenrot skiljdes från moster Agatha Stråle och kusin Jenny
Bertram begaf han sig, upptagen af hvarjehanda tankar, till
Strömparterren, der han stämt möte med några bekanta för att ännu vexla
ett par afskedsord innan han anträdde sin resa. Det glada sällskapet,
forsamladt omkring en butelj punsch, var snart funnet och Erik slog sig
ned ibland de goda vännerna.

``Du är så tyst i qväll, broder Erik'', tog den muntre skådespelaren X.
till ordet, ``hvad tjenar det till? Inte skall du med en sådan der
surmulen uppsyn taga afsked af det glada Sverge.''

``Och inte äro finnarne, som du går att helsa på, heller ett så
förtvifladt slägte att de synnerligen skulle uppbyggas af den der
likbjudarminen du har påtagit dig i qväll'', skämtade en annan,
``dessutom äro finnarne vane att se glada ansigten hos dem som komma
från Sverge.''

``Har du kanske genom din resa till Orienten jordat någon ljuf
förhoppning här hemma?'' frågade en tredje.

``O nej'', inföll åter X. skrattande, ``den vackra reskamraten (i
förbigående nämdt titeln på Eriks senaste novell) följer dig väl såsom
vanligt, och finnarne få väl se urbilden till den omtyckte novellistens
ideal.''

Men den unge författaren gaf icke vidare akt på vännen X:s anspelning.
Det yttrade ordet ``Orienten'' hade gifvit hans tankegång en ny riktning
och han utropade alldeles högt:

``Hon måtte väl inte resa till det forlofvade landet? Hennes håg har
länge stått ditåt -- men allena? Bah!''

``Ditåt står mest alla flickors håg'', menade i tvunget allvarlig ton en
af sällskapet, ``men den resan företages aldrig allena.''

Nu utbrusto alla de andra i skratt och Erik, förlägen för sin
tankspriddhet, visste ingen annan utväg än att deltaga i munterheten och
gifva samtalet en annan riktning i det han frågade:

``På tal om resor, har någon af er sett till vår finne i dag?''

``Menar du Birger Ros'', svarade X., ``så får jag upplysa att han gör
dig ressällskap i morgon; han återvänder till sitt Finland igen.''

``Det vet jag'', sade Erik, ``och det var roligt att få en finne till
reskamrat. Ros är dessutom en intelligent man. Han lofvade mig i går att
här sammanträffa med oss.''

``Det är en underlig man, vår vän Ros'', menade doktor B., ``han har
rest mycket, men det ser nästan ut som skulle han inte ha ro
någonstädes. I Upsala tycktes han inte riktigt trifvas.''

``Är han då folkskygg?'' frågade någon.

``Åh nej'', genmälte doktorn, ``inte just det, ehuru han dock i
allmänhet är ganska sluten. Han är bestämdt en stor entusiast och der
framskymta ibland nästan vilda blixtar af inre glöd, kanske tvifvelsmål
om hans fäderneslands politiska framtid. Jag tror att han är en af de
der så kallade fennomanerne.''

``Så vidt jag känner honom'', inföll Erik, ``är han det ända till
fanatism, ehuru han här i Sverge inte gerna bär sina åsigter i denna väg
till torgs. Jag håller nästan med B.; han lider verkligen af sorg och
tvifvel öfver Finlands möjliga öde.''

``Ja'', återtog B., ``det är en alltigenom djup och allvarlig karakter
och under den ofta kalla ytan glöder en inre brand, som förr eller
senare skall i förtid förtära honom om han inte finner någon praktisk
afledare för sin verksamhetsdrift. Vore han skald, så vore han bestämdt
en liten finsk Almqvist, men hans håg ligger inte egentligen åt
parnassen. Med all sin fosterländska entusiasm förenar han likväl
alltför mycket lugn beräkningsförmåga för att kunna skåda Finlands
framtid i alldeles ljusa färger. Men jag tror att der samverkar en ännu
bittrare känsla af mera individuel beskaffenhet, ty hans verldsåskådning
är verkligen något dyster, en känsla, hvars natur jag inte kan förklara
och inte vill gissningsvis uppgifva.''

Trots Eriks tankspriddhet började dock samtalet intressera honom och han
meddelade sina vänner hvad han kände om den egendomliga personlighet,
som var i fråga. Erik hade, på en af sina resor med tant Agatha och
Jenny, i Skotland sammanträffat med den unge finnen. De hade der
företagit flera turer tillsammans samt genomlefvat en ganska
njutningsrik tid af några veckor. Ros vistades i Skotland för att
studera det skotska skolväsendet. Han var för öfrigt juris doktor och
finsk publicist samt beklagade mycket att han icke fått öfvervara,
Finlands första landtdag. Men ett reseunderstöd af allmänna medel tvang
honom att en bestämd tid vistas utrikes. Hvad som då förefallit vår vän
mest besynnerligt i finnens väsende var dennes bestämda afvoghet mot
engelsmännen i deras historiska förhållande till Skotland. Få Eriks
öppet uttalade beundran för det stora öfolket hade Ros, som i öfrigt
skänkte det britiska folkets frihetssinne allt erkännande, icke utan
bitterhet frågat om han, Erik, också beundrade engelsmännen derföre att
de tillintetgjort Skotlands sjelfständighet och öfverflyglat dess
ursprungliga nationalitet. Det är en lögn, hade han ofta sagt, att ett
folk som alldeles ur sig sjelf skapat en så egendomlig diktverld som
Ossians sånger, icke inom sig skulle ha egt förmåga att arbeta sig upp
till en sjelfständig och nationel bildningsform. Men såsom historien
skiftat lotterna, bär nu imellertid all bildning i Skotland en engelsk
pregel och det rent skotska elementet har blifvit undertryckt i tidernas
lopp, dess språk undanträngdt från den högre kulturens sferer och ansedt
såsom ett halfbarbariskt idiom. För allt detta har Skotland att tacka
engelsmännen -- och likaså, brukade han tillägga, hade det väl äfven
gått med Finland och finnarne gentimot den svenska kulturen och dess
inflytande, om icke de politiska händelserna i början af detta sekel
lösryckt Finland från Sverge och anvisat detsamma en helt ny och
nationel utvecklingsbana. På senare tider trodde Erik sig dock ha märkt
att en stor förändring höll på att försiggå i Ros' åsigter i detta
afseende, men till hvilken ståndpunkt denna inre kamp skalle leda hans
vän, kunde han ännu icke bedömma.

Så långt hade Erik hunnit i sina meddelanden då han plötsligen såg sig
föranlåten att afbryta desamma med ett: ``men se der ha vi vår finne.''

Den person som nu framträdde till det bord, vid hvilket Erik och hans
vänner suto, var en välväxt, kraftfull gestalt med ett något blekt men
själfullt ansigte. Han helsade frimodigt på de närvarande, af hvilka han
kände de flesta. I hans tal röjde sig denna egendomliga brytning hvarpå
finnen, äfven då svenska språket är hans modersmål, alltid igenkännes;
det var Birger Ros. ``Grod afton, reskamrat'', helsade Erik glädtigt,
``är du nu klar att dricka ett glas punsch här och sedan på Hasselbacken
intaga en liten afskedssexa med oss?''

``Tack!'' blef svaret, ``men uppriktigt sagdt, har jag ännu ett och
annat att uträtta och skyndade hit hlott för att säga till det jag
sannolikt något senare skall infinna mig på Hasselbacken. Farväl alltså
tillsvidare.'' -- De öfriga af sällskapet broto också upp och begåfvo
sig med ångbåt till Djurgården.

Muntert glam, skämtsamma tal och afskedsskålar hade snart helt och
hållet skingrat Eriks tankspriddhet. Väl stördes glädjen något deraf att
Birger Ros icke infann sig, men å andra sidan var man van vid hans
egendomligheter. Den helsning till Finland som en af talarne haft för
afsigt att adressera till Ros uppdrogs nu åt Erik att frambära. Med den
sista båten begåfvo sig denne och hans glada kamrater till staden och
Erik gick ombord och till hvila, för att i sömnens armar för några
timmar förgäta alla hemlighetsfulla kusiner och skämtande vänner, skålar
och bålar.

\begin{enumerate}
\def\labelenumi{\arabic{enumi}.}
\setcounter{enumi}{2}
\tightlist
\item
\end{enumerate}

På förhoppningarnes kyrkogård.

Men hvart hade Ros begifvit sig då han lemnade sina vänner i
Strömparterren? Låtom oss följa honom. Klockan var ännu icke nio och i
den ljusa sommarqvällen kunde med lätthet alla föremål urskiljas. Den
unge finnen ställde med en viss brådska sina steg åt norr. Han gick
öfver Gustaf Adolfs torg, Malmtorgsgatan och Brunkebergstorg samt
beträdde den långa Malmskillnadsgatan. Ju mera han dock nalkades
Johannis kyrkogård, i samma mon saktade han ock sina steg. Han såg på
sin klocka; den var straxt nio. Det var som hade han kännt bly i sina
fötter och hans gång blef ännu långsammare. Djupa och ingalunda glada
tankar uppfyllde hans själ. Men då klockorna i kyrktornen gällt
förkunnade att den nionde timmen förlidit, for han plötsligt upp ur sina
drömmerier och liksom för att bättre draga sig någonting till minnes
eller ock uppmuntra en svigtande vilja, sade han halfhögt för sig sjelf:
``sidoalléen till höger'', och började påskynda sina steg.

Ankommen till den plats i hufvudgången der en sidoallée i rät vinkel
leder till höger stannade Birger ett ögonblick. Tvekade han väl ännu?
Solen hade gått ned och mystiska halfdagrar hade lägrat sig under de
höga, lummiga träden på kyrkogården, men den nordiska sommarqvällens
långa skymning qvardröjde ännu öfver den tysta, ensliga platsen och
tillät ögat att äfven på något afstånd urskilja alla föremål. En blick
af Birger inåt alléens dunkel visade honom en qvinnogestalt, sittande på
en af de längst bort ställda bänkarne till höger. Hans hjerta klappade
våldsamt och om någon i detta ögonblick skådat djupt in i den unge
finnens dunkla öga, så hade han der mött en blick som talade om ganska
stridiga känslor. Plötsligen tycktes dock en tanke uttränga alla de
andra och denna var lika enkel som tillfyllestgörande att häfva all
vidare tvekan: ``det är ju jag som bedt henne komma och hon har beviljat
min bön.'' Med ett slags förtviflad beslutsamhet i de af en svår inre
strid krampaktigt upprörda anletsdragen beträdde nu Birger sidoalléen
och ställde sina steg fram till det på bänken sittande fruntimret.

Då Ros nalkades steg fruntimret upp. Det var en ung dam med nästan
sylfidisk växt. Hon gick några steg imot den kommande och, slående sin
slöja tillbaka från ett vackert och kanske blott för tillfället litet
stolt ansigte, talade hon med sakta men lugn stämma i det hon räckte den
vördsamt helsande unge mannen handen:

``God afton, herr Ros, hvarföre har ni begärt ett möte på detta
besynnerliga ställe?''

``För att taga afsked af den enda förhoppning jag hyst att sällhet kunde
förenas med min framtid. Det var en ljuf dröm, hvars luftslott
verkligheten krossat. Men, min fröken, jag har inte haft styrka att gå
denna glädjelösa framtid till mötes utan att taga afsked af er.''

``Och hvarföre har ni inte besökt oss här i Stockholm?'' sporde
fruntimret med lindrigt sväfvande röst, ``hvarföre undvek ni oss, herr
Birger?''

``Af feghet. Jag fruktade att duka under för min känslas makt\ldots{}''

Det unga fruntimret rodnade djupt, de sköna dunkla ögonhåren sänkte sig
öfver de ännu skönare ögonens förtrollande ljusverldar och i en knapt
hörbar, mycket vek ton framsmögo öfver hennes läppar de orden: ``Och om
nu denna er känsla\ldots{}''

``O, säg inte ut er tanke!'' afbröt Birger nästan vildt och tillade
sedan med våldsamt dämpad rörelse: ``Jag har en tröst i sjelfva tviflet
på er, låt mig behålla detta tvifvel. Vissheten att min kärlek är
obesvarad skulle öka min börda och jag tigger fegt och egoistiskt om
förskoning. Det nästan otroliga åter -- och dock har jag stundtals trott
derpå -- att min upproriska känsla funnit gensvar skulle försänka mig i
ett kaos af förtviflan.'' Han tillade efter en kort paus, och blick och
ton tolkade bättre hans känslor än de frampressade orden:

``Låt mig derföre behålla mitt förtärande men saliggörande tvifvel -- en
helig, orygglig ed binder mig i alla fall vid ett oblidkeligt
olycksöde\ldots{}''

``Jag känner den edens innehåll'', sade den unga qvinnan mildt. En
ofrivillig skakning genomilade den starke ynglingens hela väsende och,
fattande hennes icke undandragna hand, utropade han lidelsefullt:

``Du känner min ed -- farväl!''

Då såg hon upp till honom.

Der glänste det liksom tårar i de ljufva, strålande ögonen. Var det
uppfriskande dagg på hans hjertas glödande sorg, var det en skön
morgonrodnads löfte om sol och ljus för den kommande dagen?

Såg han denna blick?

En lätt kyss brann på flickans hand -- och bort ilade den
olyckligt-lycklige mannen.

Hon stod der ännu en stund qvar, såg med ett egendomligt uttryck i de
svärmiska ögonen på sin hand, der nyss hans läppar hvilat ett ögonblick,
och hviskade: ``Jag tror inte på ett oblidkeligt olycksöde.'' Med
ljudlösa steg sväfvade hon bort från kyrkogårdens tysta rike, men öfver
grafkullarne gick aftonflägtens stilla susning och tufvor och blommor
drömde sin dröm om försoning och frid.

\begin{enumerate}
\def\labelenumi{\arabic{enumi}.}
\setcounter{enumi}{3}
\tightlist
\item
\end{enumerate}

Litet politik.

Då vår vän Erik följande morgon uppvaknade i sin hytt befann sig
ångfartyget redan på Ålands haf. Han gjorde hastigt sin toilett och gick
upp på däck för att andas frisk luft. Den förste person han mötte var
Birger Ros, som med stora steg promenerade af och an på akterdäck, såsom
vanligt försänkt i djupa tankar. Sedan vännerne helsat på hvarandra,
frågade Erik hvarföre Ros i går qväll icke kommit ut till Hasselbacken.
Svaret var undvikande och den unge finnen tycktes öfver hufvud taget
icke vara vid synnerligen godt lynne, åtminstone svarade han högst
fåordigt på den andres många frågor, till dess denne omsider blef otålig
och öfvergaf förhoppningen att med honom inleda ett ordentligt samtal.
Snart hade äfven den liflige Erik ibland ångbåtspassagerarne gjort en
hel hop bekantskaper, med hvilka han genast inlåtit sig i ifriga samtal
om de finska förhållandena. På eftermiddagen sammanträffade han åter med
den fortfarande mulne Ros.

``Hör på, bror Birger'', började han, ``du är inte vid godt kourage i
dag; hvad går åt dig? Jag tycker att du borde, för att tala poetiskt, på
förhoppningarnes vingar ila till ditt liksom pånyttfödda fädernesland,
isynnerhet som Finland verkligen med heder bestått sitt första och svåra
prof på det konstitutionella statslifvets bana. Alla dina landsmän med
hvilka jag samspråkat under dagens lopp äro lifvade af glada utsigter
för framtiden. Särskildt borde dock just du med din varma
fosterlandskärlek och dina utpreglade specielt finska idéer, ty jag vet
att du är en riktig `fennoman', glädja dig åt det erkännande som till
exempel det finska språket numera vunnit och dess nästan fullkomliga
likställande med svenskan.''

``Min vän'', svarade Ros mycket allvarligt, ``jag är också verkligen
rätt \emph{belåten} med den \emph{gåfvan}.''

``Nå, hvad kan du väl önska mera?'' återtog Erik, ``nu är fältet fritt
för en nästan obegränsad utveckling af alla folkets inneboende
krafter.''

``Och'', inföll här Ros dystert, ``hvad vill du att jag skall tro om
djupet och halten af krafter, hvilka \emph{nu} såsom en \emph{gåfva}
imottagit hvad dem rätteligen tillkommit redan för århundraden sedan?''

``En \emph{gåfva}? Du betonar för andra gången detta ord; hvad menar du
dermed?'' frågade Erik.

``Jag menar helt enkelt att den bästa egendom är den'', genmälte finnen,
``som man sjelf förvärfvat sig och att en sådan är vida att föredraga
framför hvilken gåfva som helst.''

``Du kan väl inte neka att i Finland just en påtryckning ifrån sjelfva
massan af folket hufvudsakligen framkallat denna sakernas vändning'',
utbrast den unge svensken med värma, ``så har åtminstone jag alltid
uppfattat hela den fennomanska rörelsen och äfven den så kallade
`språkfrågan'. Det finska folket har ändtligen vaknat till fullt
medvetet lif och går nu med sjelfförtroende sin nationella utveckling
till mötes. De djupa lederna ha genomträngts af ett stort och ädelt
sträfvande.''

Erik tystnade och Ros förblef svaret skyldig.

``Nå, men så tala då, gillar du kanske inte min uppfattning?'' sporde
den förstnämde ifrigt, ``just du vore då, besynnerligt nog, den förste
finne jag råkat, som numera inte tror på sitt folks pånyttfödelse!''

``Huru många \emph{finnar} har du råkat sedan du kom till Finland?''
frågade Ros sarkastigt. ``Men se der'', fortfor han och utpekade med
handen en vidlyftig skärgård, imot hvilken de styrde kurs, ``de der
öarne tillhöra Finland, det är Åland, -- nåväl, huru många af dess
sextontusen invånare tror du förstå ett enda ord finska? Var öfvertygad
derom, inte många hundra, och alla, jag säger alla, förklara att de inte
äro finnar och skulle blygas'', tillade han kallt, ``att anses för
sådana.''

``Åland är ett undantag'', genmälte Erik, ``och ett undantag gör inte
regeln. Hos oss kalla sig också Gotlands invånare inte svenskar, utan
gotländingar.''

``Du misstager dig betydligt, broder Erik'', svarade Ros med ett bittert
leende, ``gotländingarne \emph{äro} dock svenskar, men åländingarne äro
verkligen inte finnar, de äro svenskar\ldots{}''

``Nå, så låt dem då i Guds namn vara svenskar, de äro ju i alla fall
finska undersåter, dessa högvigtiga sextontusen åländingar'', inföll
Erik skrattande.

``Ja, men'', återtog Ros, ``nyländingarne och en del österbottningar äro
också svenskar och hvad som vill säga vida mera än dessa par hundratusen
menniskor, det är, att en ganska stor del af de så kallade `bildade
klasserna' jemväl är och ännu mera anser sig vara af svensk härkomst.
Svenskan är i alla fall deras modersmål. Och'', tillade han dystert,
``säg mig då, broder, hvad återstår såsom genuint finskt?''

``De djupa lederna!'' utropade svensken med värma.

``De djupa lederna'', återtog finnen i nästan skärande ton, ``de
\emph{djupa lederna} under svensk eller svenskfinsk \emph{ledning}. Ha,
ha! -- hvad äro dessa svenskfinnar, som tro och påstå att de inom sin
klass ha sammanfört all landets, hela folkets intelligens? En ädlare
race kanske? Ve min tunga, som ens uttalat ett sådant ord! Våld och
list! De ha öfverväldigat oss i sömnen och nu sitta dvergarne mysande,
ja hånleende på den fjettrade folkresens skuldror, -- men en gång skall
en annan tid komma och ur den nu sömndruckna massan skola andar framgå
som, utjemnande historiens orättvisor, skola rycka kulturens,
bildningens banér ur fåtalets klor och återställa det i den rättmätige
ärfvingens hand. Och då, först då skall man kunna tala om ett verkligt
Finland. `Svenskheten' skall icke mera beherrska vårt land, bort
derföre, bort! med alla half-finnar och svensk-finnar. Vi skola bli ett
enda och enigt folk.''

Ros tystnade och afiägsnade sig hastigt åt fören. Det låg något i hans
ansigtsuttryck liksom skulle han lia ångrat hvad han yttrat, men ännu
känna sig alltför upprörd att vilja eller kunna återtaga sina ord. Erik
Stenrot blef förvånad öfver detta sällsamma utbrott af nationel
bitterhet. Men han ansåg, vid mognare eftertanke, hela det om en viss
hätskhet vittnande utfallet, blott såsom en öfvergående yttring af en
något kittslig nationalfåfänga, som kände sig sårad till och med deraf
att se sig tvungen erkänna imottagandet af goda gåfvor från en annan
nation, med hvilken det finska folket likväl i flera århundraden lefvat
i trogen endrägt samt villigt delat både lust och nöd. Han uttalade för
sig sjelf i all tysthet den förhoppningen att icke alla fennomaner, en
benämning hvilken han för öfrigt ogillade, skulle vara lika ``vilda''
som denne hans äldste finske vän nu visat sig vara -- och han blef rätt
glad då han efter en stunds förlopp såg Ros återvända och med molnfri
blick nalkas honom.

``Det var ett föga gästfritt välkommen, min gode svenske vän och broder,
de der orden jag nyss yttrade'', sade Ros. ``Se så, kom, låt oss dricka
ett glas svensk punsch tillsammans! Jag helsar dig hjertligt välkommen
på gamla Suomis bölja, skål!''

Glasen klingade och snart var all bitterhet försvunnen ur de två
vännernas hjertan och samtal. Ros meddelade den uppmärksamme Erik flera
värderika, ofta äfven rätt pikanta upplysningar om ställ ningar och
förhållanden i Finland, hvarvid den i vissa kretsar sig utbildande
byråkratiska andan ingalunda förblef onäpst. Men ett visst svårmod
tycktes dock hvila öfver den unge finnens hela väsende, så själfullt och
ofta snillrikt hans samtal än var.

Imellertid plaskade den ståtliga ``Aura'' fram genom den åländska och
åbolänska skärgården och uppnådde den gamla Aurastaden Abo. Med ett
visst allvar i sin för öfrigt glada sinnesstämning beträdde Erik Stenrot
för första gången i sitt lif denna historiskt minnesvärda, nu för Sverge
i yttre afseende förlorade mark.

\begin{enumerate}
\def\labelenumi{\arabic{enumi}.}
\setcounter{enumi}{4}
\tightlist
\item
\end{enumerate}

Finska paralleler.

Vi förbigå alldeles Eriks korta vistelse i Abo, hvilken stad han hade
för afsigt att egna mera uppmärksamhet på sin återresa till hemlandet,
och följa honom der han nu fortsätter resan till Helsingfors, dit hans
håg också mest stod. Der skulle han ju med ens försättas i sjelfva
medelpunkten af det offentliga lifvet i Finland och der ansåg han sig
allra bäst kunna lägga grunden till sina studier om och öfver
brödrafolkets senaste utveckling. Han tillsporde sin vän Ros härom och
uttalade den förhoppning att denne skulle införa honom i några litterära
kretsar, såsom varande de sidor af det allmänna lifvet der han tydligast
och klarast finge se det allmänna tänkesättet afspegla sig, i synnerhet
hvad den för Finland egendomliga dualismen imellan det svenska och det
finska folk-elementet vidkom. Men Ros' svar utföll till en del imot hans
förmodan.

``Jag skall med nöje'', sade denne, ``presentera dig för några af våra
framstående personligheter, såvida jag är bekant med dem och de vistas i
Helsingfors, men utan ringaste afseende på hvilket parti de tillhöra
eller om de öfver hufvud taget alls tillhöra någon bestämd riktning. Om
du från början blefve införd i vissa litterära, hos oss ofta äfven
qvasipolitiska kretsar, så skulle måhända det första intrycket utöfva
ett afgörande inflytande på din lifliga själ. Må du sjelf välja, ty
hvardera sidan har ganska aktningsvärda förmågor att förete. Du skall
säkerligen träffa mera värme och entusiasm hos medlemmarne af det
specielt finska partiet, hvilket också är vida talrikare om man nämligen
räknar dertill hela den stora kohorten af egentligen principlöst folk,
men som gerna flyter med strömmen. Detta parti har i sig upptagit,
bredvid män med verkligt snille, äfven en mängd medelmåttiga förmågor,
som på fosterlandskärlekens breda och tålmodiga basis ha lätt att bringa
rökelseoffer åt sin fåfänga, emedan de i allmänhet icke ha att befara en
nagelfarande kritik och röna stort undseende för sina, såsom det heter,
i alla fall välmenta produktioner, -- en för öfrigt ganska vanlig och
äfven lätt förklarlig företeelse, i synnerhet på det litterära arbetets
område, hos unga eller hittills på ett eller annat sätt undertryckta och
förbisedda nationaliteter. På ett sådant sätt är det jemförelsevis lätt
att skära lagrar. I följd häraf räknar nu vårt finska parti ett antal
författare och litteratörer, hvilka skulle göra mera gagn om de till
exempel skulle egna sina krafter uteslutande åt folkundervisningen och
söka förvärfva sig de kunskaper som erfordras för detta högvigtiga kall,
i stället för att äflas med egna produktioner och öda sin tid på
klumpfingrade bardalekar och en med föga uddhvassa vapen förd polemik,
der elakhet får ersätta qvickhet och simpel grofhet skall föreställa
frimodighet. Du ser häraf att jag dömer strängt öfver det parti jag
sjelf allmänt anses tillhöra, men dem man älskar dem agar man.''

``A andra sidan åter'', fortfor Ros och hans ton antog en något
försmädlig skärpa, ``skall du i det svensk-finska lägret finna flera
ganska intelligenta personligheter som, eleganta så väl i sitt yttre som
ock i sitt inre, icke utan en viss anstrykning af kosmopolitism, se
sakerna, om jag så får uttrycka mig, \emph{un peu en gros}, och hvilka,
det medger jag, på sitt sätt äro goda patrioter, men som antingen icke
eftersträfva eller till och med le åt ett specielt finskt Finland. De
äro dugtiga kämpar för allt hvad frihet och rätt och socialt
framåtskridande heter, men'', och talaren blef mera lifvad, ``de brista
i den sanna fosterlandskärlekens förnämsta trosartikel:
nationalitetsprincipen. De framställa såsom sitt ideal ett dimdunkligt,
dualistiskt fantom, ett mellanting af en finsk stat med svensk kultur,
en historisk orimlighet, snarlik en stark finsk hufvudskalle med svensk
hjerna uti. Detta politiska hjernspöke tro de ha sig anvisadt en viss
mission i nordens framtidshistoria. Måhända hägrar också för deras
inbillning i fantastiska konturer något slags union med de skandinaviska
länderna -- en alltigenom oklar skapelse af de svenska sympatier som
genomtränga dem och från hvilka de icke kunna eller ens vilja frigöra
sig. Jag högaktar flera af deras ledare personligen, men jag beklagar
deras ståndpunkt, ty ett uteslutande finskt Finland är en tanke som de
icke kunna fatta.''

``Och kan du sjelf, broder Birger'', inföll Erik, ``fullt klart fatta
tanken om ett uteslutande finskt Finland?''

``Sjelfva tanken står klar för min själ'', svarade entusiasten med
blixtrande öga, ``och det är just på de `djupa lederna', såsom du kallat
dem, som denna tanke stöder sig. Det är den enda förnuftiga utgångspunkt
för att grundlägga Finlands framtid, men'', och den stolta blicken
sänkte sig och han tillade i tviflarens dofva ton -- ``men
förverkligandet af denna höga tanke har ännu hvarken funnit sin man,
eller ens sina män, ja knappast ett\ldots{}'' han tystnade tvärt och en
dyster skugga gick öfver de manliga dragen. Erik räckte sin vän handen
och tryckte den stillatigande, men inom sig tänkte han: ``Han vågar icke
säga det bittra ordet rent ut, det ordet att den stora framtidstanken
ännu icke funnit för sig ett moget folk, i stånd att af egen drift och
kraft ensamt fortsätta och befästa det påbegynta verket.'' Så tänkte den
unge svensken, men samtalet intresserade honom djupt och han ville icke
låta detsamma falla. Han yttrade derföre efter en liten paus i frågande
form: ``Och det ryska inflytandet, fruktar du inte det?'' Ros kastade en
lång blick på sin vän. Slutligen sade han i lugn ton, och det lät som en
liknelse, följande ord:

``Du har manat fram en ny demon i vårt samtal, bror Erik, men jag
bekänner uppriktigt att jag inte så mycket fruktar djefvulen utom, som
djefvulen inom mig, min egen svaghet, disharmonien i min egen
öfvertygelse. Jag tror att min sak är sann och min vilja god, men min
tro är dock inte så stark att den kan förflytta hälleberg. Jag tror att
det finska folkelementet inträdt i en helsosam jäsningsprocess, men
tviflets djefvul inom mig hviskar att hela denna process skall förkolna
i sig sjelf, om inte en liten tillsats af ett främmande element bereder
de bundna andarne tillfälle att utveckla sin inneboende kraft.''

``Och'', utropade Erik med värma, ``för att fortsätta din liknelse,
denna ovilkorligen nödvändiga lilla tillsats är -- var man, min finske
vän -- och säg rent ut, är\ldots{}''

``Är den olyckliga och dock trefaldt välsignade svenska surdegen i
Finland!'' utbrast Ros häftigt, ``men jag vill stå i egna skor och på
egen botten och bortvisar stolt alla gåfvor. `Bättre att i eget land
dricka vatten ur näfverrifva, än i främmande land dricka öl ur
krus!'\,'' {[}Ur finska nationaldikten ``Kalevala''{]}

``Manligt taladt, bror Birger'', genmälte Erik med ett vänligt leende,
``men antagom nu, med hänsyn till vår snabbt framåtilande tid, att det
är skridskor och inte genuina näfverskor du behöfver, är det inte då bra
att ha en smula svenskt bildningsstål i dem, så att du inte faller
pladask ned på den hala ryss-is hvilken du måste öfverskrida för att
verkligen komma på egen, finsk botten?''

Ros måste ofrivilligt skratta åt den skämtsamme vännens sätt att söka
skingra hans mörka tankar.

``Du skrattar?'' utropade Erik med låtsad förvåning, ``och jag som
trodde att fennomaner och läsare alldeles glömt bort att skratta.''

``Din sammanställning, bror Erik, är mera träffande än du kanske sjelf
anar. Fennomaner och läsare ha verkligen den inbördes likheten att
hvardera i många fall äro blinde ifrare, men de \emph{söka} åtminstone
sanningen och äro inte ljumma anhängare af sin sak.''

``Går likheten så långt'', frågade Erik klipskt, ``att fennomanien
liksom läseriet i sitt läger räknar talrika qvinliga medlemmar?''

``Vänta du!'' svarade Ros ovanligt muntert och ett skälmskt leende gick
såsom en ljusglimt öfver det nu dubbelt intressanta ansigtet samt tillät
en alldeles ny inblick i den unge mannens karakter, ``vänta du, det der
fordrar hämnd. Efter du så vill, skall jag införa dig i åtminstone
\emph{en} sällskapskrets med bestämd färg.''

``Bestående af idel fennomaner?'' sporde Erik med komisk fruktan. ``De
taga kanske hufvudet af mig, arme svensk?''

``Säg hellre måhända -- hjertat'', svarade vännen, ``jag lofvar
presentera dig i en liten krets af fennomanskor.''

``Hvad behagas?'' utropade svensken, verkligen förvånad. ``Fennomanskor!
Huru se de ut?''

``Jo'', sade Ros, ``de likna, såsom Runeberg säger: `ett smultron, vuxet
i skuggan'. Men få se om de skola finna nåd inför dina i den vägen af
för mycket ljus bortskämda ögon, ty de svenska fruntimren äro i sanning
de mest intagande qvinnor i verlden\ldots{}''

Här tystnade talaren och vände sig bort, en plötslig blekhet efterträdde
det nyss ännu glada uttrycket i hans ansigte.

``Du vill föra mig i en krets af finska trollqvinnor, skalk!'' utropade
Erik skrattande, "du lägger ut en snara för mig, men jag säger med den
svenske skalden:

\begin{verbatim}
"'Den höge yngling sade
Ett ord och snaran brast.'
\end{verbatim}

``Jag är ju, enligt stadssqvallret, så godt som förlofvad, ehuru mig
sjelf ohördan.''

``Med hvem?'' utbrast Ros häftigt i det han dock fortfarande vände
ansigtet åt sjön.

``Har ingen fara ännu'', ljöd Eriks svar; ``bara prat, ingen sanning,
det vet jag väl sjelf bäst. \emph{Hon} vill säkert inte ha mig och jag
förspörjer ingen lust att allaredan gifta mig.''

``Men hvem vore då den lyckliga?'' frågade Ros åter.

``Den lyckliga?'' utbrast Erik: ``Javäl, så lycklig som skönhet och ett
godt hufvud kunna göra en sjelfrådig trollslända till flicka -- det är
den rika arftagerskan, resenärskan, svärmerskan, tjuserskan,
gud-vet-allt-hvädskan, och vore hon i Finland, säkert också
fennomanskan, med få ord ingen annan, än min näpna kusin Jenny Bertram,
med hvilken juvel ett välvilligt rykte\ldots, men kors hvad felas dig?
Du är ju alldeles hvitgrå i ansigtet, jag tror minsann att du är
sjösjuk\ldots{}''

``Jag tror, jag går i min koj'', mumlade Ros och nästan vacklade fram
till kajutdörren.

``Den Jenny'', sade Erik förtretad för sig sjelf, ``hon spelar mig då
alltjemt något spratt, till och med sig sjelf ovetande. Nu blir just vid
nämnandet af hennes namn den der kamraten sjösjuk. Fatalt! Och jag som
redan hade fått den kalla finnen halfvärmd. Det minsta jag räknade uppå
vår ett namngifvande af de förnämsta skönheterna i Helsingfors. Jag
hoppas likväl att bara vi komma fram, jag ändå äfven utan katalog skall
få rätt på någon del af herrligheten. Fennomanskor -- och `ett smultron,
vuxet i skuggan' -- ganska bra, men jag skall minsann göra allt mitt
till att öfver de der finska smultronen utgjuta hela det briljanta
solljuset af min stockholmska konversation. Det blir kanske rätt pikant
till slut. Kypare, sodavatten och ett glas punsch!''

\begin{enumerate}
\def\labelenumi{\arabic{enumi}.}
\setcounter{enumi}{5}
\tightlist
\item
\end{enumerate}

En helsning från Finland.

Vi befinna oss åter i Stockholm. Tant Agatha och fröken Jenny sitta vid
sitt morgonkaffe; då ringer postbudet och kammarjungfrun lemnar fram ett
bref med utländsk poststämpel.

``Minsann, från Erik!'' sade Jenny, sedan hon brutit detsamma och
flygtigt sett på stilen. ``Vill tant höra på, så skall jag läsa upp hvad
den gunstige herrn har att förmäla från Finland'?''

``Men om der kanske förekomma några hemligheter\ldots{} så\ldots{}''

``Åhnej'', svarade Jenny leende, ``jag har inga hemligheter med kusin
Erik, fast jag tycker bra om honom.''

Och hon uppläste följande skrifvelse.

\begin{verbatim}
_Erik Stenrot till Jenny Bertram_.

Helsingfors, 15 Juli 1864.

Min skälmska kusin!

1 den förmodan att du ännu icke anträdt din "inkognito-resa"
sänder jag dig dessa rader, hvilka, jag hoppas det, skola finna
tant och dig sittande i allsköns välbefinnande i er vackra våning
vid Drottninggatan i det oförlikneliga Stockholm.

Ja, jag är nu i Finland och i dess glada hufvudstad Helsingfors,
som är en vacker och prydlig stad med breda gator, stora torg
och präktiga promenader samt ett starkt tycke af ungdomlighet i
hela sin fysionomi, hvilken icke kan förfela att göra intryck på
hvarje, i synnerhet sjövägen ankommande, resande; men mera härom
då jag åter befinner mig på hemlandets jord och, sittande i den
beqväma gungstolen i edert så kallade hvardagsrum, kan kasta en
lugn återblick på mina reseminnen, hvilka sannolikt skola bli
ganska angenäma. Jag har också allaredan börjat föra en dagbok,
för att ha en ledtråd för minnet då jag i framtiden någon gång
vill återupplifva de mångfaldiga intryck min själ här rönt.

Såsom du vet reste jag hitöfver för att studera det "nya lifvet"
i vårt gamla kära Finland. Dessa mina tillämnade studier ha
blifvit inledda på ett ganska egendomligt sätt och jag befarar
högeligen att de i många afseenden skola erbjuda stora luckor,
medan åter vissa andra partier komma att intaga ett måhända
alltför framstående rum.

Jag vet icke om det var dig bekant att vår gemensamme vän från
de skotska högländerna, den intressante finnen Birger Ros, som
någon tid vistats i Stockholm men icke kunde förmås att besöka
tant Agatha och dig, var min reskamrat hit. Nåväl, han är lika
originel som förut och jag har till och med trott mig märka att
ett visst oförklarligt svårmod hvilar öfver hela hans väsende.
Åtminstone äro hans kinder nu betydligt mera bleka än de voro
bruna då vi tillsammans med honom gjorde våra bergvandringar
i Skotland. Huru det nu än må vara, så spelar imellertid
Ros en stor roll i den korta historien om de två veckor jag
allaredan vistats i Helsingfors. Han, som åtnjuter mycket
anseende härstädes, har presenterat mig för en mängd litterära
notabiliteter och jag kunde ibland mina nya bekantskaper här
anföra äfven i Sverge välklingande namn, men sparar äfven detta
till framdeles och nämner blott att jag på mycket vänskaplig fot
umgås såväl med flera framstående medlemmar af det svenskfinska
intelligenspartiet, som ock ledarne af den stora fennomanska
falangen, hvilken med all makt sträfvar att arbeta sig upp till
samhällets höjder. Detta partis gamle hjelte, den välbekante
Johan Vilhelm Snellman, innehar också allaredan ett ganska högt
embete; han är nämligen chef för finansväsendet i Finland,
hvilket, i förbigående nämdt, lär vara i godt skick. Utom dessa
nöjsamma men främst dock för mina studier nyttiga bekantskaper
har jag att tacka vännen Ros äfven för några andra som äro
alldeles förtjusande. Hvem kunde tro det "fattiga Finland" om
sådana skatter! Du som har ett godt hufvud kan väl gissa att jag
här menar de finska fruntimren. O, de äro hänförande! Och när de
framsjunga sin i alla fall ganska korrekta svenska, låter det
rätt pikant och man tror sig vara på sirenernas ö. Ja, nu kan
jag förstå hvarföre hjelten Lemminkäinen i Kalevala-dikten så
länge qvardröjde på den af sköna ungmor bebodda ön, fängslad i
deras krets liksom genom en trollmakt. Såsom du ser börjar jag
redan göra mig hemmastadd i den finska folkpoesien, men så har
jag också lärarinnor som icke allenast med ord, utan äfven med
af inspiration strålande blickar för mig förklara det finska
språkets skönheter. Du må le, men hvad jag säger är på sätt och
vis bokstaflig sanning. Mig har nämligen vederfarits det gästvän
lighetens ynnestbevis att erhålla tillträde till en klubb af
-- fennomanskor. Har du någonsin i Stockholm, Kjöbenhavn eller
Kristiania hört talas om något sällskap af _fornnordiskor_,
hvilka beslutit att vid sina sammankomster endast tala sitt lands
ursprungliga folkspråk? I detta afseende har den finska bildade
qvinnan ådagalagt större energi. I den klubb, jag omnämt, tala
nämligen alla medlemmarne uteslutande blott finska med hvarandra.
[Faktiskt. Detta sällskap eger ännu i dag (1872) bestånd i
Helsingfors.] Herrar äro ingalunda uteslutne från sällskapet,
men jag för min del ger företräde åt de qvinliga medlemmarne.
Jag skall försöka beskrifva mitt något komiska första inträde i
denna krets af entusiastiska förtjusarinnor. Ros hade skämtvis
lofvat införa mig i en krets af "fennomanskor", men jag tog
honom på ordet och en vacker onsdagsafton begåfvo vi oss af till
församlingslokalen. Jag fick vänta en liten stund i ett yttre
rum, sedan åtföljde jag Ros till en väl upplyst, enkel salong der
ett talrikt sällskap fruntimmer och herrar var församladt. De
på finska muntert samtalande grupperna gåfvo, artigt besvarande
mina helsningar, leende plats åt oss, der vi framträdde till
ordföranden för qvällen, professorskan Z., för hvilken jag
presenterades. Det var en fin och behaglig företeelse, den vackra
professorskan. Hon höjde sin lilla, fennomanska presidentklubba,
slog ett lätt slag på bordet och höll på finska språket ett litet
tal af, såsom jag sedan fick veta, ungefär följande innehåll:
"Som den för henne nyss presenterade herr Erik Stenrot ('Eero
Kivijuuri' lät mitt fennomanska namn) önskade bli i sällskapet
upptagen, men den ärade gästen från Sverge ännu icke var mäktig
finska språket, så ville hon hemställa till medlemmarne, huruvida
det icke för denna afton skulle tillåtas att i samtal med honom
begagna sig af svenska språket; dock skulle detta medgifvande",
tillade hon skälmskt, "endast gälla föreningens qvinliga
medlemmar. Den som bifaller härtill upplyfte handen."

Ett allmänt bifallssorl följde och under muntert skratt upplyftes
en mängd täcka händer rakt i höjden. Jag tror nästan, förlåt mig
Jenny, att somliga unga damer voterade med båda händerna. Den
intagande professorskan meddelade mig nu leende att herrarne
derföre blifvit uteslutne från rättigheten att tala svenska,
emedan fruntimren, åtminstone den ena qvällen ville allena få
disponera öfver den välkomne svenske gästen.

Sådant var mitt inträde i den helsingforsska societeten.

Jag innesluter här några blad ur min dagbok, hvilka jag icke ens
hunnit genomläsa. Ursäkta derföre stilens vårdslöshet. Jag tror
dagboken börjar med tredje dagen af min härvaro. Helsa tant och
lef väl. Eder _Erik_.

_P.S._

Aina Ros är ovilkorligen ett af de intelligentaste fruntimmer jag
sett. Hon är utmärkt vacker och lika munter som hennes bror är
allvarlig.

   *       *       *       *       *
\end{verbatim}

Jenny slutade läsningen af brefvet, men gömde ``dagboken'' till lektyr
för sin egen räkning.

``Men det är ju oförsvarligt, Jenny lilla'', utbrast tant Agatha, ``han
talar ju bara om den der Ros och sina finskor i hela brefvet. Har han då
alldeles glömt bort dig? Men vänta bara, jag skall skrifva till honom en
epistel, jag, och\ldots{}''

``För all del, snälla tant, låt bli det'', afbröt Jenny och försjönk
såsom det tycktes i djupa tankar. Det var väl ändå något i det der
brefvet som smärtade den unga damen, men denna sinnesstämning räckte
åtminstone till det yttre icke länge.

``Sade inte morbror Stråle i går att alla mina papper och vexlarne äro i
ordning?'' frågade Jenny.

``Ja, kära du'', sade den beskedliga tanten, ``och vi kunna resa när som
helst. Jag följer, vet du, denna gång riktigt gerna med.''

``Nåväl tant, då resa vi i morgon'', blef Jennys bestämda svar.

Tanten nickade tyst bifall; hon var van att följa sin kanske något
nyckfulla, men i alla fall mycket älskvärda systerdotters vilja. --
Innan hon gick till hvila framtog Jenny Eriks dagboksanteckningar och
läste dem icke utan intresse. Vi meddela desamma våra läsare här straxt
nedanföre.

\begin{enumerate}
\def\labelenumi{\arabic{enumi}.}
\setcounter{enumi}{6}
\tightlist
\item
\end{enumerate}

Om fennomanskor och fennomaner.

(Dagboksanteckningar af en svensk.)

Helsingfors, 1 Juli kl. 12 på natten.

Hemkommen från min första fennomanska soirée försökte jag visserligen
att genast gå till hvila, men detta var mig omöjligt. Jag har derföre
beslutit att \emph{arbeta}. Detta kommer väl kanske att i någon mån
förvåna min värde bolagskamrat, filosofie och juris utriusque doktorn
Birger Ros, men han blir säkerligen icke nyfiken af att få del af hvad
jag författar. Medan han nu sitter i sitt eget rum, tänder jag en cigarr
i afsigt att en stund promenera i salen och sedan jag lyckats samla och
någorlunda ordna mina minnen anförtro desamma åt papperet. -- Att
företaga detta göromål just nu anser jag mig vara skyldig min vackra
kusin Jenny, på hvilken jag icke tänkt hela qvällen och som ännu icke
erhållit något bref af mig och icke heller erhåller ett sådant förr än
jag blifvit förtrognare med lifvet i den finska hufvudstaden. Men hvem
kan undra derpå, då man inom så kort tid blifvit bekant med så många
jordiska englar, som i synnerhet fallet i qväll varit med mig. Då jag
presenterades för de unga damerna nedböljade i ljufligt klingande
namnkaskader skönhetens välsignelse öfver mig: Hanna och Anna; Lilly,
Inga och Ellen: Laura, Olga och kanske betydelsefullast af alla, ditt
blida namn, Aina Ros. {[}Aina betydet på finska: alltid{]}. Glömd var i
eder närhet min allvarliga föresats att söka bekämpa det finsk-finska
partiets afvoghet mot det svenska elementet, en afvoghet som tydligt nog
genomskimrade några af herrarnes ofta trumpna artighetsbetygelser. Glömd
var hvarje sådan föresats i qväll, ty tjusande var din sång, mörkögda
österbottniska Inga, och mitt hjerta fylldes af nattviolsdoft, milda
Hanna, vid din drömmande, tavastländska blick. Laura, ditt själfulla
leende kom mig att glömma det finska klubbekrigets herrehat och din
syster Ellen var som en treflig saga af Topelius, der lyckligtvis
sluttillämpningen var bortglömd. Jag förlät Yrjö Koskinen {[}professor
Georg Forsmans antagna finska namn{]} hans senaste utfall mot
``svenskheten'', då du talade om Saimasjöns under, blåögda berätterska
Anna, och ehuru jag blott flyktigt såg dig, blomma från Ladogas fjerran
strand, svärmiskt smäktande Olga, så vacklade dock min tro att det sköna
har sitt enda hemland i vester. Slutligen satte Kuopiotärnan Lilly
kronan på ert finska omvändelseverk med mig, ty jag börjar ta ganska
förmånliga föreställningar om de kalevalitiska skönheternas behag. Men
Aina Ros var dock den jag mest tänkte och ännu tänker uppå. -- Se så, nu
har jag ju varit rätt snäll emot kusin Jenny, ty jag ämnar sända henne
denna kortfattade finska blomsterkatalog, och derföre vill jag nu sotva
de rättrådiges sömn. -- Godnatt, Birger! Hvad skrifver du der i nattens
tysta timme? Är det något nytt anslag emot ``svenskheten'', du
``finskhetens Orvar Odd?''

Den 3 Juli.

Jag har i svensk öfversättning läst hvad Ros skref natten till i går.
Det är en vacker dikt, ett slags genmäle på den finsk-finska professorns
något vilda utfall emot ``svenskheten'' i Finland, en sång, hvari
författaren talar ett varmt och ädelt språk till den ``svenske
brodern''. -- En underlig man, min vän Ros: än en hänsynslös fennoman,
än åter \emph{menniska framför allt}. Jag börjar tro med honom sjelf att
han är ett rof för en svårtbekämpad inre disharmoni, och jag ville
nästan likna honom vid en fältherre som när han rustar sig att i spetsen
för sin finska falang rusa åstad till kamp är grym att åse och höra, men
som, när stridens timma slår och hans kamrater och soldater liksom
\emph{ex officio} sluta visiret för att i blind yra ``rida spärr'',
sjelf aftager hjelmen och visar ett anlete, hvari man läser sorg öfver
den beklagansvärda brödrastriden, hvilken utföres med lika mycken
öfverspändhet å fennomanernes, som lugn värdighet å de andra finnarnes
sida.

Ros är bestämdt icke belåten med sig sjelf, sin ställning och
verksamhet.

Den 4 Juli.

En sammansättning af motsatser är denne min vän Ros. I går afton kom han
hem i vredesmod. Han hade bevistat en sammankomst i och för bildandet af
ett bolag, som skulle grundlägga en större tidning på finska språket.
Han tadlade i skarpa ordalag den ljumhet för det stora företaget som
röjt sig på alla håll. Litterära bidrag utlofvades af nog många och till
och med utan ersättningsanspråk, men aktieteckningen för att
tillförsäkra en sådan folkorgan ett flerårigt bestånd och följaktligen
den enda utsigten till verklig framgång, hade rönt ringa uppmuntran,
såväl i landsorten som ock i Helsingfors. Hela den vackra planen måste
således förfalla till hans stora ledsnad, ty Ros' ekonomiska
omständigheter tillåta honom icke att af egna medel bestrida företagets
alla kostnader, ehuru han dock tecknat sig för ett icke obetydligt antal
aktier i det tillämnade bladet, så svårt detta än blef honom efter de
utgifter hans vidsträckta resor erfordrat. -- I dag åter, då min vän
erhöll ett i mycket smickrande ordalag framstäldt anbud att öfvertaga
redaktionen af den på finska utkommande officiella tidningen samt dervid
löfte om ``full frihet att ge bladet hvilken anda och riktning han
ansåge mest gagnelig för landet'' -- afslog han med indignation denna
honom erbjudna utväg att på dess eget språk få tala till hela det finska
folket.

``Jag'', utropade han, ``en regeringens organ -- nej, hellre vill jag
reda mig en andlig graf och iakttaga en tvungen tystnad, ty politisk
tystnad i detta öfvergångsskede är politisk död. -- Men hellre det, än
lägga en hand vid verket att insöfva mitt folk i sjelfbedrägeriets
villa!''

Och tidpunkten är verkligen lämplig till en utvidgad och betydelsefull
publicistisk verksamhet i Finland, ty den preventiva censuren blir från
år 1865 upphäfven för tre år, eller intill nästa landtdag (1867) -- ett
försök ifrån regeringens (i Ryssland) sida att pröfva huru finnarne
skola bära ett slags tryckfrihet, eller riktigare en åtminstone
\emph{lagligen} icke \emph{godtyckligt}, såsom hittills, inskränkt
yttranderätt. Nu skulle den modige mannen åtminstone \emph{kunna} säga
ett och annat sanningsord. Så mycket bittrare för Ros, att företaget med
den stora, dagliga finska tidningen icke blifvit verklighet. Han lider i
sanning mycket.

Den 6 Juli.

Victoria! Ros har segrat. Hans glödande entusiasm har förmått ett antal
fosterlandsvänner, ifriga fennomaner för öfrigt, att grundlägga en finsk
``tidskrift för litteratur och ekonomi''. Det är åtminstone en lofvande
början till en framdeles utvidgad publicistisk verksamhet i sannt
fosterländsk anda. Ros skall upprätta programmet. Jag är verkligen bra
nyfiken att få del deraf, ty jag är öfvertygad om att Ros denna gång
skall lyckas öfvervinna all ensidighet och se sakerna i stort.

Den 10 Juli.

Ros' program till den nya finska tidskriften har blifvit förkastadt,
emedan han vägrade att derur stryka följande ord:

``Vi skola öfvertvga våra svenska talande landsmän derom att äfven deras
finska bröder hysa ett lefvande deltagande for tidens högsta frågor och
att de äro och känna sig fullt ut värdige att hand i hand och i förbund
med svensk-finnarne arbeta för det gemensamma fäderneslandets bästa. Vi
inse fullkomligt nödvändigheten deraf att sjelfva hufvudmassan af folket
bör förena med sig den i landet förhanden varande svenska bildningens
krafter, för att med framgång kunna verka för fosterlandets sanna väl.
All ensidig parti-hätskhet vare derföre bannlyst ur våra spalter i det
vi i det \emph{eniga} Finlands namn höja det sanna framåtskridandets
baner.''

Ultrafennomanerne lära ha sjudit af raseri öfver uttrycken ``de svenska
bröderne'' och ``förbundet'' med dem, äfvensom ``nödvändigheten att med
sig förena den svenska bildningens krafter''. Mycket bittra ord ha
blifvit fällda emot Ros, som likväl icke drog sig tillbaka från
aktieteckningen, men väl utträdde ur organisationskomitén. -- Vi
tillbragte aftonen hos hans syster, som är lärarinna vid ett här
inrättadt barnhem för flickor samt jemte föreståndarinnan bor invid den
enkla, men rymliga lokalen. Hvad som isynnerhet tycktes ha sårat Ros var
ett tal som en af hans hemliga afundsmän hållit vid bolagsstämman och
hvarvid han med bitterhet utpekat Ros såsom en ``affälling från den goda
saken'', tilläggande med oblygt hån: ``och denne man som i sitt så
kallade program uttalat så svenskvänliga åsigter har dock en gång i sin
hädangångne faders, den äkta patriotens hand, vid dennes död till och
med aflagt en helig ed att aldrig taga till äkta dottern af ett
främmande folk! Och hvad är det väl som han nu föreslår oss att göra, då
han vill att vi skola förmäla vår finska sak med svenskarnes i landet?''
-- Jag medger att den ed Birger afgaf åt sin döende fader, denne fader
som i sitt slag lär ha varit en riktig finsk Cato, var en öfverilning,
men det är i alla fall skändligt att på detta sätt bli påmind om ett
löfte, som det nästan var omöjligt att vägra och hvilket sannolikt
aldrig i bokstaflig mening torde komma att sättas på prof. -- Aina var
förtjusande i sin milda vänlighet mot den djupt kränkte brodern och jag
försökte skämta öfver det der löftet, men detta tycktes blott, eget nog,
öka Ros' dystra sinnesstämning. Jag upphörde derföre i tid dermed,
isynnerhet som Birger vid vår vänskap besvor mig att icke vidare tala om
hela saken.

Den 12 Juli.

Vi, det vill säga Birger, hans syster och jag, ha beslutit att för en
tid lemna Helsingfors och aflägga ett besök på Ros' egendom i östra
Finland, der dessutom affärer påkalla hans närvaro. Vi afresa i morgon
och taga, för att förströ molnen på Birgers panna, den stora omvägen
öfver Tavastehus och St Michel till Nyslott, samt derifrån öfver den
berömda Pungaharjuåsen till Imatra vattenfall, i närheten hvaraf Ros'
egendom, Muistola (``Minnets hem''), är belägen. Sålunda får jag se ett
godt stycke Finland och det i det angenämaste sällskap jag kunnat önska
mig. Vår resa till Muistola torde komma att upptaga tre å fyra veckor. I
September återkomma vi till Helsingfors\ldots{} och sedan?

\begin{verbatim}
   *       *       *       *       *
\end{verbatim}

Här sluta Erik Stenrots ``dagboksanteckningar'' och som nu alla våra
bekanta begifvit sig på resor, följa äfven vi deras exempel.

\begin{enumerate}
\def\labelenumi{\arabic{enumi}.}
\setcounter{enumi}{7}
\tightlist
\item
\end{enumerate}

Ödemarkens lif.

Vår berättelse tvingar oss att lemna det vackra Stockholm och äfven dess
ännu blott halfvuxna kusin ibland nordens hufvudstäder, det glada
Helsingfors. I stället för salongslif i glacéhandskar, musikaliska
aftonunderhållningar samt folkhvimlet i Stockholms Strömparterr och
Helsingfors' esplanad, bjuda vi läsaren på raka motsatsen af allt detta.
Dit vi föra honom, der skina nu väl sol och måne också och måhända till
och med i ännu renare majestät än öfver de stora städernas prunkande
palatser och bullersamma gator, men ehuru der är lif och rörelse, äfven
der, så har dock icke menniskans vilja framkallat dessa tillvarons
yttringar, utan krafter, hvilkas anor gå ännu längre tillbaka i tiden,
än de öfver sin paradisiska härkomst så stolta adamiternas. I ödemarkens
verld, der de ännu otyglade naturkrafterna ha sin fria tummelplats och
lefva sitt oförfalskade lif, i denna hemlighetsfulla verld gripes
vandrarens bröst till en början af ensamhetens ängslande inflytande, ty
han fattar blott dunkelt elementernas ursprungliga och allvarliga språk,
detta naturens urspråk som i otaliga dialekter bär ett vittnesbörd om
samma skapares allmakt, ur hvars outgrundliga väsende äfven menniskan
framgått, hon, hvars planer och verk blott äro stora då de mätas med
hennes egen uppfattnings måttstock. Vill du stanna här i denna underbara
ensamhet, vänlige läsare, lära dig förstå skogsverldens rika mångfald
och uppbygga din själ vid dess högtidliga lofsång öfver alltings upphof?
Eller bäfvar du för ensamheten och tystnaden rundtomkring dig? Du
förstår ännu icke deras betydelsefulla språk och derföre bleknar du vid
tanken på ett sådant eremitlif, såsom du kallar det på ditt
menniskospråk.

Nåväl, jag vill visa dig en annan tafla ur naturens stora lif; kom, följ
mig till den brusande forsens brädd. Bäfvar icke din själ tillbaka för
denna dialekt af naturens eviga tungomål? Fruktar du icke det oafbrutna
dånets tordönsljud?

Här står du vid fallets fot, höj din blick och skåda upp till detsamma.
Framför dig upprullas en tafla af det naturkrafternas väldiga
gladiatorspel som kallas Imatra. Från de trotsiga klipporna återstudsa
de i yrande fart framrusande jätteböljorna med krossade, skumomhöljda
pannor -- en storartad kamp, evigt förnyad med outtröttliga krafter, ty
det är sann natur, ett ståtligt stycke gudaverk i denna skapelse.

För dem af våra läsare som icke sjelfve sett denna, onekligen den
största fors i Europa, må följande tjena såsom ett slags beskrifning.
Den stora Wuoksenelfven, genom hvilken Saimas vattensystem uttömmer sig
i Ladoga, sammantränges ungefär en half mil från sitt utflöde ur
förstnämda sjö emellan höga klippväggar och bildar i den genom graniten
sprängda, smala rännan en fors, hvars vattenmassor med åsklikt dån i en
enda lavin af nedstörtande skum med otrolig hastighet och vittnande om
en oerhörd kraft ila att uppnå ett rymligare becken. Stället är beläget
ungefär sex mil ifrån staden Wiborg i sydöstra Finland samt i en ödslig,
vid elfstränderna romantisk trakt. Forsens hela längd torde vara
trehundra famnar och fallets höjd ungefär sextio fot. För att rätt
förstå detta naturskådespel måste man se det flera gånger. Det första
intrycket är ganska egendomligt. För den från forsens fot uppåt det
långa skumbandet blickande åskådaren förefaller det i början som stode
hela denna massa alldeles stilla, och först sedan ögat en stund vant sig
vid den imposanta synen urskiljer detsamma den hastiga rörelsen framåt i
de bokstafligen öfver hvarandra störtande och liksom sig sjelfva
uppsväljande vattenhvirflarne. Ett moln af fina vattendunster sväfvar
öfver det i sina granittjettrar vildt rytande elementet och solens
strålar bryta sig i alla regnbågens färger i de genomskinliga, af och
till böljande dunstmolnen. När månen en klar Augustinatt kastar sitt
trollsken öfver Imatra, då är det som skulle vidunderliga luftgestalter
sväfva öfver detta dånande kaos, än kommande den hänförde åskådaren
alldeles nära och liksom erbjudande honom ett famntag med löfte om evig
ro dernere i det evigt oroliga djupet, än hastigt vikande tillbaka
liksom vid menniskans åsyn uppskrämda elementarandar eller måhända
flyende skuggor af de olycklige som i Imatra sökt och funnit sin graf,
men hvilkas själar dock ingen ro hafva, änskönt deras kroppsliga hyddor
krossats. Tid efter annan tyckes forsen fordra en sådan tribut af den
öfriga verlden och sägner om sorgliga tilldragelser af denna art äro
mycket gängse i trakten. -- Icke allenast talrika resande från hela
Finland utan äfven från S:t Petersburg samt andra utländingar besöka om
sommarn Imatra, hvarest vid tidpunkten för vår berättelse ett godt hotel
var inrättadt. Men industrien hade icke då ännu, hvilket numera är
händelsen, här uppfört fabriker af hvarjehanda slag. Då var Imatra ännu
blott målet för skådelystne resandes och naturbeundrares besök.

Det är till detta ``berömda ställe'' vi nu fört läsaren och läsarinnan.
Det är förmiddag och få eller inga besökande tyckas ännu ha anländt den
dagen. Osedda, såsom författare och läsare städse äro för de i en
berättelse uppträdande personerna, gå vi framåt och, se der, vi mötas
snart af en blid företeelse i det vilda fallets omedelbara grannskap.
Behagfullt lutad emot ett klippblock står ensam en ungdomlig
qvinnogestalt, försänkt i betraktande af forsens nedvältande
vattenlavin. Jag ser, sköna ungmö, att ditt hulda anlete förklaras i
stum beundran och att din blick med ett otolkbart uttryck i sin varma
glans följer de brådstörtande hviflarnes dundrande tåg framåt. Jag ser
att din själ tjusas af hvad dina sinnen förnimma, och forsens språk tror
du dig förstå bättre än den ensliga skogens sus. Här fattar dig naturens
storhet omedelbart, utan att din tanke först väckes, du är
öfverväldigad, sköna betrakterska, -- hela din själ har samlat sig i
ditt öga och du bildar, tillsammans med den skummande forsen, en herrlig
tafla för min inre syn: behagen, beundrande den otämjda naturkraften. I
ett målningsgalleri skulle jag måhända kalla denna bild ``Gracen och
titanen'', men här måste jag sanningsenligt nämna den vackra
betrakterskan vid hennes menniskonamn, ty det är -- Jenny Bertram vid
Imatra fall.

Ja, så långt hade de kommit på sin nu förverkligade ``inkognito-resa'',
och tant Agatha satt på en klipphäll icke långt ifrån det ställe der vi
sett skön Jenny stå. Tanten tycktes vilja mera med örat än ögat egna sig
åt njutningen af det vilda skådespelets storhet, som åstadkom en för
henne alltför stark nervskakning. Hennes hufvud svindlade då hon
blickade ned i hvirflarne, hade hon sagt, och den goda damen hade ganska
rätt i hvad hon sade, ty detsamma har händt och händer mera än en af de
talrika Imatra-besökande.

\begin{enumerate}
\def\labelenumi{\arabic{enumi}.}
\setcounter{enumi}{8}
\tightlist
\item
\end{enumerate}

Inkognito resande.

Jenny och tant Agatha hade anländt till trakten för ungefär en vecka
sedan. De bodde icke på hotellet utan en half mil derifrån på den andra,
östra sidan af elfven, der de, för den återstående delen af sommaren,
genom den artige hotelvärdens bemedling, åt sig förhyrt det vid forsens
början belägna, af sin egare nu icke bebodda, vackra landstället Ojala.
Härifrån företogo de dagligen promenader utmed forsens östra sida, men i
dag hade de beslutit sig för en utflygt till den vestra,
``paviljongsidan'' af Imatra, som i allmänhet erbjuder en mera storartad
vy af foren. Våra damer hade passerat Wuoksen vid Siitola färja och
derifrån, under många suckar från tant Agathas sida, medelst
infödingarnes i trakten vanliga fortskaffningsmedel, en tvåhjulig kärra
förspänd med en häst, begifvit sig till fallet, der de hade för afsigt
att tillbringa hela dagen. Värden på hotellet, hvilken i yngre år varit
kammartjenare hos en grefve, var sjelfva artigheten personifierad, ty
han ansåg damerna på ``Ojala'', hvilka icke kunnat ditflytta utan att
väcka ett visst uppseende i den aflägsna landsorten, för ``ett mycket
förnämt folk''. Han var svensk till börden ooh bibehöll ``midt i
bondlandet'' sitt svenska uttal. För öfrigt respekterade han på det
högsta herrskapets ``inkognition'' ehuru han inom sig beklagade detta
förhållande, ty han visste af erfarenhet att äfven de förnäma då kunna
lefva mera ``oskinneradt''. Hvarföre de svenska fruntimren just hyrt
Ojala, kunde han icke få reda uppå, trots alla sina diplomatiska frågor,
och han kom slutligen för sig sjelf till den slutsatsen att antingen var
unga fröken ``alldeles tokig i vattenfall'', eller ock hade han fått i
sitt grannskap ett ganska hemlighetsfullt herrskap, måhända polska
flyktingar, fast de talade så bra svenska. Det var under den polska
uppresningens sista dödsryckningar. Kanske att mannen, den polske
grefven, stupat och familjen, ty att ``frun'' var svenska det var han
säker på, återvände nu till överge. ``Ja, ja'', så tänkte herr
Pettersson, ``dom ä' allt polska `emigramanter', det börjar jag nu sätta
all tro till. Och länsmannens frågor i går, hvad kunde de väl betyda?
Men jag är sjelf svensk och dom är förnämt folk; jag vill inte förråda
dom, men jag skall vid middagen ge dom en fin vink om saken. Imellertid
skall jag gå åstad och säga till att bordet är serveradt.'' Och herr
Pettersson gjorde så.

En kort stund derefter syntes tant Agatha och Jenny långsamt
promenerande komma genom ``parken'', såsom en liten inhägnad skogsdunge
invid ``hotel-paviljongen'' benämdes. Vid middagen, som var ganska
smakligt anrättad, passade herr Pettersson sjelf upp. Till hans
synnerliga belåtenhet sade herrskapet till om en butelj godt, lätt vin.
En viss fruktan deröfver att ``den goda middagen eftersom det var bara
fruntimmer'' skulle aflöpa utan vin, på hvilken artikel vår värd
förtjenade mera än på den feta laxforell, den kycklingsstek och de
delikata hallonen som serverades, hade ett ögonblick uppfyllt hans
ganymedessjäl med oro och det verkligen mera för det goda skenets skull
än förtjenstens. Nu var dock allt såsom det skulle vara och icke utan en
smått löjlig högtidlighet ställde herr Pettersson fram en butelj
``Chateau Yquem'', det fina, doftande franska vinet. Med en åtbörd af
oefterhärmligt behag och en enkel artighet som icke tillät någon
misstydning frågade vår lilla fröken om icke deras snälle herr
``kommissionär'', så kallade Jenny alltid hotelvärden, ville göra dem
nöjet att sjelf smaka ett glas af det ypperliga vinet. Pettersson,
smickrad af kommissionärstiteln, hvilken han gaf utsträckning af
``kommissarie'', blef hänförd af det nedlåtande anbudet. ``Alldeles som
salig grefven gjorde på resor'', tänkte före detta kammartjenaren och
hans hjerta svällde af välvilja för ``gentila'' middagsgäster. ``Nu
eller aldrig skall jag säga dom allt'', mumlade han för sig sjelf i det
han ställde det gröna, slipade vinglaset på kanten af bordet. Jenny slog
leende uti af den doftande, ljusgula drufsaften. Fruntimren drucko
vänligt herr Pettersson till och denne smakade med välbehag på vinet,
samt ställde glaset blygsamt längst bort vid bordsändan.

``Vigtiga nyheter från Polen'', började vår Imatradiplomat, ``det står i
tidningarne att `insurschanterne' förlorat en stor batalj.''

Damerna utbytte en blick sinsemellan, men ehuru de befunno sig så nära
ryska gränsen, kunde den frimodiga Jenny dock icke underlåta att yttra:
``Stackars Polen!''

``Herre min Gud!'' utropade Pettersson, ``om jag på något sätt sårat, så
var det emot min vilja, men se det är så att, att om deras nåder ha
slägtingar der i landet, så\ldots{}''

``Hvad då?'' frågade den yngre af damerna nyfiket.

``Så då, då'', stammade hotelvärden, ``se våran länsman\ldots{}''

``Hvad då?'' upprepade Jenny sin fråga med en min af komiskt allvar, ty
hon började nästan ana hvad det gällde.

``Herre Gud i himmelen!'' utropade nu den olycklige värden, som genom
den öppna dörren hade fri utsigt åt gården, ``när man talar om den onde
är han inte långt borta, der kommer han nu i egen person!'' Och med en
rörelse snabbare än vi kunna omtala densamma skyndade han fram till
Jenny och hviskade: ``Varen lugna, jag skall försöka att arranschera
allt till det bästa.'' Han ilade ut.

Nu sågo tanten och vår vackra, käcka hjeltinna något förvånade på
hvarandra.

``Det måtte vara något löjligt missförstånd i det här'', utbrast Jenny
och försökte skratta.

``Blott vi inte få något att skaffa med den rysliga ryska polisen!''
hviskade tant Agatha, ``ack barn, barn, nu ser du huru det går till i
Finland.''

``Annu har då ingenting händt, som kunde vara oroväckande'', sade Jenny
med återvunnen fattning. ``Se så, der ha vi dem, var blott lugn, tant
lilla, och låt mig föra ordet.''

Dörren, som värden stängt igen efter sig, uppläts och med bondaktig
vigtighet i sina åtbörder, inträdde en herreman med rödblommiga kinder
och plirande ögon. Han var iklädd ett slags uniformsrock och på hans
mössa, hvilken han icke aftog förr än han gjort några steg inåt rummet,
glänste en kokard. Det var länsmannen; han åtföljdes af Pettersson. Den
höge funktionären bevärdigade de närvarande fruntimren till en början
icke ens med en blick utan aftog långsamt sin mössa och läggande den på
ett sidobord, så att den ofvahnämda kokarden var fullt synlig,
tilltalade han värden. Han kom härvid att vända damerna ryggen. Troligen
ansåg han detta sätt att uppträda fullt enligt med sin
tjenstemannavärdighet.

``Pettersson'', frågade länsmannen i nedlåtande ton, ``är här mycket
resande i dag?''

``Endast några få,'', svarade denne, ``men annars har det varit mycket
besök i år.''

``Jag tror vi taga våra sjalar och sätta oss i lilla paviljongen'', sade
Jenny.

Damerna stego upp, hvilket föranledde länsmannen att något tvärt vända
sig om. Jenny hjelpte tanten med sjalen och de togo några steg åt
dörren.

``Vi önska dricka kaffe i lilla paviljongen dernere,'' sade Jenny, med
en lätt helsning, till Pettersson. Denne bugade sig. Länsmannen, som,
trots sin påtagna myndiga uppsyn, blifvit något förlägen för den
värdighet i hållning och skick, som damerna ådagalade, uraktlät att
helsa, och förargad öfver sitt eget oskickliga beteende, öfvergick han
med en hos obildade men egenkära personer ofta inträffande vändning från
tafatt förlägenhet till framfusig påflugenhet; de tro väl detta höra
till goda ton.

``Herrskapet är från utlandet, tror jag?'' sade han, helsande med löjlig
nedlåtenhet. ``Nå, hvad tycks om Imatra? Vacker utsigt här, inte sannt
och\ldots{}''

``God middag!'' svarade Jenny alldeles allvarsam, neg mycket djupt och
hoppade lätt och graciös som en gazell genom dörren, i det hon sände den
förbluffade länsmannen en blick, så full af återhållen skrattlust att
denne, alldeles förvirrad af dess uttryck, knappast märkte att äfven den
äldre damen med en stum, afmätt helsning lemnade rummet. --

Sedan Jenny och tant Agatha intagit sitt kaffe och under till större
delen stillatigande beundran egnat den herrliga forsen ännu någon timme,
tänkte de omsider uppå att anträda hemfärden. De begåfvo sig alltså upp
till hotellet, der de möttes af en öfverraskning, hvars art och natur vi
dock först meddela i ett följande kapitel.

\begin{enumerate}
\def\labelenumi{\arabic{enumi}.}
\setcounter{enumi}{9}
\tightlist
\item
\end{enumerate}

En ``administrativ tjensteman''.

``Fördömd liten satunge!'' hade den värde länsmannen förargad utbrustit
då han såg att fruntimren så der utan vidare krus lemnat salongen, och
han gjorde min af att följa efter.

``En himmelsk sötunge, ville väl bror säga'', inföll Pettersson i det
han framräckte den kokardprydda uniformsmössan.

``Ja, f-dt vacker är hon, den yngre menar jag'', brummade länsmannen,
tog mössan och lade den åter på bordet samt fortfor sedan, lugnande sig:

``Jasa, det der är nu de omtalade utländskorna, Hm, hm, vi
administrativa tjenstemän böra ha reda på allt som försiggår i vårt
distrikt, men, Pettersson, gif mig imellertid en butelj bäijerskt, det
är f-dt varmt i dag.''

Värden skyndade efter ölet. Under tiden kastade länsmannen en forskande
blick på middagsbordet.

``Tre vinglas'', mumlade han, ``hå, hå, det börjar se besynnerligt ut.
Kanske att någon smugit sig ut genom fönstret? Den der saken tål en
liten undersökning. Jag skulle just nu behöfva någon sådan der affär för
att visa hvad jag är för en karl.''

Pettersson inkom och serverade ölet. Efter en grundlig styrkedryck vände
sig länsmannen till honom och sporde:

``Nå, hvar är den tredje i sällskapet?''

``Den tredje i sällskapet?'' sade hotelvärden förvånad, ``dom äro bara
två fruntimmer, det jag vet.''

``Men det tredje vinglaset, hvarom vittnar det?'' återtog kronans man i
sträng ton. ``Sätt sig inte i omständigheter, Pettersson.''

Hotelvärden bleknade något och ville svara, men länsmannen fortfor:

``Jag är din vän, Pettersson, men det säger jag dig på förhand att vi
administrativa tjenstemän ha inga anseende till personen när det gäller
för staten, vigtiga frågor, och all vänskap tiger då. Hvar är den
person, som druckit ur det tredje glaset?'' Och länsmannen pekade med en
storinqvisitors uppsyn på det vid ena bordsändan stående till hälften
urdruckna vinglaset.

``Det tredje vinglaset'', sade nu den åter lugnade värden med komisk
förlägenhet, ``det tredje glaset,bror Smilander, var för mig.''

``Hvasa'', utbrast länsmannen, ``för dig? Det tror jäg inte. Jag
befaller dig i lagens namn att du lyder lagens befallning denna gång --
eljest är jag tvungen att gå helt annorlunda tillväga.''

Med möda lyckades det Pettersson, som tillika på intet vis ville förråda
sina egna tankar om fruntimren, att öfvertyga den i sina förhoppningar
något svikne tjenstemannen om sannfärdigheten af sin omständliga
berättelse rörande den af fruntimren honom visade artigheten. Smilander
drack ur sitt öl och mumlade halfhögt: ``Att bjuda en värdshusvärd på
vin -- är simpelt, eller också är det ren beräkning för att ha en vän i
viken. Men jag uppoffrar Pettersson om jag blott kunde göra något af den
här historien, någonting som å högre ort ådagalade min tjensteifver och
mitt nit. Hm, om det varit mig de inviterat, då kunde jag inte säga
någonting om saken, ty jag har alltid haft god tur hos fruntimmer, de må
nu vara födda här eller der -- men att såder `göra sig grön', han tänkte
härvid på de gröna vinglasen, för en värdshusvärd bebådar någonting helt
annat.''

Som han imellertid icke ville låta sin i eget tycke qvicka ordlek gå
förlorad, vände han sig, redan på förhand skrattande, i det han pekade
på de gröna glasen, till värden och sade: ``Jaså de der fruntimren `göra
sig gröna' för dig?''

Pettersson blef åter smått ängslig, men länsmannen märkte det icke, ty
han fortsatte sin tankegång högt:

``Ja, ser du, det hade varit en annan sak om de bett mig göra sig
sällskap, jag är fruntimmerskarl, jag, och alla flickor ä' galna i mig,
Pettersson, och det är inte att undra öfver heller, -- tretio år, god
inkomst, hygglig karl, hvasa?''

``Nej, gu'bevars'', svarade den andre förbindligt, ``bror är allt en
helt `douschuant' karl, ska' jag säga.''

``Åh ja, när man så vill'', återtog länsmannen och kråmade sig
sjelfbelåtet. ``Men när ämna de sig hem igen? De bo ju på Ojala?'':

``De ha till i qväll beställt min häst och `trilla','' skyndade sig
Pettersson att svara.

``Jag har lust'', sade herr Smilander, ``att göra en visit hos dem.
Ganska täck varelse, den yngre. Nå, Pettersson, hvad heter din
skönhet?''

``Den äldre damen heter Stråle'', svarade den sålunda tillfrågade, ``och
den yngre är väl hennes fröken dotter, kan jag förstå.''

``Ingen titel?''

``För mig uppgåfvo de ingen titel -- rätt och slätt Stråle.''

``Adligt namn kantänka'', återtog länsmannen, ``men det kan finnas
ganska vanliga `strålar' också, till exempel dina strål-likörer, bror
Pettersson.'' Och Smilander skrattade godt åt sin ordlek numro två och
herr Petterson ansåg för sin pligt att skratta med ``höga öfverheten''.

``Får det kanske lof att vara en sådander `vanlig'?'' frågade han
inställsamt.

``Qvickt sällskap skärper vettet'', utbrast Smilander, ``tag hit då en
`stråle', så skall jag dricka fröken Stråles skål.''

Sedan det ljudliga skrattet, som följde på denna tredje upplaga af de
smilanderska qvickheterna, lagt sig samt likören förtärts, tog
länsmannen afsked i det han tillade: ``Helsa de utländska damerna och
säg att jag i morgon gör min visit på Ojala. Adjö.''

``Dumt'', mumlade Pettersson efter den bortåkande, ``att jag skulle
nämna Ojala-herrskapets namn, då dom ändtligen vill vara
`inkognischonerade'! Men hvad skulle jag göra? Emellertid vill jag inte
omtala det för andra frågare härnäst, ty inte behöfver jag ju känna till
det. Men se, minsann, der kommer en vagn med resande'', -- och han
skyndade att emottaga de nya gästerna.

\begin{enumerate}
\def\labelenumi{\arabic{enumi}.}
\setcounter{enumi}{10}
\tightlist
\item
\end{enumerate}

Öfverraskningar.

Då tant Agatha och den älskvärda fröken Jenny i akt och mening att
anträda hemfärden nalkades hotel-paviljongen, hade det nyss anlända
ressällskapet, som bestod af två herrar och ett fruntimmer, redan hunnit
göra sig hemmastadt i salongen. Fruntimret och den ene af herrarne
tycktes vara gamla goda bekanta med värden på stället och de helsningar
som blifvit utbytte å ömse sidor hade varit ganska vänskapliga. Derefter
hade den ene af herrarne och Pettersson dragit sig tillbaka för att,
såsom den resande sagt, likså godt först som sist tala om affärer.

Då tanten och Jenny inträdde, voro derföre endast två af de nya gästerna
i salongen och dessa tycktes vara i begrepp att företaga en promenad för
att bese fallet. Kavaljeren, som stod med ryggen åt dörren, yttrade just
i detta ögonblick till den unga damen, ty ung och vacker var hon, det
kunde Jenny genast se: ``Jag har oaktadt det mäktigt manande dånet inte
ens kastat en blick genom fönstren och jag skall helt och hållet
underkasta mig fröken Ainas anordningar. Skall jag nu kanske binda en
duk för ögonen?''

``Det skall jag göra'', klingade en munter stämma och Jenny sprang
behändigt fram samt betäckte bakifrån med sina små händer den talandes
ögon. Tant Agatha stod mållös af förvåning öfver den oerhörda frihet
hennes ystra systerdotter tagit sig emot en främmande mansperson och den
unga finskan, ty läsaren har väl gissat att det är Aina Ros, tog först
ovilkorligen ett steg tillbaka, men hennes förvirring gaf snart rum för
en skälmsk känsla af sympati och i det hon vexlade en tyst helsning med
den leende Jenny sade hon:

``Gissa, herr Stenrot, gissa hvem det är; jag nästan anar det.''

``Vid alla himmelens makter, jag befinner mig i trolleriets hemland! --
Jenny, vackra, tokiga, snälla Jenny, -- det kan inte vara någon annan än
du!'' utropade vår vän Erik alldeles ``forbaused'' af öfverraskning.
``Men huru i himmelens namn befinner du dig här? Och'', -- han hade nu
fått ögonen fria, -- ``hvad ser jag? Tant Agatha också! Jo, det var en
nätt sommartur, det här!''

Men ingen hörde på honom, icke ens tant Agatha, som nedsjunkit på en
stol och med händerna i kors på sitt bröst blott yttrade de orden:
``Erik -- alltså det var Erik; nå då var det inte så farligt.'' Ur Aina
Ros' vackra öga föll i all hemlighet en hastig blick på Erik. Kanske hon
ville utleta hvilket intryck den mångomtalade Stockholms-kusinens
plötsliga uppträdande gjort på hennes artige reskamrat och hittills
oinskränkt hängifne riddare. Emellertid vände sig Jenny till den unga
finskan och sade:

``Förlåt, fröken Aina -- men min kusin har ju redan presenterat mig
såsom den tokiga Jenny? Fröken Aina Ros -- inte så? Min tant, fröken
Stråle.''

``Jaså'', tog nu Erik, som i någon mon hemtat sig från sin förvåning,
till ordet, ``jaså, all vidare presentation är onödig -- ja, det liknar
verkligen Jenny. Se, sådan är hon; har jag kanske sagt för mycket,
fröken Aina? Men låt mig nu riktigt helsa på er. Goddag, snälla tant och
välkommen till Finland'', han kysste den vänliga damen på hand och
fortfor sedan: ``goddag äfven du, putslustiga yrhätta. Alltså, detta var
målet för den hemlighetsfulla inkognitoresan?'''

``Ja, käre Erik'', sade Jenny och räckte honom sin hand, ``sådana turer
gör man utan resehandbok.''

Erik rodnade vid denna anspelning på hans egenskap af Jennys vanliga
resekavaljer, men han kunde dock icke bli förargad på den vackra kusinen
såder vid det första återseendet. Medan de två damerna gjorde hvarandras
bekantskap medelst det slags frimureri, som i dylika fall är egendomligt
för unga flickor, vände han sig derföre till tanten med en fråga huru
allt detta gått till?

``Jenny gaf sig ingen ro förr än äfven hon skulle komma till Finland'',
blef svaret, ``och, gode Erik, du får allt höra ännu besynnerligare
saker och ting om och när allt sker såsom Jenny planerat, -- men det är
ju sannt, jag får inte sqvallra ur skolan.''

Jenny, som kanske befarade att tanten under inflytande af den första
öfverraskningen skulle göra Erik några alltför förtroliga meddelanden,
gaf dock snart samtalet en mera allmän riktning.

Nu återkom Ros, som af hotelvärden fått veta att till trakten anländt
två förnäma damer från Sverge, äfvensom att den äldre af dem bar namnet
Stråle. Blott med möda hade han kunnat dölja sin öfverraskning och
sinnesrörelse då han derjemte af den meddelsamme värden erfor att dessa
damer nu befunno sig inne i salongen och att de sannolikt sammanträffat
med hans ressällskap. Då han inträdde var han visserligen något blek,
men till det yttre lugn, och han helsade nästan hjertligt på tant Agatha
i det han påminte henne derom, att då de sist togo afsked af hvarandra,
hade detta skett vid brädden af ett litet vattenfall i Skotland och nu
egde deras alldeles oförmodade återseende rum vid dånet af en fors i
hans undangömda hemland.

Mötet mellan Jenny och Ros var af helt annan art; det vittnade väl icke
om ömsesidig förlägenhet, de voro ju båda för stortänkta att behöfva bli
``förlägna'' för sina känslor, men der låg ändock ett visst tvång öfver
dem. Det var liksom hade de frågat hvarandra: och hvad skall nu hända? I
Jennys blick låg dock mera förtröstan än oro, medan Birger, så godt han
kunde, sökte dölja sin inre rörelse. Det skulle ha gifvit de tre öfriga
personerna i det lilla sällskapet mycket att tänka på, detta sätt att
helsa på hvarandra, om de fått rådrum dertill, men tant Agatha som
småningom började ``finna sig i situationen'' afbröt alla hitåt lutande
betraktelser genom att vända sitt tal direkte till den ena af de icke
handlande utan fastmera lidande personerna.

``Jenny, min skatt, betänk huru egendomligt! Då vi sist råkade herr Ros
var det vid ett vattenfall och nu träffas vi åter vid ett sådant!''

``Ganska egendomligt'', ljöd Jennys svar i nästan hviskande ton.

``Mår du illa, Jenny lilla?'' frågade nu tanten oroligt. ``Kanske det
evinnerliga bullret angripit dina nerver?''

``Åh nej'', skyndade Jenny att svara och försökte att le, ``men härinne
är så varmt.''

Nu kom Ros den stackars Jenny till hjelp.

``Skola vi inte bese fallet?'' inföll han, ``Erik börjar väl bli
otålig?''

``Hm'', svarade denne nästan torrt, ``min förvåning öfver tants och
Jennys härvaro har gjort att jag nästan glömt bort hela Imatra.''

``Derom borde du dock alltjemt påminnas af dånet'', sade Jenny, och
tillade i det hon gick förbi honom: ``Snälle Erik, spela inte längre
förvånad.''

``Spela?'' mumlade Erik. ``Det är fullaste allvaret.''

``Men din förvåning är plågsam för mig'', återtog Jenny i samma ton som
förut.

``Den är'', sade Erik bitande, ``ett kapitel i resehandboken.''

``Tack, kusin Erik, det der skall jag inte glömma'', hviskade Jenny,
``men det är oartigt mot fröken Ros att stå här och hviska.'' Erik
rodnade tvärt emot sin vilja, men Jenny skyndade till sin nya
bekantskaps sida.

Emellertid hade sällskapet brutit upp. Ros bjöd sin arm åt tant Agatha.
Aina och Jenny ilade förut och Erik såg sig tvungen att följa efter, men
inom sig erkände han att hans sinnesstämning ingalunda var den
lämpligaste för att riktigt uppfatta och kunna njuta ett storartadt
naturskådespel. -- Hvarföre hade hon rest efter honom? Och huru listigt
hade hon icke vetat att passa på hans ankomst till Imatra! Erik kom nu
ihog att han i sina dagboksanteckningar talat om denna resa. Var hon
svartsjuk? tänkte den egenkäre unge mannen. Detta var visserligen
smickrande för hans fåfänga, men ändock fatalt, mycket fatalt, ty Aina
Ros var i hans tycke såväl vackrare, som ock behagligare och mildare, än
den sjelfrådiga Jenny. Och det åtlöje sedan, för hvilket han utsattes
genom detta sin kusins uppförande! Han såg redan i andanom den muntre
skådespelaren X. på Hasselbacken för vännerna derbortä skildra ``den
ertappade fästmannens öden''. Olidligt att tänka på! Det gör oss
verkligen ondt om den stackars Erik att han skulle befinna sig i en så
olycklig sinnesstämning första gången han besökte det berömda Imatra och
det i Ainas sällskap. Ja, det är så mycket förargligare som han, enligt
sin egen tanke, icke allenast blifvit ``näsledd af sin kusin'' -- utan
till på köpet: hvad skulle Aina Ros och hennes bror tänka om denna
historia? Det såg ju minsann ut som om han verkligen varit en förlupen
fästman. Det var fatalt, mycket fatalt i den -- af sin egenkärlek
missledde Eriks tycke.

\begin{enumerate}
\def\labelenumi{\arabic{enumi}.}
\setcounter{enumi}{11}
\tightlist
\item
\end{enumerate}

Vid Imatra.

Vid åsynen af den stora naturföreteelsen slets det oaktadt en flik ur
det täckelse af moln, som höljde Eriks sinne. Den i början likgiltiga
blicken öfvergick, honom sjelf ovetande, i en beundrande åskådning af
det imposanta fallet, och i samma mon som de yrande skumhvirflarne i
bestämdare konturer framträdde för hans på deras vilda lek riktade öga,
veko det inbillade bryderiets töcken från hans för allt skönt och stort
mottagliga själ, och den herrliga taflan framför honom uppfyllde med
sitt majestät hela hans väsende. Hans bekymmer flögo bort på forsens
dunstmoln, som försvunno i ethern, och det mäktigt väckta medvetandet af
hans egen krafts otillräcklighet, som här så tydligt framställde sig för
honom, återgaf hans sinne dess ädla hållning och jemvigt. Hans hjerta
svällde af tillfredsställelse deröfver att han, utan all reflexion, på
den omedelbara förnimmelsens väg erfor tillvaron af en högre makt än
hans egen ringhet. Den symboliska handlingen att med sin hand liksom
stryka bort molnen från sin panna, åtföljdes af en större klarhets
utbredning i hans inre. Anden höjde sig ur hvardagslifvets och prosans
verld och vår vän var åter samme poetiske Erik Stenrot som förut. En
sympatetisk känsla, för hvilken han hvarken kunde eller ens ville
redogöra för sig sjelf, dref honom till Ainas sida. Jenny, som gått ett
stycke längre fram, hade han glömt; för honom funnos blott Aina och han
vid den eviga forsens brädd.

Tant Agatha hade satt sig på en af den lilla paviljongens bänkar, och,
utan att veta sjelf huru, stod Ros några ögonblick derefter i Jennys
närhet. Hon märkte icke hans annalkande och hvarje muntligt meddelande
från hans sida var, der de befunno sig i det dånande fallets omedelbara
grannskap, en omöjlighet.

Imatras hvirflar ha den egendomligheten att ehuru naturligtvis det hela
öfver hufvud taget städse erbjuder samma tafla af den, så att säga, mest
rörliga oföränderlighet, så byta dock -- kanske en optisk villa -- de
skilda partierna alltjemt om utseende. På samma ställe der nyss en
jättebölja trotsigt reste sin skumhöljda hjessa högt upp\ldots{} och de
lösta, fina vattenpartiklarne, såsom ett vildt fladdrande hår, skakades
i luften, på samma ställe gapar i nästa ögonbliek ett inåt sitt djup
mörknande svalg, bekransadt med hvitfradgande skumbräddar, och man
tycker sig kunna skåda, ned till sjelfva bottnen af forsen.

Man skulle tro att detta skådespel åter och åter upprepar sig, och så är
väl också i sjelfva verket förhållandet, men åskådarens öga, huru länge
han än betraktar forsvågornas jättetåg framför sig, skall nästan aldrig
finna att samma slags grupp af sammanstötande böljor bildar sig på samma
ställe som förut. Uppjagad af en plötslig nyck rusar en från det hela
liksom lösryckt våg med sin kam högt upp emot klippstranden och
öfversköljer oförmodadt den plats som åskådaren nyss trott vara
fullkomligt trygg för sådana påhelsningar och hvilken var alldeles torr
då han beträdde densamma samt följaktligen på en längre tid icke varit
utsatt för någon framforsande böljas vilda famntag. Ögonblickligt rusar
dock vågen, liksom manad med trollmakt, tillbaka i det kokande kaos,
blottande på flera ställen forsens branta sidosluttning, en hemsk
nedgång i den brusande afgrunden, till hvilken i otaliga små rännilar de
högt uppkastade vattendelarne med brådskande ifver åter söka bana sig
väg utmed klippans slätslipade afsatser, glittrande i den oväntadt
skådade dagens ljus. En dylik våg, en slintande fot, och den sålunda
öfverraskade åskådarens öde är afgjordt, ty bortryckt i Imatras famn af
det tillbakarullande svallet, skulle hans jordiska tillvaros ögonblick
snart vara räknade.

På en sådan plats, men okunnig om sin fara, stod den föga nervsvaga
Jenny Bertram alldeles försjunken i åskådandet af de gigantiska syner
som i hastig oinvexling bildade sig framför det magiskt fängslade ögat.
Der delade sig just nu en väldig vattenmassa isär och medan den ena
hälften med höjdt skumbanér i ilande fart fortsatte sin gång, drog sig
den andra hälften liksom tillbaka. Emellan de sålunda uppkomna
vattenväggarne gapade ett djup, snarlikt en öppnad famn, som emot
åskådarne utsträckte sina hvita, skumklädda jättearmar. -- Jenny lutade
sig ovilkorligt framåt i namnlös bäfvan, liksom hennes själ icke velat
förlora ett enda moment af den hänförande synen -- då halkade hennes fot
på den sluttande hällen och i nästa stund spolade en väldig våg öfver
den plats der den älskliga flickan nyss stått, en tjusande bild af den
hyllning behaget egnar kraftens genius. Det vilda elementet röt af
raseri att dess sköna rof gått förloradt, -- ty Jenny hvilade blek och
mållös mot Birger Ros' axel. Denne hade, förtrogen med fallets
egenheter, i det hotande ögonblicket med kraftig arm ryckt henne undan
den öfverhängande faran och hoppat ett steg tillbaka med sin ljufva
börda. Räddarens fötter sköljdes af det fräsande elementet. Den stora
vågen vek tillbaka och fortsatte sin brusande gång, men de små
rännilarne flydde i glittrande kaskader igen till den dånande
Imatra-drottens fot, förmälande sitt nederlag och hans svikna
förhoppning på det sköna rofvet. De slungades af den mäktiges vrede
ögonblickligt i form af dunster högt upp i skyn, der de, upplösande sig
i vänliga solstrålars famntag, för dessa täljde sagan om sin tillvaros
skönaste och sista syn, den ljufliga menniskodottern som de, förenade
med en väldig våg, velat röfva bort från jordisk fröjd och njutning. Nu
dogo de gladt för det de icke lyckats. Men de goda luftandarne fläktade
de små såsom ett uppfriskande duggregn öfver den räddade jungfruns
marmorhvita anlete och aftonsolens milda strålar väckte snart ungdomens
röda rosor igen till lif på hennes fina kind. Jenny slog upp ögonen och
tackade sin räddare med en blick, som vatten- och luft- och
solstrålsandar afundades honom, ty den var innerligt skön emedan den
hade sitt upphof i ett rent och friskt, af kärlek och tacksamhet
uppfyldt hjerta. De två menniskorna som nu hand i hand stodo der bredvid
hvarandra på Imatras klippbrädd, hade båda nyss gifvit och mottagit
stora gåfvor. Han hade ju återgifvit åt lifvet en af dess vackraste
rosor och denna ädla ros åter hade, såsom föremål för hans egen lilla,
men hurtiga handling, i tviflarens bröst väckt hågen för det sanna
lifvet, som icke vill grubblas bort i töcknigt drömmeri utan
förhoppningsrikt verka för stora, höga ändamål. Den besvarade kärleken,
vaknad till klart sjelfmedvetande, hade åstadkommit denna förvandling.
Sålunda mognade i ett enda ögonblick till verkligt lif två goda
menniskor: den yra Jenny Bertram till en lycklig brud och den tviflande
Birger Ros till en nyttig samhällsmedlem.

De stodo, öga i öga och själ i själ, invid hvarandra, sälla i hvarandras
sällhet, åtminstone ett ögonblick, och intet ord vexlades eller behöfde
vexlas, ty de talade ``blickarnes alltförkunnande andespråk''.

Men kall som en obeveklig frostnatt sveper sitt dödande dok öfver
vårlifvets lofvande brodd, -- så stod plötsligen framför Birger Ros'
strålande lefnadsutsigter den i hans döende faders kallnande hand
aflagda eden: ``att aldrig taga till äkta dottren af ett främmande
folk'' -- och med bleknande kinder tryckte han den älskade flickans hand
till sitt af onämbar sorg sammanpressade hjerta. Emellan dem stod ju den
sällhetsdödande skepnaden af en oblidkelig fordringsegare; bilden af den
döende fadren uppreste sig förebrående framför den affallige sonens inre
öga. Men en underbar klarhet tycktes dock ha utbredt sig öfver Birgers
hela själ. Utan inverkan hade den korta sällhetens ögonblick icke varit
och den nya ed att tro på lifvet, hvilken han svurit i det stolta
medvetandet af besvarad kärlek, gaf hans väsende lugn och fasthet -- men
det var vinterdagens kalla lugn och fastheten i det våldsamt kufvade
lycksalighetsbegärets till is frusna framtidsspegel.

Annorlunda gestaltade sig dock framtiden för den sköna Jenny Bertrams
själ. Blekheten på den älskades kind, vittnande om styrkan af den strid
som kämpades inom honom, undgick icke hennes klara blick och äfven för
hennes inre syn stod den manande fadrens ljusskvmmande bild. Men genom
den i oskuldsfull kärleksglädje rodnande ungmöns vaknande väsende gick
en så onämbart behaglig ström af lefnadslust och tro och ur hoppets
framtidsblomma uppstego så tjusande aningar och bilder att hon icke
trodde den kalla skuggan ega kraft att tillintetgöra all denna skönhet.
Och så var det: hon slog sina vackra, liljehvita armar omkring den
älskade mannens hals och hviskade i hans öra, förnimbart för honom till
och med öfver forsens dån, de vingade orden: ``Jag tror inte på ett
oblidkeligt olycksöde!''

\begin{enumerate}
\def\labelenumi{\arabic{enumi}.}
\setcounter{enumi}{12}
\tightlist
\item
\end{enumerate}

``Ett smultron, vuxet i skuggan.''

``Hvar har du dröjt så länge, söta Jenny?'' frågade tant Agatha oroligt
den till paviljongen framskyndande unga flickan. ``Jag ser inte heller
till Erik och de andra?''

``Se der komma de alla tre'', svarade Jenny i det hon låtsade syssla med
själen för att dölja sin rörelse, ``och nu kunna vi på allvar anträda
vår hemfärd.''

Erik sällade sig till tanten, förklarande att han nu gjorde anspråk på
att få leda henne. Jenny och Aina ilade åter förut och Ros, försjunken i
djupa tankar, afslutade tåget. Erik uttalade naturligtvis för tanten sin
beundran för det storartade vattenfallet och återkom städse mycket
fintligt till detta ämne hvar gång den goda damen sökte föra samtalet på
Aina Ros. Sjelf yttrade Exik icke vidare någon nyfikenhet öfver
anledningen till tantens och kusinens oförväntade resa till Finland.
Hvad åter de unga flickorna meddelade hvarandra få och vilja vi icke
förråda, men så mycket kunna vi dock säga, att då sällskapet hunnit upp
till hotellet, kallade de två vackra tärnorna hvarandra du, och icke det
ringaste tecken till svartsjuka tycktes grumla deras unga vänskaps
morgongryning.

Här vore det nu väl vår länge uraktlåtna skyldighet att söka gifva en
liten skildring af huru de två hjeltinnorna i denna berättelse
egentligen ``sågo ut''. Men detta kan vida lättare åläggas än
verkställas. För det första är det i allmänhet svårt att beskrifva unga
damers utseende, ty detta vexlar alltjemt karakter efter de personer, i
hvilkas föreställning en bild af dem skall tecknas. För det andra måste
beskrifvaren vara mycket på sin vakt för att icke, såsom det heter,
``mista koncepterna'' vid uppräknandet af alla de otaliga behag som
hvarje ung flicka kan utveckla, om hon bara vill. Må derföre den hulda
läsarinnan med undseende döma öfver vår ofullständiga skildring. Vi
kunna till vår ursäkt blott säga att om vi icke ingått i
detaljbeskrifningar, detta skett i följd af vår öfvertygelse att den
qvinliga skönhetens väsende icke genom den noggrannaste fotografi kan
återgifvas fullkomligt sanningsenligt. Vi anropa dock vår sånggudinna
att vänligen föra vår pensel, då vår föresats är så föga egoistisk att
vi icke helt och hållet vilja behålla för oss sjelfva de vackra syner vi
förvisso tro oss ha sett.

Aina Ros, öfver hvars af naturen jemna sinne aderton vårar strött sina
blida blomster ur ett fridfullt hemlifs ymnighetshorn, hade tillbragt
större delen af sin tid på landet under sina föräldrars hägn. Fadern, en
lärd prestman och ifrig fennoman, hade sjelf undervisat sin dotter,
hvilken efter föräldrarnes nästan samtidiga frånfälle, fortsatt ökandet
af sitt kunskapsförråd under brodern Birgers ledning. Hon hade en god
del af den der likheten med ``ett smultron, vuxet i skuggan'', som
hennes bror engång skämtvis uppgifvit såsom en egendomlighet hos
``fennomanskorna''. Aina var dock icke alltför landtligt blyg, hon var
tvärtom ganska frimodig och ett års vistelse i Helsingfors hade icke i
ringaste mon förändrat den henne medfödda lugna hållningen, ett
ypperligare vademecum i lifvet, än den mest raffinerade salongsvana i
verlden. Hon förde sig ledigt och behagfullt i hvilken krets som helst,
men hon hade en stor ehuru omedveten fördel framför många af sina så
kallade väninnor i staden: hela hennes väsende var liksom kringflutet af
en egendomlig, vårlig morgonfriskhet. Det oskyldiga uttrycket i hennes
glada, blåa blick förädlades af ett känslans aningsrika skimmer, och
hennes tal klingade som lärkans drill öfver ett vaknande vårlandskap.
Och vår var det också, idel vår inom den jungfruliga barmen och djupa
men ljusa tankar stodo att läsa på den klara, molnfria pannan. Den raka
näsan, icke alltför liten, gaf det milda anletet karakteren af fasthet,
men på den friska munnen log ett muntert löje och en liten skalk tittade
fram ur gropen på kind. Det enda yppiga i den harmoniska företeelsen var
det ljusa hårets rika svall, som böljade långt ned öfver plastiskt
bildade skuldror och hvilket, upplöst ur sina flätor och fangsel, kunnat
omhölja hela hennes intagande gestalt. Sådant var Aina Ros' utseende och
vi tillägga ännu, för sanningens skull, att hon var en mycket
regelbundnare skönhet än vår lilla väninna Jenny Bertram, hvars
beskrifning vi ha all anledning att uppskjuta till en annan gång, emedan
vagnarne nu köras fram, det vill säga, Ros' vagn, i hvilken Aina och han
togo plats, samt hotelvärdens ``trilla'', som i sitt sköte upptog tant
Agatha och Jenny. Erik Stenrot deremot skulle stanna qvar på hotellet
till dess hans vän fått hushållet på Muistola i ordning igen efter den
fleråriga frånvaron.

Afsked togs både hjertligt och ceremoniöst, alldeles såsom det vanligen
plägar ske, men man beslöt ingenting om nästa sammanträffande. Detta
öfverlemnades åt slumpen.

Klatsch! och hästarne drogo till, några vänliga nickningar ännu och den
unge svensken stod ensam qvar på trappan till hotellet vid Imatra fall.

\begin{enumerate}
\def\labelenumi{\arabic{enumi}.}
\setcounter{enumi}{13}
\tightlist
\item
\end{enumerate}

Huru Erik tillbragte qvällen.

``Hur det än må vara och bli'', sade Erik Stenrot leende för sig sjelf i
det han blickade efter de bortåkande, ``i den der gamla vagnen åker mitt
hjerta sin kos.'' Som han emellertid var nog mycket ``af denna verlden''
så gick han snart åter in i salongen, tände en cigarr och satte sig vid
ett af fönstren för att anställa betraktelser. Dagens händelser och
sinnesrörelser, som i afskedets stund alla sammansmält till en enda
uteslutande tanke, framställde sig nu åter för hans själ i alla möjliga
toner och färger. Trött som han var af resan och de andra intrycken,
ville han dock snart åter slippa denna mönstring af ännu så färska
minnen och beställde derföre en butelj rhenskt vin samt bad sin värd
göra sig sällskap. Herr Petterson tycktes ha god tur med fint vin i dag,
och då han fryntligt framsatte vinet och glasen utpekade han leende
etiketten: ``Liebfraunmilch'', sade han, men Erik förstod icke
kyparvitzen, utan drack värden till och inledde ett samtal om väderleken
och skördeutsigterna för året samt grannarne i trakten. Han ville
antagligen begagna den beskedlige herr Pettersson såsom en sömngifvande
aftonlektyr. Denne märkte dock icke sin gästs afsigt, men lyckades
deremot i någon mon motsvara hans förhoppningar genom sina vidlyftiga
historier ur traktens krönika, hvarvid alla herrskapen i grannsocknarne
fingo passera revy, allt till vår stockholmares stora uppbyggelse och
lokalisering i trakten. Erik undertryckte med möda en gäspning och
smuttade på sitt glas. Elfvens vestra sida var affärdad och nu kom turen
till den östra. Länsmannen, häradsskrifvaren med flera företogos och
skildrades, men den otacksamme, eller fastmera tacksamme åhöraren kände
nu otvetydigt sömngudens annalkande. Den slocknade cigarren föll ur hans
fingrar, han blundade, och beställsamma drömelfvor började sitt
fantastiska bestyr omkring honom, ackompagnerade af den brusande forsens
oafbrutna dån, på hvars ljudvågor nu, då det yttre ögat var overksamt,
själen tyckte sig bäras bort till rymder der ett virrvarr af omvexlande
syner mötte den halfinslumrades lösgjorda fantasi. Han tyckte sig, buren
i en stor silfversnäcka af Imatradrottens molnlika vattenandar, sväfva
fram högt öfver lifvets mörka svalg. Och så var det likväl såsom skulle
han i sin ståtliga bärstol glida fram utmed den skummande Imatra. En
outsäglig känsla af stolthet vidgade hans bröst. Men på den höga,
skrofliga klippranden stod hans kusin Jenny Bertram och viftade med ett
långt skärp, som liknade en resekarta öfver Rheinfloden med
sidoteckningar, föreställande bönder som ur doftande drufvor pressade
Liebfraunmilch -- och hennes vackra läppar rörde sig och orden hördes
tydligt öfver forsens larm af honom som åkte i den hvirfveluppburna
snäckan: ``Glöm inte att anteckna denna färd i resehandboken!'' Han
harmades högligen och ville slunga ett strängt ord till den gycklande
kusinen, men ett lent rosenfinger lades på de i vredesmod öppnade
läpparne -- och bredvid honom i silfversnäekan satt leende och blid Aina
Ros\ldots{}

``Ja, det gör mig ondt om Rosen'', ljöd här hotelvärdens något gnällande
stämma, synen försvann och Erik for upp ur sin dröm.

``Ros?\ldots{}'' utropade han förvirrad, hvilket Petterson tog för
tankfull uppmärksamhet, ``Ros? hvad hör det namnet hit?''

``Jo jo men'', återtog värden, ``det hör allt hit också. Han har en
egendom här i nejden, men den är mycket illa skött. Doktorn har ju sjelf
för det mesta varit på resor de senare tiderna och hushållet går som det
går.''

Nu blef Erik vaken och afbrytande värdens vidlyftiga reflexioner bad han
honom helt enkelt berätta huru det stod till med saken. Hvad han fick
höra var af mera lugnande art än han väntat sig. Emellertid fyllde han
glasen, tände en ny cigarr och ställde till värden några frågor angående
de svenska fruntimren. Men här stötte han till sin förvåning endast på
undvikande svar, till dess han slutligen meddelade att unga fröken var
hans kusin, utan att han dock lät Pettersson ana den öfverraskning mötet
med Jenny beredt honom. Han fick nu veta om ``Ojala-herrskapet'' hvad
hotelvärden hade sig ``med visshet'' bekant; ibland annat att från
Wiborg ankommit ett ``dyrbart piano'' och en präktig ridhäst till fröken
äfvensom att hon ``brukade företaga långa ridter utmed elfvens östra
strand, till Wallinkoski och till och med längre''. Detta ansåg dock
Pettersson vara mycket oförsigtigt, ty vägen ditåt, som gick genom mörk
skog, var enslig och fram på hösten ``spökade det röfvare i ödemarken''.

Omsider begaf sig Erik till hvila och tillbragte en ganska lugn natt. De
vid värdens lokalskildringar till hälften bortslumrade timmarne tycktes
för den gången ha tillfredsställt hans fantasis behof af drömmar och han
vaknade följande morgon kry och munter, gjorde sin toilett och steg upp
i värdens från Ojala återkomna ``trilla'' för att aflägga en visit hos
tant Agatha och sin kusin, der vi väl snart åter skola sammanträffa med
honom.

\begin{enumerate}
\def\labelenumi{\arabic{enumi}.}
\setcounter{enumi}{14}
\tightlist
\item
\end{enumerate}

Hvad ett månsken får se.

Då tanten och Jenny kommo hem den qväll vi sist omtalat, skildes de
snart åt. Tant Agatha var trött och Jenny längtade att vara allena.
Deras afsked från Ros och hans syster, som en liten sträcka haft samma
väg som de, hade också varit kort.

``Får jag spela litet, snälla tant?'' frågade den unga flickan i det hon
kysste den vänliga damen på hand.

``Gerna, mitt barn'', blef svaret, ``jag sofver till och med mycket
bättre när du musicerar. Det evinnerliga bullret från vattenfallet går
då inte så omkring i mitt hufvud, och sjung, söta Jenny, om det roar
dig, men tänk också på hvilan, vi ha i dag ju riktigt ansträngt oss.''

Derefter togo de godnatt af hvarandra; tanten gick till sig och Jenny
inträdde i sitt sofgemak. En liten sal, der pianot stod, åtskiljde de
två rummen. Tanten bodde på kökssidan och deras svensk-finska jungfru,
Thilda, som var köksa och tolk tillika, logerade i köket. Bakom Jennys
rum, åt gårdssidan till, var en obebodd kammare och emellan denna och
köket en tambur. I en liten sidobyggnad på gården bodde ett bondfolk som
skötte om det lilla hemmanets fyra kor, en arbetskamp och Jennys
ridhäst. Så mycket om lokaliteten, nu om något annat.

Jenny var icke upprörd oaktadt det som hon i dag upplefvat gjort ett
djupt intryck på hennes själ. Hon hade icke ens för någon omtalat den
fara, i hvilken hon sväfvat, emedan densamma för henne blott hade en
ringa betydelse i bredd med den tysta förklaring, som följt derefter.
Jenny förenade med en rörlig och liflig själ en sällsynt sinnets
spänstighet och ehuru hennes känslor ganska lätt kommo i svallning, voro
dock såväl hennes lynne som ock nerver så starka att någon egentlig
efterdyning icke egde rum i hennes själ. Der låg tvärtom utbredd öfver
densamma en nästan lugn stämning. Då en djup karakter känner sig sannt
lycklig, erhålla dessa hjertats högtidsstunder en egendomlig pregel af
tyst tacksamhet, i motsats till den mera ytliga menniskans känsloutbrott
vid någon lyckosam vändning i dess öde. Men denna stilla
lycksalighetskänsla hade hos Jenny icke den ringaste anstrykning af
sentimentalitet och hennes stämning närmade sig mera ett belönadt barns
naiva, friska glädje öfver hvad det vunnit, än ett reflekterande öfver
sin lyckas höjd och djup. Hon njöt af den sköna blommans vällukt som
öppnat sin kalk i hennes hjerta och tänkte, lifvad af glada
förhoppningar, utan att tillgjordt rodna, på den tid då äfven han, den
hon så högt älskade, skulle i lä njuta samma lycka, som nu redan kommit
henne till del. Med frid i ton och blick sjöng hon sakta några strofer
af en enkel svensk folkmelodi i det hon gjorde sin lätta natt-toilett.
Då hon var färdig dermed trädde hon fram till fönstret, sköt undan
gardinen och blickade ut i den tysta natten. Hennes öga riktades mot den
mörkblå himlen och liksom utbytte en vänskapshelsning med stjernorna
deruppe, medan öfver de fint öppnade läpparne en enkel bön smög fram
till verldarnes herre, som ju äfven unnat hennes lilla hjerta dess
ljufva lycka. Snart stod hon dock vid sängen och i det hon lösgjorde
sitt rika kastanjebruna hår och smått kokett slog detsamma öfver sina
vackert rundade axlar så att det täcka ansigtet tittade fram liksom ur
en mörk ram, neg hon skalkaktigt för sin egen bild, som hon såg i
spegeln midt emot, och sade leende: ``godnatt, du lyckliga flicka
derborta!'' blåste ut ljuset och lade sig med en känsla af inre och
yttre välbefinnande i den svällande, hvita bädden, som i sin kyska famn
mottog den älskliga varelsen. Hon hvilade der under de afundsjuka
omhöljena liksom en skatt af dolda behag. Månen, älskad af och sjelf en
älskare af skönheter, ty hvarföre skulle han annars så troget följa i
skönheten Jordens ledband, utsände nästan torgäfves sitt blida sken att
emellan gardinerna uppsnappa några drag af den slumrande flickan. Skenet
skulle sedan beställsamt skildra hvad det sett för den högtuppsatte
kännaren deruppe, förljufvande sålunda hans ensamhet i rymden, ty de
andra stjernorna togo föga notis om honom, alldenstund han, sålänge de
utfört sin ringdans på fästet, ansetts såsom en Jordens förklarade
kavaljer. Lyckligt inkommet i rummet, tack vare gardinerna, dem Jenny
glömt att åter draga till och som räknas till förhängenas medgörliga och
blundande, icke ``förklädenas'' stundsamma familj, smög sig månskenet
fram till tärnans bädd. Det sade i sitt drabantaktiga hjerta något om
det angenäma uti att stå på vakt vid unga flickors bäddar, men kände sig
liksom något sviket i sin herres förhoppningar, ty ett hederligt månsken
har inga sådana för egen räkning, då det förgäfves sökte leta sig fram
för att belysa den slumrande ungmöns former. Dock voro de få partier af
den hvilande skönheten, som uppenbarades, af sådan art att till och med
det i sitt innersta väsende jungfruligt sinnade månskenet kunde lofva
sin herre och utsändare en herrlig skörd om det blott lyckats att lyfta
en liten flik af det omhölje, som dolde nu blott anade skatter. Den ena
armen var lagd under det vackra hufvudet, den andra hvilade behagfull
och till en del blottad på det litet undanskjutna täcket, hvarvid
tillika en skymt af den hvita, fint rundade axeln blef den ifrige
åskådarens belöning. Det silkeslena håret bildade omkring det i ljuflig
slummer näpet rodnande anletet en krans af mörka ringlar, hvilka, liksom
betagne af förtjusning öfver den lilla rosengård de omslöto, stannat i
sitt fritt böljande, lockiga svall. Ögat var tillyckt, dess tindrande
strålar lyste nu inåt, men en leende engel hade förglömt sig på de
purprade läpparnes friska bädd. Månskenet, som såg detta, suckade öfver
sin natur att blott vara ett sken. Så högt dess ursprung än var och så
oegennyttig dess karakter, nu hade det dock velat vara en aldrig så
liten smula jordisk verklighet, ty det trodde förvisst att detta
behöfdes för att riktigt kunna uppfatta stundens behag. Och det stackars
månskenet blef i hela sitt okroppsliga väsende så närgånget längtande
att -- Jenny slog upp sina ögon. ``Jag kan inte sofva för månskenets
skull'', sade hon och gled lätt ur sin bädd, sköt de små fötterna i de
nätta silkestofflorna, och stod, med det mörka håret böljande öfver den
hvita, tunna nattdrägten, på golfvet just midt i månens sken. Men
jorddrabanten deruppe hade velat jubla högt af tillfredsställelse om han
blott haft röst dertill och det hänryckta skenet utgjöt nu hela sin
glänsande ljusflod öfver denna den tysta nattens tjusande uppenbarelse.

``Jag vill spela litet'', sade Jenny och sväfvade lätt som en nattlig
syn till pianot, ur hvars tangenter de fina fingrarne snart framlockade
en liten tonverld af smältande ackorder. Och tanten sof godt i rummet
nästintill och drömde vid tonernas milda klang om sin egen flydda ungdom
och, hvem vet det, sin första och enda, kanske i en för tidig graf
bäddade kärlek. Men månen deruppe ``på fästet blå'', der han med sitt
blyga sken omfamnade den vackra pianisten, tänkte så för sig: ``Du
förtjenar det, och jag hoppas, du hulda jordedotter, att snart få lysa
öfver din lyckliga bröllopsnatt.''

Om nu Jenny anade till den ensamme himlavandrarens tankar, veta vi icke,
men hon lemnade snart pianot, tilldrog gardinerna i sitt fönster i det
hon vänligt nickade upp till månen och smög åter i sin bädd.

``Nyss drömde jag halfvaken om Aina Ros och kusin Erik att de voro
förlofvade'', sade hon och qväfde med fingret på den skälmska munnen ett
nästan hörbart skratt, ``men nu vill jag riktigt sofva och drömma en
glad dröm om den bistre fennomanen.''

Sådan var, vänliga läsarinna, Jenny Bertram, måhända en afbild af dig
sjelf.

\begin{enumerate}
\def\labelenumi{\arabic{enumi}.}
\setcounter{enumi}{15}
\tightlist
\item
\end{enumerate}

Två drömmar på en natt.

Vi lemnade vår vän Erik Stenrot på väg till Ojala, på hvars lilla, men
prydliga gårdsplan vi nu möta honom. Vi böra här anmärka att Erik aldrig
fallit på den tanken att hans kusin möjligtvis hyste någon böjelse för
Birger Ros. Ehuru bekantskapen i Skotland varit ganska intim och Jenny
synbarligen var intresserad af umgänget med den allvarlige finnen och
ofta nog tillbragt flera timmar med honom utan annat sällskap än den
beskedliga tant Agathas, så trodde dock Erik att det endast var
deltagandet för hans, det måste han medgifva, högsinta planer och varma
fosterlandskärlek, som fängslat den unga flickans eldfängda själ. På
tillvaron af en djupare känsla af alldeles personlig natur kom han,
såsom sagdt, icke ens att tänka och om han tänkt derpå, så ansåg han sin
kusin, något som icke vittnade synnerligt förmånligt om hans
menniskokännedom, för alltför -- \emph{liflig} att kunna fästa sig vid
en sådan man som Ros. Att emellertid Jenny ur hela djupet af sin rika
själ älskade Ros har läsaren sett och han skall äfven under berättelsens
lopp erfara att den ädla flickans kärlek icke inskränkte sig till en
böjelse af det alldagliga slaget.

``Ett egendomligt infall af Jenny, det här'', mumlade Erik, ``att så der
plötsligen utan vidare bosätta sig här i ödemarken, men jag skall
minsann utforska tant och skaffa mig ljus i den här något
besynnerliga\ldots{}''

``Godmorgon Erik!'' ljöd Jennys morgonfriska röst från den
granrisbeströdda förstuguqvisten, ``det var snällt att du kom så tidigt.
Medan tant drar försorg om frukosten kunna vi göra en liten promenad
utåt den här sidan af fallet. Gif mig armen.''

Erik följde uppmaningen och de beträdde en väg som ledde utmed forsens
venstra strand.

``Kan du tänka dig att dånet är vida svagare på den här sidan än på den
andra; här kan man tala utan att behöfva skrika'', sade Jenny.

``Jaså'', sade den tilltalade torrt, men inom sig tänkte han: ``Hon
spelar sitt spel förgäfves; jag måste komma sanningen på spåren.''

``Hvad tänker du på, Erik?'' frågade Jenny vidare. ``Se här ha vi en
liten paviljong, vi också. Tag plats och låt oss prata litet; tänd en
cigarr om du behagar.''

``Minsann, goda Jenny'', utbrast nu vår vän, ``du tyckes taga vårt
plötsliga sammanträffande såsom en helt vanlig händelse; jag åtminstone
var alldeles oförberedd derpå.''

``Men deremot inte jag, kusin Erik'', genmälte vår lilla forsfröken,
``jag visste ganska väl att du skulle komma hit med Ros'' -- hon rodnade
litet -- ``och hans syster.''

``Huru visste du det då? Aha, ja, af mina dagboksanteckningar, kan jag
förmoda. Men hvad i all verlden vill du göra här?''

``Gissa, mon cher, kanske har jag samma planer som du -- jag ämnar
måhända i likhet med dig\ldots{}''

``Jenny!'' afbröt den unge mannen och nu var det hans tur att rodna,
``hvad ämnar du i likhet med mig?''

``Jag skall till svar citera dina egna ord. Jag vill se huru en ung
nation går till väga då den arbetar uppå att grundlägga sin framtid.''

``Du? Och det här vid Imatra?'' utropade Erik.

``Ja, nästan lika bra som i Helsingfors' salonger, men \emph{de} finnas
också på min promemoria.''

``Och derföre har du öfvertalat tant Agatha att hyra detta landtställe?
Jenny, goda, snälla Jenny, var nu ett ögonblick allvarsam och bekänn att
det blott var ett underligt infall af dig att företaga denna resa. Tänk.
hvad skola menniskorna i Stockholm säga derom och vore det roligt om du
i skämttidningarne finge läsa hvarjehanda qvickheter om Jenny
Bertram\ldots{}''

``Vet du hvad, kusin Erik'', afbröt den unga flickan helt allvarsam,
``jag börjar nästan tro att Jenny Bertram inte återvänder till
Stockholm.''

Om Imatra plötsligen stannat i sitt lopp eller dess dån med ens
förstummats, lemnande rum åt en graflik tystnad, hade detta knappast
kunnat till den grad förvåna Erik, som dessa hans kusins ord nu gjorde.

"Ämnar du kanske helt och hållet bosätta dig \emph{här?"} utropade han
och såg helt bestört ut.

``Det tror jag \emph{inte"} svarade Jenny lugnt.''Men hvartill dessa
frågor? Jag är här -- och du är ju också här. Jag forskar likväl inte
efter \emph{hvarföre du} är här."

``Ja, emedan du känner mina planer\ldots{}''

``Dina planer -- vid Imatra, gode Erik? Vet du, till kunskap om dem kan
jag endast drömma mig -- och det har jag gjort i natt. Ja ja, jag drömde
verkligen i natt om dig och att vi voro\ldots{}''

Ett muntert skratt afbröt fortsättningen.

``Vi voro?'' upprepade Erik något orolig, ``vi voro?''

``Hvad mer, kusin Erik!'' inföll Jenny helt naivt, och stora
svettdroppar af ängslan frampressades på Eriks panna då hon med en
utomordentligt skalkaktig blick på honom fortfor: ``Vi voro, hvad jag
länge förutsett skulle inträffa, vi voro, du och jag -- i Helsingfors.''

``Jaså, inte förlofvade!'' sade Erik med lättadt hjerta för sig sjelf
men ansåg dock för tids- och platsenligast att något skarpt tilltala sin
kusin. Dessutom kände han sig i viss mon förnärmad genom det gyckel
Jenny uppenbart tillät sig med honom. Han sade derföre:

``Jag kan inte gilla dessa tokerier. Du vet att jag menar det uppriktigt
väl med dig och jag fruktar\ldots{}''

``Nåväl Erik'', föll den unga tärnan honom, plötsligt alldeles
allvarsam, i talet, ``nåväl, jag vill då också uppriktigt berätta för
dig min underliga dröm i natt. Tänk sedan om mig hvad du vill.''

``Berätta!'' sade Erik icke utan en viss nyfikenhet.

``Vi voro i Helsingfors, du och jag, kusin'', började Jenny, ``och vi
befunno oss i ett stort sällskap. Der talades mycket om allehanda
allvarliga ting, om folkets rätt och nationalitet, om gammal och ny
kultur och andra stora lifsfrågor. Jag tyckte, i drömmen förstås, att
det var mycket uppbyggligt, i synnerhet då en af talarne med varm
vältalighet skildrade sitt ur seklerlång dvala uppvaknande folks stora
framtidsförhoppningar och tillika med varnande stämma hotade med andligt
beroende och undergång om detta folk skulle lita på någonting annat, än
sina egna krafter; men han sade att dessa krafter voro två och att man
icke behöfde misströsta om framgång ifall de inginge en förening med
hvarandra. Jag förstod nog att det var fråga om det finska folket och
dess ädla kamp för sitt andliga sjelfbestånd och jag kände inom mig att
den ena af de två verkande krafterna måste ådagalägga mycken hängifven
kärlek för den stora saken, och, väl icke afsäga sig sina minnen, sin
färg och sitt namn, men dock i viss mon strida under den andres baner.
Det var den svenska kraften, medan det finska elementet deremot i fulla
andedrag insöp det nya lifvet och med väckt hopp gick sin framtid till
mötes. Men der uppträdde en annan talare och sade att bredvid det finska
folket fanns, och det just inom dess eget land, dess allra farligaste
fiende. Denne bar skepnaden af en vän, men var likväl genom hela sin
ställning i samhället så mäktig och framstående att han hotade qväfva
det egentliga folkets utveckling, och denne falske vän var det svenska
elementet i Finland. -- Då, Erik, vredgades jag och såg mig om efter
dig, ty jag hoppades att du skulle uppträda och med ett manligt ord
tillbakavisa beskyllningen om falskhet, men jag såg dig inte, och då jag
gick att söka dig, fann jag dig omsider i en krets af muntra unga
flickor, med hvilka du skämtsamt bytte ord om samma sak, för hvilken
derborta hjertan och hjernor glödde. Jag vände om och, man är så
underligt stark i drömmen, uppträdde sjelf inför den stora församlingen
och sade högt att jag skulle bevisa dem huru hjertligt och uppriktigt
svenskarne menade väl med Finland och att de båda folken ingått ett
evinnerligt fostbrödralag, som beseglats i hundrade blodiga strider. De
församlade männen bugade sig blott artigt leende, men den förste talaren
störtade ut och en underbar, oemotståndlig makt förmådde mig att följa
honom. Då jag kom ut befann jag mig i en vacker trädgård och der såg jag
dig gå arm i arm med ett ungt fruntimmer. Jag förstod då att I voren
förlofvade med hvarandra. Jag vaknade.''

Erik såg brydd ut, men Jenny fortfor, och vid det hon talade antog hela
hennes väsende ett sällsamt uttryck af inspiration:

``Jag inslumrade dock snart åter och en annan dröm framställde sig för
mig, helt olik den förra. Det var ett stort haf, på hvilket en liten
farkost sträfvade att uppnå en aflägsen ö. I den vackra slupen befunno
sig en roddare och jag, men vi förföljdes österifrån af ett stort,
hemskt vidunder. Roddaren var en stark man och skötte sina åror med
kraft och ifver, men som han rodde mycket starkare med den högra än den
venstra armen var det lätt att förutse att han, i stället för att komma
till målet, skulle ro förbi detsamma samt till slut upphinnas af
vidundret. Om han nu, för att få kosan ställd rätt fram, icke ansträngde
fulla kraften af sin högra arm, så saktades åter farten -- och det mörka
spöket österifrån var i raskt antågande. Dystra tvifvel om framgång
bemäktigade sig roddaren och jag led med honom. Men då rann det mig i
hågen att jag ju kunde hjelpa till på den sidan der hjelp behöfdes.
Hurtigt fattade jag en åra och gick till verket. Vår färd gick nu rakt
på den efterlängtade ön, hvilken vi också uppnådde. Här stego vi i land
och roddaren vände sig till mig och sade: `Stor tack för handräckningen;
när två krafter förenas medför det välsignelse, och som vi arbetat
tillsammans i nödens stund så torde vi väl också framdeles trifvas
tillsammans.' Så skedde, och vi bodde på ön i allsköns endrägt. Medan
han plöjde åkern, skötte jag om blomstergården. Hans kraft tröttnade
aldrig, och jag tröttnade aldrig att bereda honom glädje med mina
blommor. Han var lika stolt öfver mina blommor som jag öfver hans
präktiga skördar. Och de utländingar som besökte oss prisade vårt samlif
såsom förnuftigt och godt. Men vidundret lemnade oss i ro, ty, ehuru det
gerna velat uppsluka våra skördar, så kunde det å andra sidan icke
riktigt uthärda doftet af våra blommor. Likväl måste vi vara på vår vakt
för detsamma. Derföre, när den ena blundade väcktes han genast af den
andra. Det var ett lif, fullt af verksamhet och värde, ett sådant lif
som jag ville lefva för alltid. Och nu, Erik, får du tänka om mig hvad
du vill.''

Så talade Jenny och den unge mannen svarade tankfull:

``Du har gifvit mig en sträng lexa, Jenny lilla, och du kan vara
öfvertygad derom att jag förstår de två drömmames dubbelmening. Men säg
mig nu'', tillade han i något muntrare ton ehuru han dervid icke kunde
undertrycka en viss förlägenhet: ``hvem var min vackra drömbrud och hvem
var roddaren, hvars tvifvelsmål om framgång du så lyckligt häfde?''

``Bruden, kusin Erik'', ljöd Jennys svar, ``var inte jag och roddaren
till den aflägsna ön var inte du.''

``Aha'', utropade kusinen, som plötsligt fick ljus i saken, ``hvad jag
länge varit blind!''

Med ett leende af det mildaste behag räckte Jenny honom handen och sade:

``Låt det nu vara såsom det är och fråga inte vidare.''

Hon steg upp, tog åter hans arm och de två kusinerna gingo under tystnad
hem.

\begin{enumerate}
\def\labelenumi{\arabic{enumi}.}
\setcounter{enumi}{16}
\tightlist
\item
\end{enumerate}

Mellanspel.

Då Jenny och Erik kommo fram till bostaden stod en ispänd, grönmålad
schäskärra på gården och tant Agatha skyndade emot dem, utropande:

``Ack, så bra att ni kom, i synnerhet, du Erik, välkommen, välkommen! Vi
ha, såsom du ser, fått alldeles oförväntade främmande, herr länsman
Salamander\ldots{}''

``Smilander, om jag får be'', inföll en herre som nu vände sig om och
helsade.

``Min systerson, ingeniör Stenrot'', fortfor den gamla damen sin
presentation och tillade artigt: ``Var så god och tag en liten smörgås,
sedan kan ju herr Sim\ldots{} befallningsman för min systerson utlägga
sitt ärende, han förstår sig bättre på sådana saker än vi fruntimmer.''

``Vi administrativa tjenstemän'', sade kronans myndige man, ``böra gå
ordentligt tillväga; men'', tillade han i en nedlåtande ton, hvilken på
det högsta förargade Erik, ``men jag vill inte afslå fruns inbjudning.''

Fruntimren drogo sig tillbaka och herrarne satte sig till bordet, som
var dukadt i den rymliga förstuguqvisten.

``Godt bränvin, det här'', inledde länsmannen, sedan han tagit sig en
sup, samtalet, ``af Imatravärdens bästa kummin. Ja, ja, svenskarne
förstå sig på bränvin, det är bekant.''

``Finnarne jemväl'', svarade Erik torrt. ``Hvad behagas? Öl eller
porter?''

``Porter!'' utbrast Smilander plumpt. ``Det var gentilt, det.''

Frukosten intogs under nästan ömsesidig tystnad. Då den var afslutad och
Erik bjudit länsmannen en cigarr, sporde han höfligt:

``Och hvad förskaffar min tant äran af ert besök, herr befallningsman?''

``Ah -- ja -- hm, jag gjorde liksom bekantskap med damerna i går vid
fallet'', genmälte Smilander, ``och så ville jag i dag aflägga min visit
hos herrskapet. Är den äldre verkligen er tant, herr ingeniör, eller är
det bara såder som på resor brukas? Den yngre\ldots{}''

``Herre!'' utropade Erik, ``säg mig kort och tydligt, hvad är ändamålet
med er visit?''

``Inte så het, min herre'', sade nu den objudne gästen i myndig ton,
``vi administrativa tjenstemän måste ha noga reda uppå de främlingar som
komma hit till orten.''

Erik rodnade af förtrytelse och afvägde just de uttryck, i hvilka han
skulle ge den pållugne mannen en tillrättavisning, då Jenny, som af den
ängsliga tanten fått höra anledningen till besöket, inträdde från salen
och, vändande sig till länsmannen, yttrade: ``Är det fråga om våra pass,
så kunna vi dermed inte tjena nu, men vi skola skrifva efter dem till
Wiborg om herr Simlander önskar det?''

``Simlander'', utbrast denne på ett föga höfligt sätt, ``jag heter
Smilander och fordrar att herrskapet skall dokumentera sig, eftersom ni
ä' utländingar.''

``Så mycket höfligare borde ni bemöta dem!'' föll nu Erik något häftigt
i talet. ``Herrn har ju hört att fruntimren äro min tant, fröken Stråle,
och min kusin, fröken Bertram, hvad vill herrn mera?''

``Stråle, Bertram?'' återtog länsmannen förargad. ``Och höflig kan jag
nog vara, om man visar mig tillbörlig respekt. Nå, den här saken börjar
se kinkig ut för er, skall jag säga. Är då den yngre inte den äldres
dotter?''

``Fröken Stråle är fröken Bertrams moster'', svarade Erik häftigt och
steg upp. ``Jag tycker det kan nu vara nog med förklaringar.''

``Niin se puuhta kahtowa luulis, så den tror som från trädet glor
{[}finskt ordstäf{]}, men se jag vill se passen, jag'', menade
länsmannen och afslutade sin sats med en föga höfvisk hvissling.

``Jenny, gå in igen, jag ber'', sade Erik med knappt återhållen vrede,
``så skall jag nog lära den här herrn litet folkvett.''

``Folkvett?'' skrek Smilander. ``Jag skall säga munsjörn att vi
administrativa tjenstemän här i Finland inte låta hutla med oss. Skaffa
hit passen eller\ldots{}''

``Se så, nu har min kusin lemnat oss och vi äro på tu man hand, herre'',
afbröt Erik med bister uppsyn, ``säg nu, hvad vill herrn egentligen ha?
Passen skall ni få se när de anländt från Wiborg.''

``Hå, hå, var lagom stursk, herre. Hvar har han då sitt eget pass, hä?
Kan herrn dokumentera sig sjelf?''

``Jag vet inte med hvilken rättighet ni examinerar mig, herre, -- men se
der kommer, såsom kallad, någon som skall ge er administrativa myndighet
bättre besked än jag i denna sak. Halloh, Birger, stanna för en liten
stund din häst och stig af, jag ber.''

Den sålunda anropade var vår vän Birger Ros. Han steg af hästen och
Erik, som gick honom till mötes, meddelade i korta ordalag det bryderi,
hvaruti fruntimren råkat. Ros log åt länsmannens enfaldiga tjenstenit,
såsom han kallade det, och nalkades med en höflig helsning denne värdige
personage, hvilken under tiden förfogat sig till grinden och med synbart
intresse tycktes betrakta den sist ankomnes ridhäst.

``Ädelt kreatur, herr doktor'', sade han sedan han besvarat Ros'
helsning.

``Åhja'', menade denne, ``men här är nu fråga om något annat.''

``Jaså, hvad då?'' sporde länsmannen i det han fortfor att noga betrakta
hästen.

``Jo'', återtog Ros, ``var god arrangera den här saken med passen. Jag
känner herrskapet, det är fröken Stråle och hennes systerdotter fröken
Bertram från Stockdolm.''

``Ja, ja, men passen'', sade länsmannen, denna gång dock i betydligt
höfligare ton än förut.

``Dem skall herrn få, så snart de anländt från Wiborg. Se så, var inte
vidlyftig nu, utan lemna herrskapet i fred. Jag känner fruntimren från
Stockholm och\ldots{}''

``Hur f-n känner doktorn dem också? Alltså alldeles inte polskor -- det
var dumt -- utan bara svenskor.''

``Vackert så!'' inföll Erik.

``Åhja vackra, men inte farliga, nå strunt då i passbestyret tills
vidare, fastän det var skada att det inte blef litet krångel utaf. Hvad
sägs emellertid om fyrahundra mark för kreaturet der? Skulle tro att
fyrahundra kontant ä' pengar för doktorn i dessa tider?''

``Hästen är inte till salu'', svarade Ros torrt; ``men var nu god och
kom med mig, jag har några andra affärer att göra upp med herr
Smilander.''

``Resonabel och gentil, jag är alltid gentil och resonabel. Ber om min
komplimang till damerna. Åh, se der äro de. Adjö, mina damer, tag inte
illa upp besöket, men vi administrativa\ldots{}'' I detta ögonblick
körde drängen fram schäsen och fortsättningen afbröts. Ros, som lofvat
Erik att snart återkomma och redan en stund sutit i sadeln, helsade
vördnadsfullt fruntimren och lät sin muntra springare sätta af i raskt
traf. Länsmannen följde efter.

``Herr Ros ser bra ut till häst'', anmärkte tant Agatha.

``En fulländad ryttare'', menade Erik.

``Kommer du ihog'', frågade Jenny skalkaktigt, ``vår äfventyrliga ridt
från Inveloch Castle till Mount Nevis?''

``Ganska väl'', genmälte kusinen muntert, ``och jag minnes jemväl
återvägen då vi fingo tillbringa natten i ett bondvärdshus, alla tre i
ett rum, medan tant derborta i staden höll på att dö af ängslan. Det var
egentligen du och Ros, som med edra evinnerliga sidoutflykter fördröjde
hemfärden.''

``Men det var roligt'', sade Jenny tankfullt. ``Tant glömde snart sin
öfverståndna ängslan och det var vid detta tillfälle vi egentligen lärde
känna Ros. Han var utmärkt intressant då han med sina skildringar från
Finland förkortade de ofrivilliga väntningstimmarne under den stormiga
natten. Och vid Loch Lomonds stränder, Erik, kommer du ihåg hans djerfva
fantasier om Finlands framtid?''

``Ja visst, Jenny'', svarade denne smått ironiskt, ``och den höga
klippan, i hvilken Ros, med ögonskenlig fara för sitt lif, högg in
bokstäfverna J.B., tvåhundra fot öfver vattenytan, samt huru förlägen
han blef då jag påstod att hans B på afstånd såg ut som ett R\ldots{}''

``Och ditt melodiska flöjtspel sedan'', afbröt Jenny honom, ``den der
vackra sommarqvällen!''

``Jaså, då ni lemnade mig ensam på stranden, medan ni rodde omkring på
sjön. Ja, ja, Jenny'', tillade Erik, ``jag börjar kunna tyda drömmar,
åtminstone sådana som handla om sjöar och båtar och så vidare.''

En blick af Jenny, en af de der underliga blickarne igen, tystade
talaren och samtalet antog snart en mera likgiltig riktning. Erik
förtäljde om sina reseintryck från sin tur genom Finland. Efter vid pass
två timmar återkom Birger Ros. Men visiten blef endast kort. Affärer af
vigt fordrade hans snara hemkomst, posten måste expedieras, med mera.
Det oaktadt framställdes vördsamt och upptogs vänligt en inbjudning från
Aina och tant Betty (ty äfven Birger och hans syster hade en gammal
tant, det passar så bra i berättelsen) att tillbringa hela den följande
dagen på Muistola, då Erik äfven skulle medtaga sina saker och tills
vidare uppslå sitt högqvarter hos sin finske vän. Sålunda var allt väl
arrangeradt för den närmaste framtiden.

\begin{enumerate}
\def\labelenumi{\arabic{enumi}.}
\setcounter{enumi}{17}
\tightlist
\item
\end{enumerate}

Idylliskt lif.

För Birger Ros begynte en tid af mycken verksamhet. Reparationer å
mangård och ladugård samt åkerbruksarbeten utan all ända togo hans
omsorger i fullt anspråk. Icke alltid lätt häfda affärsbekymmer
undanträngde för en tid nästan alla högre flygande tankar -- en hvila
för hans själ som tycktes bekomma honom synnerligen bra. Åtminstone gaf
blekheten på hans kind småningom rum för en friskare färg och ögat
blixtrade åter af lefnadslust och medveten inre kraft. Huruvida den
nästan dagligen återkommande åsynen af Jenny Bertram jemväl bidrog till
denna fördelaktiga förändring i den unge Muistolaherrns yttre utseende,
lemna vi dock osagdt. I alla fall utöfvade den helsosamma praktiska
verksamheten, såsom vanligtvis hos kraftfulla naturer, ett välgörande
inflytande på honom. För de öfriga medlemmarne i vår lilla krets af
bekanta vid Wuoksens strand antogo förhållandena också rätt angenäma
former. Tant Agatha och den fryntliga tant Betty blefvo mycket snart
goda vänner, och som den tiden var inne då ``bär insyltas och safter
beredas'', en konst, hvari tant Betty icke hade sin like i hela trakten,
så befann sig den vänliga tant Agatha snart åter försatt i de tider då
hon såsom blomstrande prostdotter skötte hushållet för sin salig fader,
kyrkoherden i N. församling i Jemtland. Lilla kaffepannan illustrerade
för- och eftermiddagar de sträfsamma gamla damernas köksbestyr, och till
ett litet parti ``mariage'' om qvällen hade den artige Erik förskaffat
kort åt de två tanterna, såväl den ``verkliga'', hvilken ibland tog sig
friheten att litet ``banna upp'' honom för hans pojkstreck, som ock den
``präktiga'' tant Betty, hvars förklarade gullgosse den muntre svensken
inom ett par dagar blifvit. De gamla spelade alltså mariage, medan de
unga damerna promenerade och musicerade på Ojala samt musicerade och
promenerade på Muistola, troget biträdde af den outtröttligt
beställsamme Erik. Ros var sällan ledig utom om qvällarne, då han ibland
kunde vara ganska underhållande i sitt samtal, och Jenny företog någon
gång om dagarne långa ridfärder helt allena, troligen, vi förmoda det,
för att begrunda allt hvad hon fått höra så intressant framställt
qvällen förut. En följd häraf blef att Aina och Erik ganska ofta
uteslutande voro hänvisade till hvarandras sällskap, hvilket förhållande
alldeles icke tycktes framkalla någon ledsnad hos dem.

Sålunda förflöto ett par veckor af Augusti månad i mycken trefnad och
allsköns ro. September månad inträdde och väderleken började bli
ombytlig. Lifvet inom hus fick nu företräde framför det i skog och mark.
Qvällarne blefvo nu längre och promenaderna kortare. Aina började tala
om den instundande skolterminen och Jenny kunde icke dölja ett visst
intresse för den blott två gånger i veckan ankommande posten. Hon
väntade måhända någon underrättelse af vigt, ty äfven unga flickor kunna
någongång föra en innehållsdiger brefvexling. Erik var i allmänhet
densamme som förut, utom att han stundom såg helt djupsinnig ut, och
hans kusin förklarade att han antingen förde en dagbok eller skref vers.
Han delade Jennys otålighet när postdagen stundade, och man hade anmärkt
att han skrifvit flera bref till Stockholm. Nyss hade han erhållit ett
bref derifrån, hvilket synbarligen bragte honom i det yppersta lynne.
Den dag detta inträffat, var Ojalaherrskapet om qvällen på besök hos den
Ros'ka familjen. Tant Agatha, som fått del af innehållet i Eriks bref,
strålade af förnöjelse och stolthet samt kastade betydelsefulla blickar
på Jenny, hvilka denna icke förstod eller icke låtsade förstå.

Det lilla sällskapet satt församladt omkring thebordet då Ros, efter att
ha vexlat en leende blick med Erik, hviskade något till tant Betty, som
derefter mycket förvånad lemnade rummet för att dock snart återkomma,
åtföljd af en tjensteflicka som framsatte en bricka med höga glas och en
butelj champagne. Ros ifyllde glasen och uppmanade vännerna att med
honom förena sig uti en välgångsskål för hans vän Erik Stenrot, som med
dagens post fått en glad underrättelse från Stockholm, den nämligen, att
han erhållit ordinarie anställning vid bergsstaten. Skålen dracks under
lifliga lyckönskningar från alla sidor och de två tanterna voro icke de
som minst gladde sig åt vår unge väns framgång.

Jenny räckte sin kusin handen och sade leende:

``Lycka till, nu gifter du dig bestämdt snart, snälle Erik'', hvaröfver
denne blef något förlägen och tant Agatha alldeles ``perplex'',
isynnerhet som hon icke kunde begripa, hvarföre Aina Ros skulle rodna
och Jenny bara skratta. Det var i den gamla damens tycke mycket
opassande å hvardera sidan och hon ämnade just ge sin systerdotter en
vink derom att unga flickor icke brukade fria sjelf, då Erik raskt
uppsteg och, fattande den ännu djupare rodnande Aina vid handen samt
glädtigt nickande åt Birger, med strålande anlete och fri från all
förlägenhet förklarade att Aina och han voro trolofvade. Han utbad sig
de båda tanternas välsignelse, ``ty Jenny, den skalken, har gifvit mig
sin välsignelse på förhand.''

Nu följde omfamningar och kyssar i tillbörlig mängd. Tant Betty tycktes
liksom ha förutsett att så ske skulle, men den nog mycket öfverraskade
riktiga tanten utbrast, likväl i den vänligaste ton i verlden: ``Ja, det
der hade jag bort förutse, då du i ditt bref till Jenny ordade så mycket
om fennomanskorna.'' Den goda damen sökte derefter med sin blick Jenny,
men mötte icke sin systerdotters öga. Hvad tant Agatha deremot såg
tillfredsställde henne icke, ehuru hon ganska väl drog sig till minnes
att Jenny för längesedan förklarat att hon och Erik blott höllo af
hvarandra som kusiner. Nu lutade Jenny tyst sin panna mot Ainas axel:
den öfvergifna, såsom tanten tyckte, stödjande sig på den lyckliga. Men
denna bittra föreställning gaf dock snart rum för hennes hjertas behof
att deltaga i det unga parets sällhet. Icke länge dröjde heller Jenny i
Ainas armar utan hon lyfte åter sitt lilla hufvud, och skakande trotsigt
sina mörka lockar tryckte hon ett ögonblick derefter med en blick full
af tro och hopp Birger Ros' hand och skyndade sedan att muntert
lyckönska de två tanterna.

Qvällen förflöt under hvarjehanda glädtigt samspråk; blott öfver Birger
Ros' breda panna hvilade ett moln som hans starka vilja förgäfves sökte
förjaga. Han var öfver hufvud taget ganska fåordig och då hans blick
någon gång, liksom af en händelse, mötte Jennys, kunde den anande
flickan, trots alla hans bemödanden att dölja det, upptäcka att ett
djupt vemod åter tyngde på den älskades själ. Intet ord om kärlek hade
sedan det betydelsefulla sammanträffandet vid Imatra blifvit utbytt dem
imellan, och ehuru de ofta i hvarandras sällskap besökt det ställe der
Birgers sinnesnärvaro räddat dens lif, som utgjorde föremålet för hans
lika djupa och passionerade som stumma och grannlaga kärlek, ehuru de
ofta besökt denna för båda så minnesvärda plats, der naturen af deras
ömsesidiga känslor uppenbarat sig på ett så egendomligt sätt, så hade
dock endast deras blickar talat -- hans om djup sorg och den ödesdigra
edens orygglighet, medan hennes deremot hade belönat hans tysta hyllning
med en lifvande förhoppnings glans i de sköna ögonen. Hvarifrån hon
erhöll kraft till denna förhoppnings upprätthållande var en gåta för
Birger, men att Jenny sjelf trodde på sina ögons löften, derom var han
förvissad. Hennes blickars trösterika evangelium hade för den unge
mannen blifvit ett verkligt själsbehof. -- Han talade ofta med Jenny om
sin framtida verksamhet i fosterländskt syfte. Allt klarare och klarare
framstod för honom öfvertygelsen att blott ett broderligt arbete hand i
hand med den svenska bildningen i Finland, hvilken för detta land
utgjorde dess enda pålitliga föreningslänk med mensklighetens allmänna
framåtskridande i högre frihet och kultur, var dess
framtidsförhoppningars säkraste ankargrund. En utveckling i denna anda
förhindrade, enligt hans åsigt, på intet vis, ja, den fastmera
understödde och påskyndade uppblomstringen af den finska nationaliteten.
Vid sådana tillfallen, då Jennys rika själ med en älskande qvinnas hela
hängifvenhet hängde vid hans läppar, då mötte ofta hans inspirerade öga
det hoppfulla löftet i hennes strålande blick, det löftet att äfven han
skulle blifva en af dem som med gladt mod lade handen till det goda
verket och hon den lyckliga qvinna, vid hvars barm han finge hvila ut
och samla nya krafter till nya strider i ljusets tjenst. Det var andliga
högtidsstunder i bådas lif, kända och njutna endast af de två invigda.

Då stunden nalkades att Ojalaherrskapet skulle bryta upp, förklarade
Birger att han till häst ville åtfölja damerna, något som mycket bidrog
att lugna tant Agatha, hvars inbillning blifvit uppskrämd af
tjenstefolkets berättelser om fångar som rymt från Wiborgs häkte och
skulle uppehålla sig i skogarne deromkring, berättelser som för öfrigt
med mera eller mindre skäl hvarje annalkande höst upprepas i dessa
trakter af Wiborgs län. Emellertid anlände tant Agatha och Jenny
lyckligen till sitt provisoriska hem och Birger Ros anträdde efter ett
kort afsked sin återridt till Muistola. Nästan fullständigt mörker hade
nu inträdt, ty dystra moln jagade på himlen och tilläto månen att blott
då och då upplysa landskapet.

\begin{enumerate}
\def\labelenumi{\arabic{enumi}.}
\setcounter{enumi}{18}
\tightlist
\item
\end{enumerate}

Nattligt äfventyr.

Försjunken i djupa tankar fortsatte emellertid Birger Ros sin ensliga
ridt i den tysta Augustiqvällen. Månen dolde sig bakom moln och mörkret
tilltog isynnerhet då vår ryttare inkom i den alldeles obebodda
skogstrakt af vid pass halfannan fjerdingsvägs längd som han hade att
passera innan han uppnådde Muistolas för öfrigt också blott glest
bebodda egor. Hästen, återhållen af tygeln, tvangs att hejda sin
otålighet medan dess ryttare fortsatte sina betraktelser, hvilka icke
tycktes leda hans sinne på den behagliga stråten till ett fridomstråladt
framtidshem. Birger Ros älskade den andliga kamp för sitt fäderneslands
utveckling, i hvilken han hade för afsigt att snart åter kasta sig, nu
sedan han lyckats att i någon mon utreda den intrasslade härfvan af sina
ekonomiska angelägenheter, hvilkas tillstånd ovilkorligen hänvisade
honom till nödvändigheten att uppsöka åt sig en annan förvärfskälla än
det lilla godsets knappa, hårdt anlitade inkomster. Han älskade denna
kamp emedan han älskade sitt fosterland varmt och uppriktigt och ansåg
det för sin pligt att deråt egna sina krafter. Hans ekonomiska ställning
oroade honom på långt när icke så mycket som den kinkiga belägenhet, i
hvilken han befann sig gentemot sina bundsförvandter på stridens fält
och som i hans ekonomiska betryck endast sågo en passande förevändning
för honom att på ett skickligt sätt draga sig ur deras leder, ty hans
afundsmän predikade i tal och skrift att han blifvit en affälling från
den \emph{finska} saken. Ryktet om det afslagna anbudet från regeringens
sida hade i mycket vanställda former blifvit kringspridt. Det berättades
nämligen att han afslagit detsamma hufvudsakligen blott derföre att det
erbjudna honorariet icke tillfredsställde hans fordringar och i detta
afseende påpekade man, såsom sqvallret städse gör, illvilligt nog, hans
ekonomis klena skick såsom den klippa, emot hvilken hans forna
föresatser strandat. It detta lumpna förtal log Birger stolt, ty han
kände med sig sjelf att han innan kort tydligt skulle ådagalägga sin
karakters renhet och sjelfständighet. Men han log icke åt den afsöndrade
ställning i lifvet som nu fallit på hans lott i följd af brytningen med
hans vapenbröder, hvilkas välmenande men ensidiga ifver dref dem, detta
förutsåg han, i en riktning som ha.ns nu mognade ande icke kunde gilla,
emedan denna riktning ledde förbi, icke till det föresatta målet: den
finska nationalitetens kraftiga höjande. Han var måhända mera finne i
sitt innersta hjerta än mången af de mest öfverspända, så kallade
fennomanerne, men han var icke längre blind för sitt partis brister och
han erkände, emedan han var en högsint ande, tacksamt den verksamma
hjelp som den goda saken rönte af alliansen med dess naturligaste
bundsförvandt i hela verlden, det svenska elementet i Finland. Han hade
på senare tider, ehuru icke utan en svår inre strid och många återfall i
sin förra lidelsefulla och hänsynslösa ensidighet, kommit till den
öfvertygelse att en innerlig förening af landets andliga krafter är den
enda praktiska utväg att med tiden, bredvid och trots den ryska
påtryckningen, kunna i Finland väcka och underhålla ett sannt folkligt
lif. Hvad den svenska talande delen af befolkningen vidkom var det
öfvervägande flertalet af de densamma till buds stående krafterna
villigt att, utan förnekande af sitt andliga samband med det svenska
moderlandets kultur, träda i det gemensamma fosterlandssträfvandets
leder. Ett större motstånd väntade han sig deremot från den fennomanska
sidan, framkallad i icke obetydlig mon af ofta omedveten, i alla fall
obefogad fruktan för det svenska elementets fortfarande andliga
hegemoni, en plats, från hvars bibehållande denna sidas kanske yppersta
män med ädel hängifvenhet för det allmänna bästa frivilligt afstått,
fordrande blott ett lika berättigande för \emph{sin} nationalitet som
för den finska i den gemensamma utvecklingen, och en förnuftig, icke
brådstörtad öfvergång till detta nya skede af Suomifolkets lif såsom en
egen, odelbar nation. Det var en sådan ståndpunkt som Birger Ros
sträfvade att göra fullt klar för sig, ty han insåg att hans personliga
uppgift i det förestående arbetet var ganska svårlöst. Det gällde
nämligen, sådan var den vanne fosterlandsvännens åsigt, ingenting mindre
än att öfvertyga såväl massan af det finska talande folket som ock
särskildt, hvad svårare var, det starka fennomanska partiet om den
oafvisliga nödvändigheten att förbinda sig med det i landet förhanden
varande svenska elementet. Med själen helt och hållet upptagen af detta
svåra värf kände Birger Ros dubbelt tyngre omöjligheten att förverkliga
sin egen individuella lycka och behofvet af den själens styrka som blott
tillfredsställandet af denna naturliga önskan kunde bereda honom, ty
hjertats frid är en god skattkammare att i lifvets strid draga vexel
uppå. Hans väns och systers äfven i yttre form försiggångna
kärleksförening kunde naturligtvis icke underlåta att i ännu bjertare
färgton framställa för hans själ mörkret i hans eget hjertelif. I denna
sak såg han ingen möjlighet till en lycklig utgång. Han längtade derföre
å andra sidan till och med efter den ädla stridens hetta för att i
densamma kunna glömma sin personliga olycka.

Leka ett blindt öde eller väsenden af en högre naturordning, än den vi
menniskor tillhöra, med händelsernas gång i vårt lif? Sitta andar af
högre rang, än vi äro, vid rodret af den makt vi kalla slump? Det fromma
hjertat har bredt en herrlig slöja öfver vår naturs bristfällighet i
detta afseende och kallar denna sin barnaljufva tro en gudomlig försyn,
som omedelbart ingriper äfven i vårt personliga lif; men upproriska
hjertan finnas som våga, till föga båtnad för sitt själslugn, tvifla på
sanningen af denna allt försonande föreställning. Då Ros' högtsträfvande
ande kommit till denna punkt i skärskådandet af hans framtidsutsigter
sammandrog en känsla af onämbar sorg hans varma, liffulla hjerta, som
derjemte hotades att krossas under den nästan öfverväldigande tyngden af
tanken på det glädjelösa lif som jemväl förestod den älskade, till lycka
och lefnadsnjutning berättigade, sköna qvinnan.

Förgäfves sökte han skaka ifrån sig dessa dystra bilder, då lyckligtvis
en yttre omständighet kom hans feberaktigt arbetande själ till hjelp.
Ljudet af ett ovanligt buller ett stycke framför honom nådde nämligen
genom nattens stillhet plötsligt hans öra. Han tyckte sig förnimma något
som liknade en häftig ordvexling, åtföljd af svordomar och utrop,
hvarvid finska och svenska ord vildt blandades om hvarandra. Påskyndande
sin villigt i galopp öfvergående häst befann sig vår enslige ryttare
snart på stället. Vid skenet af den ur molnen frambrytande månen mötte
en besynnerlig syn Birger Ros' förvånade blickar.

En vagn låg kullstjelpt vid vägkanten och två karlar höllo på att, under
utösandet af vilda svordomar och hotelser på finska språket, ur densamma
framdraga en häftigt emotstretande mansperson, som på god svenska bad
våldsverkarne ``draga åt h-e'' samt förgäfves ropade på hjelp af kusken
och skjutsbonden. Dessa lågo nämligen ett stycke derifrån på marken
bevakade af en tredje karl med yxa i handen. Linorna som hållit hästarne
vid vagnen voro afskurna och tistelstången var afbruten. En fjerde
person höll de tre urspända och uppskrämda hästarne några famnar från
stället.

På vida kortare tid än vi behöft för att beskrifva uppträdet hade Ros
öfverskådat situationen och med ett ljudligt utrop af: ``Stå på er
derinne en stund!'' red han utan vidare ikull den på vakt stående karlen
samt vred dervid yxan ur dennes hand.

``Perkele!'' röt en af de vid vagnen sysselsatte, öfverraskade röfrarne
på finska, ``hvem ha vi der?''

Ett döfvande slag af yxan fällde honom till marken. Kusken och
skjutsbonden började nu resa sig.

``Hitåt, hitåt, befallningsman'', ropade Ros med thordönsstämma på
finska, ``hitåt! och tag röfrarne innan de hinna fly!''

Denna krigslist och kamratens fall väckte oro hos den andra vid vagnen
sysselsatte våldsverkaren och då Ros nu styrde sin häst mellan honom och
ekipaget, gaf detta den derinom befintlige resanden rådrum att något
hemta sig från sin förvirring.

``Är ni sårad?'' frågade Ros i det han kastade en blick in i det
kullstjelpta åkdonet.

Han erhöll blott till svar en framräckt revolver och de hastigt utstötta
orden: ``Se här, nu har jag den, skjut!''

Fattande den laddade revolvern såg sig Ros om efter fienderna som,
beväpnade med störar, tycktes vilja upptaga striden. Två skott ur
revolvern hejdade deras framträngande, och då nu kusken och
skjutsbonden, svängande stycken af den brutna tistelstången samt i
förhoppning på den anropade länsmannens ankomst, jemväl nalkades, drogo
sig röfrarne skyndsamt tillbaka. Den karl som hållit hästarne släppte
dessa, hvilka, så godt de kunde med de sårade benen, började lunka utåt
vägen.

``Mina hästar, mina hästar!'' skrek skjutsbonden och ilade efter dem,
medan kusken, på Ros' befallning, hjelpte den resande ur vagnen. Så
snart denne hade fötterna på marken sträckte han på sin lilla, runda
figur och utropade modigt: ``Jag har ännu en revolver'', samt aflossade,
utan att sigta, två eller tre skott efter de i skogsbrynet försvinnande
röfrarne, som flydde med all hast.

``Spara edra skott, herre'', sade nu Ros, som återvände från en rundridt
omkring vagnen och försökte tygla sin uppskrämda häst, hvilket omsider
lyckades honom. ``Är ni sårad eller skadad?''

``Nej, inte det jag vet; jag fick blott en duktig stöt när vagnen
stjelpte'', svarade den resande med genuint svenskt uttal, hvarvid Ros
dock icke vidare fästade sig. Den främmande tillade i det han försigtigt
nalkades Ros: ``Men hejda då er ystre Bucephalus så att jag kan få
trycka min räddares hand. Det var, minsann, hjelp i rättan tid -- uff!''

Ros steg af och öfverlemnade sin häst att hållas åt den resandes kusk,
hvilken såg mycket förlägen ut. Den återvändande skjutsbonden, släpande
efter sig sina ledbrutna hästkrakar, jemrade sig deröfver att de blifvit
skadade vid vagnens stjelpande, och den lille, runde resanden trädde
fram till Ros. Månen utgjöt sitt fullaste sken öfver denna invid den
kullfallne vagnen i den ensliga skogen församlade, egendomliga grupp.

Den resande skakade imellertid hjertligt Ros' hand utan att uttömma sig
i ett onödigt ordsvall och denne vände sig snart derefter till kusken
med en fråga huru det gått till att vagnen stjelpt?

``Jo'', svarade den tillsporde, ``det var så mörkt att man inte kunde se
handen framför sig och röfrarne höllo väl, såsom de bruka, en lina spänd
tvärs öfver vägen, hästarne stupade och vagnen stjelpte i
landsvägsdiket. Skjutsbonden och jag, som sutit på bocken, slängdes i
skogen och stötte oss så att vi förlorade sansen samt inte kommo oss
före innan månen tittade fram och nådig herrn var rackrarne på halsen.
Det var bestämdt tio karlar.''

``Åh nej'', invände Ros, ``jag såg blott tre utom den som höll hästarne.
Men min herre'', tillade han vändande sig till den resande, ``det
klokaste är väl nu att vi söka lyfta upp vagnen och så godt sig göra
låter reparera skadorna samt sedan bege oss till mitt hem, hvilket inte
är så långt härifrån.''

``Tack, tack, min herre'', svarade den resande och alla grepo verket an.
Lyckligtvis voro hjulen oskadade och med tillhjelp af yxan och några
repstumpar voro snart ett slags fimmelstänger bragte till stånd samt
Ros' häst spänd framför vagnen. Kusken fattade tömmarne och följde efter
Ros och den resande herrn, som gingo till fots i motsatt riktning mot
den hittills iakttagna; skjutsbonden med sina skadade hästar slutade
tåget.

\begin{enumerate}
\def\labelenumi{\arabic{enumi}.}
\setcounter{enumi}{19}
\tightlist
\item
\end{enumerate}

Nya öfverraskningar.

Knappast hade den i slutet af förra kapitlet beskrifna karavanen satt
sig i rörelse, innan åter ljudet af menniskoröster hördes. Snart framkom
på en ginväg ur skogen Erik Stenrot med några af gårdens karlar, som
voro beväpnade. Man hade hört skotten och skyndat efter ljudet i den
förmodan att möjligtvis de af ryktet omtalade förlupne fångarne vågat
anfalla någon resande. På qvällen hade man sett en vagn passera förbi
Muistola gård.

``Jag tror'', utropade Erik, icke utan en viss oro då han med blicken
öfverfor det besynnerliga tåget, ``att här varit strid å färde? Nå,
gudskelof, att du är helbregda, Birger! Var det verkligen röfvare och
hvem är det som skjutit?''

``Den herrn'', svarade Ros, ``hade blifvit öfverfallen af några karlar,
troligen förrymda fångar, och ett par skott lossades.''

``Det vill säga'', inföll nu den resande, ``den här herrn, hvars namn
jag inte ens ännu efterfrågat, har enligt all sannolikhet räddat såväl
mitt som ock de der två karlarnes lif.''

Erik studsade synbarligen vid ljudet af den resandes röst och närmade
sig honom för att om möjligt i månskenet uppfånga den resandes
anletsdrag.

``Ert namn, min ädle räddare!'' fortfor denne. ``Var god säg mig hvem
det är, som jag kanske har att tacka för mitt lif?''

``Jag heter Ros och har en egendom här nära intill'', var Birgers svar.

``Nåväl, herr Ros'', återtog den andre, ``ni har gjort mig och dessa två
menniskor en ganska väsentlig tjenst och gamle Jakob Stråle från
Stockholm är eder tacksamme vän.''

``Hvad!'' utropade, utom sig af förvåning, Erik och fattade ifrigt den
resandes hand, ``morbror Stråle! Hvad i Guds namn för morbror hit?''

Den gamle herrn, ty Stråle var en äldre man, betraktade med icke mindre
förundran vår vän Erik och utbrast:

``Och hvad, hvad gör \emph{du} här, här midt i ödemarken, gunstig herre?
Jag tror minsann att detta Finland är ett förtrolladt land. Men jag kan
gissa, om inte hvad du gör här, så dock ungefär hvem som lockat dig just
hit. Huru fick du reda på Jenny?''

``Till höger kusk'', hördes nu Ros' stämma och en af karlarne skyndade
att öppna grinden. Efter några befallningar till karlarne vände sig Ros
till den resande och sade:

``Välkommen hos mig, herr Stråle, men ursäkta att jag straxt åter måste
aflägsna mig för att varna mina grannar på Ojala, dit rymmarne
möjligtvis tagit kosan. Min vän Erik, var du värd i mitt ställe. Farväl,
tills i morgon.''

Och i yrande sporrsträck bar det af på vägen till Ojala på den från
vagnen urspända hästen. Två beridna drängar åtföljde Ros, beväpnade med
hvar sin bössa och yxa.

``Rask karl, den der Ros'', sade gamle Stråle, ``men Ojala, hvad har han
att skaffa med Ojala?''

``Stig in, morbror'', sade Erik och tillade sedan tveksamt: ``Då morbror
befinner sig i denna trakt af verlden, så trodde jag att det var bekant
hvem som bodde på Ojala.''

``Visst vet jag det'', menade Stråle, ``och jag var just på väg dit
då\ldots{}''

I detta ögonblick skyndade emot dem ett ungt fruntimmer. Det var Aina.

``Ack, Erik!'' utropade hon, ``hvar är Birger? Tjenstefolket har
alldeles skrämt upp tant Betty, så att\ldots{}''

Hon tystnade rodnande då hon nu först bemärkte den främmande herrn.

``Med Birger har det ingen fara'', genmälte Erik, ``han har blott ridit
tillbaka till Ojala för att der hålla vakt, och när vi väl kommit in,
skolen I få höra det intressanta äfventyret af den här herrns egen
mun.''

Med dessa ord förde han den främmande till Birgers rum, hvarest herr
Stråle i hast iordningställde sin något illa medfarna drägt. Under tiden
frågade han Erik, hvem det unga fruntimret var som kallat honom vid
namn?

``Min värds syster'', svarade Erik undvikande. ``Men se så, morbror,
stig in till fruntimren och tillfredsställ deras och min nyfikenhet
angående den förskräckliga händelsen på landsvägen.''

Vi öfverhoppa presentationerna, hvarvid intet nämdes om förlofningen,
äfvensom herr Stråles berättelse om förloppet vid röfvarnes anfall samt
hans och de två karlarnes räddning genom Ros' modiga mellankomst. Efter
några utrop af förvåning och fasa samt glädje att ``allt slutats så
väl'' inbjöd tant Betty med landtlig artighet den ärade gästen att efter
så mycken skrämsel och ansträngning hålla tillgodo ett glas ``varmt''.
Herr Stråle, en jovialisk, gammal ungkarl, tackade fryntligt för
inbjudningen och bryggde sig en rätt stadig toddy, drack och hade snart
så intagit båda damerna till sin fördel att tant Betty, då hon efter den
händelserika dagen ändtligen kom i säng, sade till Aina: ``då alla
svenskar jag hittills träffat äro så hyggliga karlar, så är det väl bara
prat hvad pastorsadjunkten (som sannolikt var fennoman) brukar säga,
nämligen att svenskarne `kuggat' Finland.'' Aina åter var rätt belåten
dermed att hennes fästmans morbror varit så vänlig emot henne. Då kunde
han väl icke heller ha något emot deras förlofning. -- Efter intagen
landtlig qvällsvard åtskildes sällskapet och lade sig till hvila icke
utan att dock egna ett visst deltagande åt den frånvarande Birger, som
begifvit sig af för att hålla vakt öfver fruntimrens på Ojala nattro.

``Den vakten gör han gerna'', hade tant Betty yttrat, hvartill herr
Stråle svarat ett menande ``jaså''. I öfrigt tycktes den främmande herrn
helst undvika att tala om ändamålet med sitt oförväntade besök hos sin
syster och sin myndling Jenny.

\begin{enumerate}
\def\labelenumi{\arabic{enumi}.}
\setcounter{enumi}{20}
\tightlist
\item
\end{enumerate}

Vid kaffebordet.

Den första menniska som den gladlynte stockholmaren mötte då han
följande morgon efter omsorgsfullt gjord toilett utträdde på husets
lilla balkong var hans räddare från gårdagsqvällen, hvilken redan
återvändt från sin nattliga vaktgöring och nu, gående sin gäst till
mötes, helsade honom välkommen. Ehuru Birger, som af tant Betty fått
höra att deras gäst var en slägting till såväl Erik som ock
Ojalaherrskapet, gerna velat utforska den gamle herrns egentliga afsigt
med detta oväntade besök, hvarom ingenting förljudits i den förtroliga
sällskapskretsen, hade han dock häri vida mindre framgång än den
fintlige juristen i det välbemantlade förhör han lät sin värd undergå.
Understödd af de meddelanden Stråle redan qvällen förut under samtalets
lopp lyckats erhålla, visste han nu att så leda tråden af deras samspråk
att han ganska snart trodde sig vara det sanna förhållandet på spåren.
Den väl öfverståndna nattliga vakten gaf honom härvid en ypperlig
anledning till hvarjehanda spörjsmål. Stråle tackade varmt men icke utan
en liten skymt af skalkaktighet i blicken den unge doktorn för dennes
omsorg om hans systers säkerhet samt prisade den Stråleska familjen
lycklig som i det aflägsna Finland i en och samma person vunnit en
räddare, en beskyddare och en tvifvelsutan mycket förekommande gästvän.
Om Jenny nämde den gamle herrn intet enda ord. Men då Erik, som med sin
närvaro ökat sällskapet, frågade huruvida det icke vore lämpligt att han
skulle rida ut till Ojala för att förbereda tant Agatha och kusin Jenny
på det oväntade besöket, svarade herr Stråle helt enkelt att då Erik
öfverlemnat den för behöflig ansedda nattvakten öfver slägtingarnes
säkerhet åt en annan, han lika gerna kunde låta morbrodern sjelf anmäla
sitt besök. Arten af det svar, som tycktes sväfva på Eriks läppar,
föranledde honom dock att skämtsamt tillägga: ``att det tvifvelsutan
fanns heligare föremål att bevaka än unga kusiner, till exempel gamla
onklar, att icke tala om den frånvarande gästvännens härd.''

Erik rodnade märkbart men han hjelptes ur sin förlägenhet af den vackra
Ainas inträde, som, efter ett glädtigt och lika hjertligt ``godmorgon''
till alla tre herrarne, började servera kaffet. Birger omtalade nu att
han till frukosten ditväntade länsmannen som skulle anställa
polisundersökning angående det nattliga öfverfallet, samt öfvertaga
ledningen af våldsverkarnes efterspanande, om hvilkas sannolika tillhåll
man dessutom redan fått någon kunskap, väggledd af de spår de vid sin
flykt lemnat efter sig.

``Stackars karlar'', sade herr Stråle, ``jag skulle nästan önska dem
friheten, redan derföre att deras misslyckade anfall beredt mig en så
öfverraskande angenäm bekantskap som min räddares och fröken Ainas.''

``Säg lika gerna blott Aina, snälla morbror'', inföll Erik, ``ty jag
säger det genast rent ut. Aina är min fästmö.''

``Då säger jag med synnerlig glädje `Aina' till den älskvärdaste af mina
finska vänner,'' utropade den gamle herrn muntert och fattade med en
anstrykning af galanteri Ainas hand för att kyssa densamma, men den
rodnande unga flickan slog sina vackra armar omkring onkelns hals och
tryckte en kyss på den gamle ungkarlens ingalunda motsträfviga mun.

``Bra gjordt, Aina lilla!'' sade den upprymde juristen; ``men blif icke
svartsjuk på mig, Erik, ehuru jag är ungkarl'', tillade han med en
komisk suck. -- Han ökade den allmänna sinnesstämningens munterhet, då
han med utsökt belefvenhet och gammaldags sirlig artighet framträdde
till tant Betty, som tillkommit under tiden, och vördsamt anhöll att i
betraktande af det blifvande intima förhållandet emellan deras familjer,
nu redan få kalla fröken Ros för ``kusin'', hvartill den fryntliga gamla
damen rodnande gaf sitt samtycke. Med Birger utbytte Stråle ett
hjertligt handslag i det han menade: att det var mycket roligare att
kalla sin räddare du än herr doktor, hvilken titel ovilkorligen påminte
honom om läkare, menniskor som han väl i allmänhet högaktade, men af
hvilka han hittills icke behöft och hoppades att icke så snart heller
behöfva någon ``räddning''.

Ett bud från länsmannen anlände emellertid och aflemnade ett bref. Detta
gaf anledning till uppbrott från kaffebordet. Det högvigtiga brefvets
innehåll skola vi meddela i nästa kapitel.

\begin{enumerate}
\def\labelenumi{\arabic{enumi}.}
\setcounter{enumi}{21}
\tightlist
\item
\end{enumerate}

Länsmans-diplomati.

Den värde herr Smilander hade af ryktet förnummit att den öfverfallne,
men räddade utländingen var en mäkta förnäm man, hvilken kommit hit för
att från Ojala afhemta de der boende utländska damerna samt återföra dem
till deras i hemlighet öfvergifna hemland. Genast vaknade underliga
tankar i hans hjerna. Kanske var det sjelfve polismästaren i Stockholm,
den berömde Wallenberg, som tillika med egna ögon här vid Imatra ville
studera de naturliga vattnens kraft. Inom sig hade derföre den
förtänksamme länsman Smilander beslutit att på allt möjligt sätt gå den
förnäme främlingens önskningar till mötes, och till och med Ros steg i
hans aktning i följd deraf att han varit försynens redskap att
verkställa räddningen. Då han erfarit att den resande under titel och
namn af ``häradshöfding Stråle'' hedrade hans aflägsna födelsebygd med
sitt besök, stadfästades inom honom den en gång fattade föreställningen
att denne var en mycket förnäm man, som allrahelst, för att väcka mindre
uppseende, ``inkognito'' gjorde sin resa. Den främmande hade
tvifvelsutan antagit namnet Stråle blott för att låtsa vara en slägting
till den gamla damen på Ojala. Smilander ville derföre på ett försigtigt
sätt låta honom förstå, att han, Smilander, var en alltför routinerad
administrativ tjenstemän för att icke genast ha genomskådat den resandes
högre samhällsställning. I enlighet härmed ville han ock från första
början inrätta hela sitt uppträdande. Wallenberg-Stråle skulle då med
sitt vana öga genast se hvars andas barn han hade framför sig, och man
kunde icke veta hvilken utländsk utmärkelse möjligtvis blefve en följd
af det nit Smilander skulle ådagalägga vid röfrarnes förföljande och
gripande. Han ansåg det derföre vara klokare att icke besöka Ros till
frukosten, utan medelst ett skickligt och i högre juridisk stil uppsatt
bref inbjuda främlingen till den polisundersökning han ernade hålla på
sjelfva stället der öfverfallet egt rum. Detta bref, som saknade adress,
sände han med ett bud till Muistola. Det skulle öfverlemnas till den
``främmande öfverfallne herrn''. Stråle tog brefvet och läste icke utan
förvåning följande löjliga aktstycke:

\begin{verbatim}
"Min herre!

I brist på adress som jag saknar vill jag icke tro på ryktet om
min herres namn hvilket, såsom måhända varande falskt, kunde
inverka kränkande; men till saken, såsom vi administrativa
tjenstemän säga.

Min herre har natten emot i dag, den 16:de hujus, genom herr
doktor Ros' förvållande lupit fara att blifva räddad, eller
fastmera af doktor Ros blifvit räddad ur en fara som min
herre bemälde natt lupit genom ett ännu okändt antal röfvares
förvållande, hvadan jag i min egenskap af kronobefallningsman
i den af denna olycka hemsökta socknen, på anmodan af sagde
Ros, bosatt å Muistola, funnit mig föranlåten att dels medelst
utsändande af spejare, äfvensom prolimitär polisundersökning på
platsen, dels ock genom hållandet i förvar, dock icke i jern,
af sistlidne nattens skjutsbonde, såsom skäligen misstänkt att
ha varit med i komplotten beträffande sina hästar, som icke
blifvit honom frånstulna och derföre skola afföras till mitt
stall för att besigtigas af vid slika försök till rån vane
personer, som skola fastställa huruvida hästarne blifvit med
flit skonade, -- så och i betraktande nu af allt detta och dessa
åtgärders vidtagande, får jag härmedelst förständiga, det vill
säga ödmjukast anhålla det min herre ville infinna sig å den
plats der det tillämnade brottet lyckligtvis icke föröfvats
genom meranämnde Ros' föregifna mellankomst och såsom af vittnen
vidhandengifvits, fem med välberådt mod ur min herres revolvrar
aflossade skott, för att jag på stället i behörig form måtte få
anställa polisförhör och undersökning med alla vid tillfället
utom röfrarne närvarande personer samt likaså den kullstjelpta
vagnens besigtigande för att utröna huruvida våldsverkarne
medelst inbrott försökt någonting; och ehuru smärtsamt det är
att innan röfrarne genom gripande och häktande befordras till
laga straff, dermed se mig tvungen att besvära min herre och
afgifva berättelse om hela förloppet, ithy att vi administrativa
tjenstemän i Finland, i likhet med hvad i Sverge plägar ske,
måste gå summariskt tillväga då det är fråga om rackare och
förlupna personer, bedjande tillika om ursäkt för besväret, men
utlofvande min herre att med all behöflig handräckning bistå min
herre i den der andra saken, hvilken längesedan icke blott väckt
min nyfikenhet såsom varande oloflig, utan derhos föranledt ett
för mig illa aflupet besök på ett ställe, som må vara onämdt
oss emellan, och hvilket min herre ämnade besöka redan i går
men nu måste uppskjuta till i dag, men der enligt egen uppgift
inga dokumenterande handlingar finnas, hvilket jag vill till min
herres tjenst och upplysning meddela, ehuru jag tror mig kunna
framtvinga riktiga pass om icke falska sådana redan anskaffats,
och förblir jag med högaktning och i förbidan på min herres
ankomst till det olyckliga räddningsstället men der ingen fara
finns om dagen och derföre att jag är der, min herres ödmjukaste
tjenare.

                                      _Johan Smilander_,
                                 Kronobefallningsman i orten."
\end{verbatim}

Efter genomläsandet af denna sorglustiga skrifvelse öfverlemnade herr
Stråle densamma skrattande åt Ros, i det han sade sig tro meningen vara
en kallelse att infinna sig vid en preliminär polisundersökning på
stället. ``Men slutet förstår jag inte riktigt'', tillade den gamle
juristen, ``dock förmodar jag att det har afseende på någonting som rör
min syster.'' Ros genomläste leende brefvet men då han kom till frågan
om passen blef han allvarsam. Under tiden hade Stråle följt ``kusin
Bettys'' inbjudning till frukostbordet. Nu erhöll ändtligen Erik
papperet. Han försökte uppläsa detsamma högt men afbröts som oftast af
en skrattlust, den han icke kunde qväfva; dock upphörde äfven hans
munterhet då han kom till det ställe der passen voro omnämda. Men hans
oro yppade sig på annat sätt än finnens, som blott eftersinnade hvilka
obehag fruntimren kunde ha att befara af den påhängsne länsmannen. Erik
gick denna gång längre i sina farhågor än vännen.

``Gubben Stråle är Jennys förmyndare'', hviskade han till Ros och detta
var en nyhet som tycktes börja oroa denne, isynnerhet då han
sammanställde densamma med hvad tant Betty meddelat honom om kuskens
vinkar angående ändamålet med den främmande herrns besök på Ojala. Sedan
emellertid frukosten intagits begåfvo sig herrarne på väg till platsen
för föregående qvällens äfventyr.

\begin{enumerate}
\def\labelenumi{\arabic{enumi}.}
\setcounter{enumi}{22}
\tightlist
\item
\end{enumerate}

Polisundersökning i skogen.

Men derborta i den nu åter fredliga skogen, hvarest ännu för tolf timmar
sedan hästar störtat, vagnar blifvit kullstjelpta och resande anfallits,
der yxor svängts i månskenet och den krigiska knalleffekten af
amerikanaren Colts uppfinning, revolvern, injagat skräck hos den
europeiska nordens landtligt beväpnade banditer, der satt nu den
administrativa maktens representant i den finska socknen Ruokolaks,
kronobefällningsman Johan Smilander, på en sten vid landsvägskanten. Han
satt der omgifven af två nämdemän samt bro- och skallfogde,
föreställande hans myndighets armar och ben, hans handtlangare och
adjutanter. De rådplägade sinsemellan med vigtiga miner i de välmående
ansigtena, och en liten skara af ortens allmoge, män, qvinnor och barn,
stod gapande rundt omkring dem. Då syntes ett dammoln på vägen från
Muistola och herr Stråle i vagn samt Birger och Erik till häst anlände.
Herr Smilander steg upp och förberedde sig att med en viss högtidlighet
emottaga de ankommande, som i en ståtlig procession närmade sig. Han
``harklade'' och hostade, fick strupen ändtligen klar och skulle just
börja sitt tillämnade tal, då den raske gamle Stråle, sedan han
återvunnit sin jemnvigt efter nedstigandet ur vännen Ros' gammalmodiga
och något höga vagn, med en oefterhärmlig komisk grandezza tilltalade
den vördsamt helsande sålunda:

``God morgon! Ackurat på sin post, det ser man! Edert nit i tjensten,
herr befallningsman, är öfver mitt beröm.''

``Tackar ödmjukast'', svarade Smilander och fattade mod då Stråle räckte
honom handen till helsning. Han tillade derefter: ``Vi administrativa
tjenstemän i Finland äro mycket påpasslige.''

``Ja, och tiden är dyrbar'', återtog herr Stråle, ``jag borde om en
timme eller så vara på Ojala.''

``Vi kunna genast skrida till verket'', inföll länsmannen, ``ursäkta att
jag sätter mig.''

På länsmannens fråga huruvida den öfverfallne hyste misstanke till någon
särskild person -- han blinkade dervid med ögonen -- om att ha varit i
komplott med röfrarne, svarade Stråle ``nej'' och afgaf derefter i
korthet en berättelse om förloppet. Ros gjorde likaså om hvad som timat
efter hans inblandning i saken. Nu uppropades den pratsjuke kusken. På
dennes vittnesmål hade länsmannen synbarligen räknat för att såsom
ransakare lysa inför den förmente höge polismannen.

``Kurikka'', började länsmannen, ``ni är från Wiborg, inte sannt? Jag
tror mig känna er?''

``Isvoschtschik (åkare) Kurikka n:o 6'', svarade denne på dålig svenska.
``Jag kände herr befallningsman mycket bra då herrn var i kansliet i
Wiborg. Mången gång om natten\ldots{}''

``Det hör inte hit!'' afbröt den förhörande. ``Huru kom Kurikka i den
här herrns tjenst?'' Smilander bugade sig härvid för Stråle.

``Jo si, för det att jag är gammal kusk och kan `rata venska' (prata
svenska), så gaf polismästaren mig åt herrn här, som jag följer på resan
och tillbaka för sex mark (4 rdr) om dagen.''

``Polismästaren och sex mark om dagen'', sade Smilander för sig sjelf,
``det är \emph{han} och ingen annan. Tillåt mig'', han vände sig nu till
Stråle, ``att förhöra vittnet på finska. Doktor Ros kan ju vara er
tolk?''

``Alltför gerna'', blef svaret. Nu tilltalade Smilander kusken med hög
röst sålunda:

``Tala rent ut, Kurikka, hvad du har att säga. Jag vill att lag och rätt
skola vördas i mitt distrikt'' -- härvid kastade han en sträng blick på
de församlade bönderna -- ``tala rent ut allt hvad du vet om den här
saken och hvarföre röfrarne just anföllo den här herrn?''

Efter erhållen tillåtelse att få begagna sig af sitt modersmål, afgaf
Kurikka en vidlyftig berättelse, den vi, med förbigående af en mängd
biomständigheter, här återgifva betydligt förkortad. Emellertid satte
sig herr Stråle och hans följeslagare på ett par stenar för att höra på.

``Jo, si, det var så'', började vittnet, ``att jag blef den rike herrns
kusk fast Järvinen, min granne, rände andan ur halsen på sig för att få
den platsen, men polismästaren sa', att jag var mera `skicklig i ett och
annat'. Och majoren, polismästaren ni vet, herr befallningsman och `höga
rättvisa' (`korkia oikeus'), är mycket gemen när han talar med simpelt
folk, och han sade till mig, att djefvulen och, med respekt till
sägandes, alla länsmän i landet skulle ta mig, om jag inte skulle föra
den herrn helbregda tillbaka till Wiborg. Och såsom ni ser, höge herr
befäliningsman och domare, så tyckes ju den främmande herrn vara kry och
rask hittills, så att jag ärligen förtjenat min lön.''

``Till saken'', afbröt länsmannen, ``berätta, huru kom det till att ni
råkade ut för den här rysligheten?''

``Jo, si'', började kusken åter, ``det var så, att vi skulle fara till
två ohyggligt rika fruntimmer, som lära ha rymt från\ldots{}''

``Ursäkta, herr befallningsman'', afbröt Stråle, ``kanske vi lemna
fruntimren åsido, alldenstund denna sak alldeles inte angår dem.''

``Riktigt'', anmärkte ransakaren till vittnet, ``berätta hvad som händt,
inte hvad som kan komma att hända.''

``Nåja'', fortfor den sålunda tillrättavisade, ``vi skulle till Imatra,
och derifrån med fyra hästar till Lauritsala (första slussen vid Saima
kanal) och så till Wiborg igen. Nåväl, resan gick bra ända till Jäskis
färja, och medan herrn drack kaffe inne i gästgifvargården, frågade
folket, som stod omkring mig, hvad det var för en herre, som jag
skjutsade, och jag sa' sanningen, att det var en rik man från utlandet,
som reste till Ojala och derifrån tillbaka till Wiborg. Och som jag fått
respengarna om hand, gaf jag skjutsbonden från Rautanen en mark (70 öre)
i drickspengar, för si så hade herrn befallt. Och der talades ett och
hvarje om Ojala-herrskapet och att de också voro från utlandet och hade
mycket pengar samt att Muistolaherrn (härvid rodnade Ros starkt då han
öfversatte vittnets ord) blifvit mycket god vän med dem, och att han
hade hos sig en herre, som var från samma land som fruntimren, och att
de sällskapade med hvarandra såsom slägtingar. Nåväl, det var allehanda
folk vid färjan och vår nye skjutsbonde hörde också uppå, men herrn
derinne dröjde så länge, att färjan gjorde två resor innan vi kommo att
fara öfver elfven. Flera karlar hade före oss begifvit sig till andra
sidan. Det var mycket bråk med vagnen, men vi kommo dock i skymningen
öfver. Orsaken var den, att herrn i Wiborg spisat middag med
polismästarn, borgmästarn och många andra, så att, ehuru vi körde fort,
det likväl blifvit sent på dagen. Så bar det åter i väg, men det var
mörkt när vi kommo till den här f-e skogen, och huru det var, så
skyggade venstra sidohästen och vi lågo här i diket, vagn och hästar och
allt, just der. Jag och skjutsbonden fingo en så duktig knuff i fallet,
att vi voro sanslösa, kan jag tro, ty när vi vaknade igen var det
månsken och ett hiskligt svärjande och skrikande och huggande, och
herrarne skjöto, jag tror, fem skott. Röfrarne, hvilkas antal jag inte
räknat, flydde till skogs, och hästarne, som de spännt ifrån vagnen för
att stjäla dem med, rymde, men skjutskarlen fångade dem åter, ty de hade
skadat sig i benen då de stupade på repet, som röfrarne spännt öfver
vägen, och hvilket Muistola-herrn tog upp från landsvägen; det låg så
här, just midt öfver vägen.''

Sedan kusken slutat sin långa berättelse framfördes skjutsbonden, hvars
utsago öfverensstämde med den förres. Han tillade blott, att kusken i
mycket skrytsamma ordalag talat om den resandes rikedom och många
kappsäckar med dyrbart innehåll samt en låda med välinpackade
vinbuteljer som de hade bakbunden på vagnen.

``Kände du'', sporde länsmannen, ``någon af de karlar, som åhörde detta
tal?''

``Jo några, och det är hederligt folk, men så voro der också två
zigenare (tattare) och de foro öfver första gången färjan gick, det mins
jag väl.''

Sålunda fortgick förhöret ännu en stund. Derefter undersöktes platsen,
och Ros utpekade ett träd på den ena af vägkanterna, vid hvilket det
ödesdigra repet varit fästadt, medan det på andra sidan sannolikt
hållits af karlarne. Då hästarne stupat emot den blott qvartstums tjocka
linan, hade den häftiga ryckningen i densamma lemnat tydliga spår efter
sig i trädets bark. Några mindre blodfläckar utvisade platserna der
striden stått och i hvilken riktning våldsverkarne flytt. Stråle och
herrarne togo nu afsked af länsmannen, som skulle bege sig på spaning
efter röfrarne. Den stackars skjutsbonden sattes emellertid på fri fot
och Stråle gaf honom en riklig gåfva för att påkosta hästarnes
återställande.

``Välkommen i qväll till Ojala på en liten toddy, herr befallningsman!''
sade herr Stråle artigt. ``Jag hoppas min syster inte skall ha något
deremot, när det är jag som bjuder, och inte min vackra systerdotter
heller.''

I detta ögonblick och innan Stråle ännu hunnit klifva upp i vagnen,
framstälde sig inför den icke litet öfverraskade församlingens blickar
en ny, lika vacker som i dessa trakter i allmänhet sällspord syn. På
vägen från Ojala nalkades, styrande med behag sin lifliga häst,
gestalten af en ung ryttarinna. Hon parerade skickligt sin muntra
springare i det hon helsade den förvånade församlingen och vände sig
till Erik, som var den förste hon kände igen, med den något hastiga
frågan:

``Hvad har händt, kusin Erik? Folket på Ojala är alldeles uppskrämdt och
tant lät mig inte bege mig af utan den der beväpnade drabanten.''

Hon pekade på en med bössa utrustad karl, som ridande barbacka på en
arbetskamp, följde efter henne med andan i halsen. Men då hon härvid
gjorde en vändning åt sidan, blef hon varse gubben Stråle, som med
utropet: ``Men der har jag, minsann, henne sjelf, min yra, präktiga
amazon!'' skyndade henne till mötes. Ros, som sutit af, räckte Jenny
artigt sin hand och hon hoppade muntert i den glade herr Stråles armar
och gaf honom, oombedd, en hjertlig kyss på hans mysande mun.

``Välkommen, snälla morbror, välkommen ändtligen!'' utropade den vackra
ryttarinnan och tillade sedan: ``Och det är kanske morbror som haft det
der äfventyret i natt?''

``Jo du, min sprittande Jenny d'Ojala'', svarade Stråle glädtigt, ``och
der ser du min räddare ur en den allra sannolikaste dödsfara jag
hittills i mitt lif lupit; det är just din nyaste stallmästare.''

``Herr Ros!'' sade Jenny med en blick, omöjlig att beskrifva, men så
full af den renaste, mest oförstälda glädje, att dess uttryck icke kunde
undgå morbrodren, och han tillade derför i allvarligare ton:

``Ja, Jenny, vill du tacka den ynglingen för mitt lif och kanske två
andra menniskors till, så gif den finske riddaren ett hjertligt
handslag, du min lilla svenska affälling.''

Djupt rodnande räckte Jenny sin hand åt den stumme Ros.

``Se så, nu kan det vara nog, till en början åtminstone'', återtog
Stråle; ``stig upp till mig i vagnen, så åka vi tillsammans till Ojala.
Jag hoppas Agatha mår väl? Kom ihåg bjudningen, herr befallningsman, och
du, broder Ros, tag dina fruntimmer med dig! Jag skall genast sända
vagnen tillbaka. Erik följer med oss.''

Så ordnade den gamle herrn allt för den dagens afton.

Men Smilander kastade en tacksam blick på de afresande och mumlade:
``Han vill mig väl, jag måste, ta mig f-n, ha reda på de der
rackrarne.''

Vi lemna nu länsmannen och hans följe, der de satte sig i rörelse till
en enslig skogsäng, hvarest man förmodade att våldsverkarne i en gammal
lada hade sitt tillhåll.

\begin{enumerate}
\def\labelenumi{\arabic{enumi}.}
\setcounter{enumi}{23}
\tightlist
\item
\end{enumerate}

Hvarför Jenny kom till skogen.

``Det stundar främmande, kära du'', sade tant Agatha till skön Jenny på
morgonen af den dag som följde efter Eriks och Ainas trolofningsafton,
``jag försäkrar, att det har stundat långväga främmande hela morgonen.
Nå, Erik kommer visst ut med sin fästmö. Ja, den Erik, den Erik'',
fortfor den gamla damen, ``att så der ge dig på båten, Jenny lilla, det
var ändå styggt.''

``Det var snällt och klokt gjordt af Erik att inte äflas med att fria
till mig, tant lilla, ty jag hade varit tvungen att ge honom korgen'',
svarade Jenny med den naturligaste ton i verlden.

``Ja, Gud blott förstår sig på flickhjertan nu för tiden!'' återtog
tanten. ``Och jag som trodde, att du tyckte om honom, efter han alltid
skulle följa med på våra resor?''

``Det var just för det jag inte alls var kär i kusin Erik, som han var
en passande och kär reskamrat för mig -- men'', afbröt hon det grannlaga
ämnet, ``hvad kommer åt vår Thilda? Hon ser ju alldeles blek och
förskrämd ut. Hvad fattas dig, Thilda?''

``Ack, kära hjertandes fröken!'' ropade tjenstflickan, der hon med
förfärad uppsyn ilade från drängstugan till de i förstuguqvisten stående
damerna, ``rysliga saker ha händt i natt, vet herrskapet.''

Tant Agatha och Jenny sågo förvånade på den andfådda flickan.

``Jo'', fortfor denna, ``i natt har en rysligt rik resande herre blifvit
öfverfallen af röfvare på landsvägen häremellan och Muistola och
röfrarne höllo just på att slå ihjel honom, men så kom der en lång,
stark karl på en svart häst och red öfver dem och slog dem och der
skjöts och den svarte ryttaren dref bort röfrarne och försvann, men den
gamle resande herrn, ty han var gammal och från ett främmande land,
fördes till Muistola och doktorn\ldots{}'' Thilda hemtade andan ett
ögonblick.

``Och doktorn?'' frågade Jenny ifrigt.

``Jo, han har visst bedt oss tiga för att inte oroa herrskapet'', sade
Thilda, ``men si, det kan jag inte. Jo, doktorn, han kom hit om natten
med två drängar, som hade bössor, och han höll vakt hela natten, men om
morgonen sände han bud till länsman och så red han hem och bad mig tiga
med alltsammans, men si, det kan jag inte, och den ene drängen är ännu
med sin bössa qvar i stugan -- herre Gud, huru rysligt!''

Jenny lät genast kalla den omnämde drängen för att af honom erhålla
litet bättre besked om hvad som händt; men som samtalet fördes med den
uppskrämda Thilda till tolk, så kom der icke mycket ljus i saken. Den
främmande resandens kusk hade, enligt drängens utsago, berättat att en
mörk ryttare kommit fram i månskenet och liksom från himlen samt jagat
bort röfrarne.

``Var det doktorn?'' frågade Jenny med klappande hjerta, ``och blef den
resande sårad?''

``Ne-ej'', ljöd svaret, ``doktorn, Pelle och jag ha vakat här hela
natten och i morgse red Pelle till länsman.''

``Och är den främmande herrn sårad?''

``Det vet jag inte, men nu är han på Muistola och nog kom han dit till
fots, det såg jag sjelf.''

Jenny befalte att hennes häst genast skulle sadlas. Hon ville och måste
höra åt i Muistola hvad som timat, ty hon var så orolig för den
``främmande gamle herrns öde'', sade hon.

Tant Agathas föreställningar hjelpte till ingenting. Jenny bad henne
blott för all del vara lugn, något som hon sjelf imellertid förgäfves
bemödade sig att synas. Slutligen förmådde tanten henne att såsom
beskyddare medtaga den beväpnade drängen från Muistola. På Ojala
qvarstannade landbonden och hans två söner hemma.

Med hjertat uppfyldt af oro och ängsliga aningar anträdde Jenny sin
ridt, och vi ha sett, att dessa aningar på sätt och vis icke bedragit
henne, ehuru allt aflupit lyckligt nog. Sålunda kom det att Jenny
plötsligt uppenbarade sig för de i skogen församlade männen, och mången
gammal bondkäring spådde af allt detta, att någon stor förändring skulle
hända Muistola-herrn.

Men huru förhöll det sig dermed, att vår raska hjeltinna, som icke det
ringaste brukade vara besvärad af aningar och oroande inbillningar, just
denna gång vid tanken på den ``gamle främmande herrns öde'' blifvit ett
rof för en så utomordentlig oro? Hade Jenny måhända några skäl att
emotse ett besök af sin gamle, älskade förmyndare? Derom skola vi väl i
det följande få veta något mera.

\begin{enumerate}
\def\labelenumi{\arabic{enumi}.}
\setcounter{enumi}{24}
\tightlist
\item
\end{enumerate}

Gäster på Ojala.

Tant Agatha, som till en början blifvit något ``perplex'' öfver sin
broder häradshöfdingens plötsliga ankomst och äfventyr, lugnades dock
snart åter af de förklaringar, som han gifvit henne. ``Men det bästa,
mina vänner'', hade gubben Stråle sagt, ``sparar jag till aftonen; spara
alltså också ni er nyfikenhet till dess.'' Erik återvände med vagnen
till Muistola och morbrodren hade på eftermiddagen ett allvarsamt samtal
på tu man hand med sin sjelfmyndiga älskling, såsom han kallade Jenny.

Då qvällen stundade blef der lif och rörelse på det lilla Ojala.
Gästerna från Muistola anlände, och äfven länsmannen kom, efter lätt och
väl förrättadt ärende, pösande i sin nya uniform med stickert vid sidan
och kokarden i mössan, som blänkte i aftonsolens sken, och nådigt
besvarande de mötande böndernas ödmjuka helsningar. Han blickade dock
ibland med ett egendomligt uttryck af saknad på den venstra sidan af
sitt bröst, liksom skulle hans hjerta längta efter någon betäckning
utanpå uniformsrocken. Och i sanning, nu efter de fyra rymmarnes
gripande, öfver hvilken tilldragelse han ämnade anmoda Erik Stenrot att,
såsom ``skicklig i svenskan'', hjelpa honom uppsätta en omständlig
berättelse, nu kunde han ha skäl att emotse en sådan utmärkelse.

Äfven den beskedlige ``kommissionären'' Pettersson från Imatra hade
erhållit en bjudning till Ojala och anlände i sin ``trilla'', uppsträckt
i frack och hvit väst, tacksam för den honom visade artigheten. Han blef
litet stött då han, kort efter sin ankomst, i mystiska ordalag erhöll en
vink af Smilander, att i afton icke kalla denne herre ``bror'', ``ty'',
hade länsmannen sagt, ``vi administrativa tjenstemän och isynnerhet
polisembetsmän få inte stå på alltför förtrolig fot med allmänheten,
allraminst i närvaro af utländska högre dignitärer'', hvarpå Pettersson
bugat sig och frågat: ``om kanske denne Stråles namn klingade mera
adligt än systerns?'' Med en blick af medömkan öfver den andres
bristande förmåga att bedömma förnämt folks verkliga ``karakter'',
trädde länsmannen stolt förbi honom in i salongen, der tant Agatha
serverade théet.

Ehuru Smilander ingalunda var en synnerligen angenäm gäst för henne,
bemötte dock den goda damen honom med all artighet, och Jenny lät
åtminstone bli att skratta åt hans nästan krypande sätt emot morbror
Stråle, för hvilket hon förgäfves sökte en förklaringsgrund. Jenny var
öfver hufvud taget i qväll mindre munter än vanligtvis, men fäste dock
Eriks uppmärksamhet vid Smilander, och denne beslöt att drifva ett litet
oskyldigt gyckel med den administrativa tjenstemannens fåfänga, i loflig
liqvid för den brutalitet han fått röna vid frågan om fruntimrens pass.
Han egnade honom derföre vida mera uppmärksamhet än han i annat fall
brytt sig om att göra. Emot Pettersson var Jenny vänligheten sjelf, och
tant Betty afhandlade med honom allehanda intressanta frågor i
landtbruket. Aina sysslade mycket omkring tant Agatha, hvilken hon
erbjöd sig att hjelpa i hushållsbestyren, icke utan att härtill måhända
föranledas af en viss, ganska naturlig önskan att ännu högre stiga i
gunst hos den gamla damen. Stråle talade politik med Ros och kunde icke
nog förvåna sig öfver den unge mannens skarpa omdöme och det egendomliga
och nya ljus, i hvilket denne framställde den finska nationalitetsfrågan
för honom. Men då théet var drucket vände sig den gamle herrn till Jenny
och yttrade i synnerligen godlynt ton:

``Och nu, sedan du blifvit en finsk förtrollerska, kunde du just åt din
gamle svenske morbror och hans gäster i all hast frambesvärja en liten
toddy.''

``Skall ske såsom herr processrådet befaller'', svarade Jenny skrattande
för att dölja en plötslig rodnad. De behöriga reqvisita voro snart
framsatte i ett sidorum, der herrarne slogo sig ned, tände sina cigarrer
och bryggde sina glas.

``Nå'', menade den jovialiske värden när detta vigtiga värf var
undangjordt och man smakat på drycken, ``nu få vi be herr befallningsman
att omtala för oss huru det gick till vid fångandet af de stackars
landsvägsriddarne, som i går qväll gjorde så dåliga affärer?''

Herr Smilander tog sig en ny styrkdryck ur glaset och beredde sig att
tala. Med händerna på sina knän satt herr Pettersson och lyssnade och
Stråle, med sin runda lekamen behagligt placerad i en beqväm fåtölj,
tycktes också vara idel öra, medan deremot de unga männen, först Erik
och sedan äfven Ros, sakta lemnade rummet. Ros dröjde borta en längre
tid, men Erik återkom om en stund, just lagom att få höra berättelsens
slut.

``Och'', fortfor herr Smilander, i det han fattade sitt glas och med
värdighet liksom drack sig sjelf till, ``och sedan vi utställt tio
vakter, inträngde andra tio man modigt i ladan, sålunda att skallfogden,
sitt embete likmätigt, gick främst och sedan brofogden med karlarne.
Derefter kom jag ensam, utgörande liksom en påtryckning å hela
anfallskolonnen, hvilket också i sanning var af nöden, ty i annat fall
hade rackrarne möjligtvis kunnat rymma sin kos. Men se, den som från
början ställde sig vid ingången, det var jag, färdig att vid ringaste
tecken till tvekan å mina mäns eller kronovidrigt motstånd å fiendens
sida beslutsamt skynda ut och jämväl, just i rättan tid, sända
nämdemännen med reservmanskapet i elden. Men detta behöfdes inte. De
sluga skälmarne låtsade gifva sig på nåd och onåd samt föllo på knä och
bådo om förskoning och mat, hvilket jag genast insåg var en krigslist,
ty man skall veta att en sådan der rymmare, rätt så utsvulten han är, i
sig förenar fem mans styrka då det för honom gäller att vinna den der
förbjudna frukten, friheten. Jag lät derföre genast basta och binda de
skälmarne, och sedan de, väl bundne och liggande på golfvet, åter kunde
anses som myndigheternas säkra och välförvärfvade egendom, skyndade jag
att medelst några passande tecken å deras ryggar och hufvuden påminna
dem om min närvaro i egen person samt dymedelst förmå dem till
bekännelse af hvart de `andra rymmarne' tagit vägen, ehuru jag, hi hi
hi, aldrig af landshöfdingeembetet fått underrättelse om flera än fyra
af den sorten -- och de lågo der. Sålunda räddade jag vårt samhälle från
den öfverhängande faran att bli offer för dessa illgerningsmäns
lagstridiga framfart. Som de voro något illa medfarne af föregående
nattens strid samt antagligen ganska hungriga, förbjöd jag att gifva dem
någon annan torplägning än stryk för att få dem rätt spaka.''

``Nedrig grymhet!'' mumlade Erik för sig sjelf, men gamle Stråle frågade
lugnt om skallfogden åtföljt länsmannen till Ojala?

``O, hvilken förutseende blick!'' utropade Smilander, ``ja visst, han är
här!''

``Var god, herr befallningsman'', sade Stråle med egendomlig tonvigt,
``och gif order att fångarne genast skola få sig en extra forplägning;
jag betalar det.''

``Mycket gerna'', svarade Smilander något tveksamt, ``men jag vet inte
om jag just har rättighet att ännu mera än redan skett\ldots{}''

``Erik!'' afbröt Stråle otåligt, ``bed din fästmö genast sända något bud
till fångarnes bevakningsställe med lämpliga förfriskningar åt de
stackrarne.''

``Aha'', tänkte Smilander, ``han menade \emph{sådan} forplägning'', och
tillade derefter högt: ``Vid Gud, en ädel hämnd!''

``Skynda dig Erik'', bad Stråle.

Erik skyndade ut men Smilander började begrunda hvad det der ordet
``fästmö'' skulle betyda och om det gällde den spottska svenska fröken
eller doktorns syster. I alla fall beslöt han att passa på tillfälle och
närmare taga reda på denna sak samt tillika rycka fram med anhållan om
uppsättandet af berättelsen angående röfrarnes gripande.

Erik återkom snart och meddelade att Aina dragit försorg om att
morbroderns önskan skulle uppfyllas. Derefter visste han att skickligt
gifva samtalet den vändning att Pettersson och onkel Stråle kommo i
samspråk om laxfänget i forsen, ett ämne som intresserade den gamle
ungkarlen. Under tiden egnade sig Stenrot helt och hållet åt Smilander
och uppmanade denne slutligen att i sin ``rapport'' infläta en antydning
derom att han, Smilander, gjort staten en väsentlig tjenst genom att
visa sig så nitisk i en sak som rörde en utländing af herr Stråles
karakter.

``Och skål derpå!'' tillade skalken, ``får herrn inte en rysk orden, så
kan ta' mig böfveln nog gubben Stråle göra er till medlem af en svensk
orden.''

``Hvilken?'' frågade befallningsmannen i hviskande ton och rodnande i
försmaken af tillfredsställd fåfänga.

``Hertigar och grefvar bära dess tecken'', sade Erik, ``skål!''

``Dess namn, dess namn!'' bad Smilander och klingade med Eriks glas.

``Tala närmare om saken med herr Stråle -- han är kommendör'', blef
svaret.

Till Smilanders harm tillkännagaf nu tant Agatha att aftonvarden var
färdig.

\begin{enumerate}
\def\labelenumi{\arabic{enumi}.}
\setcounter{enumi}{25}
\tightlist
\item
\end{enumerate}

Skuggan viker.

Medan de andra herrarne voro sålunda sysselsatte, en hvar på sitt håll,
företog vår vän Ros en promenad utåt forsen. Flera af de yttranden som
under dagens lopp undfallit gamle Stråle, hade i den unge mannens bröst
återväckt tankar af samma natur med dem hvilka uppfyllt hans själ under
gårdagens nattliga ridt. I smärtsamt lifliga färger stod framför honom
bilden af den älskades gestalt, men den mörka skuggan af det oblidkeliga
olycksödet lade sin iskalla dödshand på det liffulla hjertats skönaste
förhoppningar. Dyster i hogen, ett rof för de mest omvexlande känslor,
som än ryckte honom med sig till en besvarad kärleks svindlande höjd, än
störtade hans arma själ i hopplöshetens kaos, vandrade han fram sin
ensliga väg utmed den dånande forsens klippbrädd. Han stannade, aftog
hatten och lät den friska, af fina vattendunster mättade aftonflägten
svalka hans panna, omkring hvars hvita, fasta hvälfning det mörka håret
bildade en dunkel ram. Hvem hade kunnat tolka de tankar som nu rörde sig
inom densamma, der han stirrade ned i den brusande afgrunden? Så stod
han länge, men på fästets mörkblå grund försvunno småningom äfven de
sista molnen och Septembermånens blanka sken utgjöt sig i underbar
fullhet öfver det vilda landskapet. Så stod han der den sörjande
menniskosonen med armarne korslagda öfver det stolta men qvaluppfyllda
bröstet. Han stod med sin andes öga förfarande klart fästadt på en
pligtuppfylld, men fröjdlös framtid: ofvanom sig den lugne himmelske
väktaren och nedanföre forsens eviga brus.

Lätt lades två små händer på hans skuldror och i det försilfrande
månskenet stod emellan honom och fallet, just på klippans brant, en
liten elfva, hvilken, höjande sig på tåspetsarne, sänkte sina blickars
ljus genom tviflarens egna ögon djupt, djupt till dess det nådde till
bottnen af den älskades själ.

``Kom, Birger'', hviskade hon, ``jag har sökt dig.''

De trädde ett steg tillbaka från branten och i det hon blygt lade sin
ena arm om ynglingens hals, fattade hon med den andra hans hand och
sade:

``Du har inte kunnat tala, Birger, jag vet ju hvarför, men skuggan har
vikit och jag kan nu tala mitt hjertas språk och få lyssna till ditt.
Hvad du varit för mig, det skall du alltid förblifva, men jag är
\emph{nu} inför dig och verlden -- en Finlands dotter. \emph{Den}
underrättelsen medförde morbror Stråle; jag har så velat det.''

Ett djupt andetag höjde Birger Ros' bröst och han sade sakta:

``Och denna uppoffring har du hemburit mig!''

``Ingen uppoffring'', svarade Jenny och rodnade varmt, ``jag var väl
förut din svenska syster men nu ar jag din finska brud.''

Och hon gömde sina ögon, i hvilka kärlekens perlor lyste, vid den
älskade mannens bröst. Men snart höjde hon åter sitt lockiga hufvud och
blickade förtröstansfullt leende upp till honom. Hans läppar rördes ännu
liksom i stilla bön och sedan -- i nästa ögonblick -- tryckte han den
ljufliga flickan till sitt bröst och en helig kyss förenade de två ädla
menniskorna för evigt med hvarandra.

Men i det klara månskenet svingade alla Imatras elfvor vid forsens
dånande jubelmusik i en glädtig dans och sjöngo dertill, för dem hvilka
förstodo orden, en sång om den sanna kärleken som ``icke tror på ett
oblidkeligt olycksöde.''

\begin{enumerate}
\def\labelenumi{\arabic{enumi}.}
\setcounter{enumi}{26}
\tightlist
\item
\end{enumerate}

Den vackra adoptivdottern.

Efter qvällsvarden, när vinet perlade i de fyllda glasen, höll morbror
Stråle ett kort tal i följande ordalag:

"Han, såsom Jennys förmyndare, hade på hennes allvarligt förklarade
beslut att för alltid flytta till Finland, enligt hennes plan och egen
vilja, i hennes namn vidtagit följande åtgärder:

``Han hade till Finland förflyttat och i Föreningsbanken deponerat Jenny
Bertrams hela förmögenhet samt dessutom inköpt landtstället Ojala,
hvarest med ett kapital af etthundratusen mark till grundfond, den
första finska folkhögskola skulle inrättas. Vidare hade hans myndling
anslagit en summa af etthundratusen mark till grundkapital för
uppsättandet af en större tidning på finska språket, hvars
hufvudredaktör, vald för fem år, städse skulle utses af den studerande
ungdomen vid Helsingfors universitet. Sålunda hade Jenny Bertram
tillfört sitt nya fosterland ett ansenligt kapital, hvilket han hoppades
skulle blifva rikligt fruktbärande. I den förening, som skulle bli en
sannolik följd af detta hans myndlings steg, älskade han att se en
framtidshägring af det finska och det svenska elementets
sammansmältande, ty om \emph{de} tu varda \emph{ett}, då bygger Finland
sin framtid på en ömsesidigt bepröfvad kärleks grundval, -- och liksom
han nu frivilligt nedlade sitt förmyndarembete, så har ock moderlandet
förtröstansfullt nedlagt sitt ledarekall såsom en gärd af aktning för
det unga samhällets sjelfbestämmelserätt och framtidsförhoppningar. Med
denna liknelse till utgångspunkt tog sig derföre talaren friheten att
föreslå en välgångsskål för den svenska fröken Jenny Bertram, Finlands
vackra adoptivdotter!''

Och härmed slutar vår lilla sommarsaga från de tusen sjöarnes land.

Efterskrift.

Vår uppgift har icke varit att skrifva Finlands kulturhistoria. Vi ha
blott velat lemna ett ringa bidrag dertill, en liten tafla ur detta med
Guds hjelp och på grund af egen kraft blifvande framtidsfolks historia
för dagen. Och det är sannt: Finland har sig anvisad en ganska
egendomlig roll i den europeiska nordens historia. Om vi äfven icke
kunna hylla aflidne Hwassers (se hans bekanta skrift om Borgå landtdag)
mera snillrika uppfattning af hvad Finland \emph{kunde} blifva, än
praktiskt taget möjliga bestämmelse, så kan man dock i alla fall inrymma
en stor betydelse åt detta lilla folk. Tron att Finland skulle bli den
brygga på hvilken vesterländsk civilisation skulle bana sig väg till
Orienten, måste anses såsom en öfvervunnen ståndpunkt. Deremot är dess
betydelse för ``vesterlandet'' i följd häraf icke mindre. Frågan gäller
nämligen icke \emph{Finlands inflytande på östern}, utan tvärtom dess
förmåga att motstå tyngden af Orientens påtryckning. Skall denna af den
vesterländska bildningen till \emph{europeiskt} kulturlif väckta nation
nu, sedan densamma på sätt och vis blifvit sjelfständig, af uteslutande
egna krafter kunna häfda dess under svensk egid vunna rykte om ``finsk
ihärdighet'' -- ett verkligen historiskt hedrande namn? Vi för vår del
vilja tro detta, nämligen i fall de finska förhållandena gestalta sig
sålunda att folket får begagna sig af sina \emph{båda} axlar för att
lyfta den i sanning icke ringa börda af kulturhistorisk bestämmelse, som
verldshistorien pålagt det goda Suomifolket. Gud har gifvit finnarne
uthållighetens oskattbara förmåga, och detta i hög grad, men hvad
Finlands europeiska betydelse vidkommer beror värdet af denna egenskap
helt och hållet uppå den riktning som denna uthållighet behagar taga. Vi
nämde här nyss förut att Finland, det ``unga Finland'' ville jag säga, i
likhet med alla andra välskapade \emph{barn}, äfven då dessa framträda
på politikens fält, har \emph{två} axlar. Nu gäller det, såsom vi redan
i vår lilla berättelse liknelsevis påpekat, huruvida icke den ena armen,
den ursprungligt starkare armen, det finska folket i fennomansk mening,
den ``finska massan'' på det sättet söker att styra sin båt i hamn att
den venstra armens, det svenska elementets, verksamhet blir af
jemförelsevis ringa båtnad. Men just denna andra axel i det ``nya
Finlands'' (sit venia verbo) lif är den del af dess statskropp som
historiskt ovedersägligen, och för alla kommande tider, utgör den
förbindande länken emellan Suomifolket och vesterlandet.

Finland kan visserligen numera, derigenom att en del af dess söner
förnekar sina andligen svenska anor, till och med inom en jemförelsevis
kort tidrymd framställa taflan af en ``finsk-finsk'' kulturform. Svårt
vore dock för hvarje fördomsfri att säga hvari egentligen denna specielt
finska kulturform skulle skilja sig ifrån samma banor som folkets
utveckling hittills följt. Eller skulle Finland vilja afsäga sig sin
från vesterlandet ärfda politiska konstitution och skulle dess religiösa
lif vilja gå samma väg? Vi tro det icke. Återstår således
\emph{språkfrågan}, uppdrifven till sin spets af de så kallade
fennomanerne. Men hvad återstår \emph{numera} att önska i denna väg?
Kanske en egen finsk-finsk litteratur? Intet annat förhindrar ju
uppkomsten af en sådan än möjligtvis den ryska politiken; men denna
verkar i ännu vida högre grad hämmande på det svensk-finska
skriftställeriets utveckling. Eller ligger härinunder måhända en annan
orsak? Det är ju icke språket som \emph{``gör''} litteraturen, fastmera
återverkar en högre utbildad andlig verksamhet på språkets förädling.
Deri ha vi således icke heller roten till ``det onda'', hvilket
härvidlag är svårigheten för den finska litteraturen att arbeta sig fram
till en sjelfständig betydelse. Orsaken bör således sökas antingen hos
den finska folkmassans oförmåga att uppbära ett vittert lif i högre
mening, eller ock i bristen på skapande förmågor. Men hvar skall
anledningen till denna sistnämda fattigdom sökas om icke i folkets
oförmåga att producera dessa behöfliga litteratur-representanter?

Såsom bekant saknar den svensk-finska litteraturen ingalunda namn, som i
vida kretsar klinga smickrande för det folk de tillhöra. Och dessa namns
uppbärare äro finnar till själ och hjerta. Eller vågar måhända någon af
herrar fennomaner verkligen säga att så icke är fallet. Detta tro vi
icke.

Men liksom det finnes vissa ganska ädla växtarter som, trots alla de
goda egenskaper de utveckla efter befruktningen, emellertid af naturen
-- eller kalla det hellre försynen -- i detta afseende hänvisats till
behofvet af medverkan från medlemmar af till och med andra naturriken,
-- hvarför skulle man icke häruti kunna se en fingervisning att äfven de
skilda nationerna behöfva hvarandra och hvartill skulle då öfver hufvud
taget en verldshistoria tjena?

Nåväl: den svenska nationaliteten har trädt i beröring med den finska.
Från hvilkendera sidan ha nu anledningarne till ett högre andligt lif
utgått, det är en fråga, som hvarje öfver denna sak tänkande menniska
har lätt att besvara. Om nu finska folket tror att timmen slagit för
detsamma att utan all vexelverkan med sin naturligaste bundsförvandt
framdeles dana sin egen historia -- så kommer väl Sverge att sörja öfver
detta misstag, det är visst och sannt, och svårast blir de
hundratusentals svensk-finnars ställning, som bygga och bo i
Suomilandet, men det oaktadt torde icke ens den djerfvaste
nationalfåfänga i denna sorg kunna se annat än den äldre broderns
grämelse öfver den yngres \emph{förhastade} beslut att vända ryggen åt
sin egen bildningskälla. Hvad Sverge \emph{fordrar} af Finland är icke
annat än att Finland, i de förlåtliga utbrotten af sin vaknande
nationalkänsla, icke skall förgå sig emot sig sjelf, att icke den ena af
dess armar skall höja sig emot den andra, som så villigt blifvit
framräckt för att förenas till ett gemensamt arbete för andligt
sjelfbestånd.

Några af de företeelser som egt rum vid den senaste finska landtdagen
(vi skrifva nu Maj år 1872) måste ovilkorligen framkalla betraktelser af
detta slag i hvarje sinne, öppet för rättvisa och hvars verldsåskådning
icke ensidigt riktats åt parti-intressen. Visst önskar Sverge fästa
Finland vid sig, men detta i den mening att de band hvarmed detta skulle
ske icke äro maktens eller styrkans, icke heller den andliga
öfverlägsenhetens, icke ens tacksamhetens och de gemensamma minnenas i
alla fall vackra och här erkända begrepp -- utan det band, medelst
hvilket den skandinaviska norden vill och önskar fästa Finlands kultur
vid sig och sin utveckling, är öfvertygelsen derom att Suomifolkets enda
utväg att bevara sig åt sig sjelf förenas med det höga uppdraget att
derjemte utgöra ett vesterländskt skyddsvärn emot orientaliskt våld.
Finnarne, de nämligen som bo emellan Ladoga och Bottniska viken, äro
enligt vår oförgripliga åsigt \emph{numera} europeer -- och ha i följd
deraf ovilkorligen samma pligter i och för den vesterländska kulturens
upprätthållande sig ålagde som vår verldsdels öfriga nationer. Att den
finska nationen i politiskt afseende ännu är ung, veta vi, men vi hoppas
dock att denne vår yngste broder i kulturlifvets led just i och genom
sin nu vunna sjelfständighet (isynnerhet sådan denna de fakto är)
påminnes derom att: en fullvuxen yngling bör kunna utveckla så mycken
omdömesförmåga att han icke, såsom det heter, ``uppbränner skeppen bakom
sig''.

End of Project Gutenberg's En sommarsaga från Finland, by Johannes
Alfthan

\begin{verbatim}
        *** END OF THE PROJECT GUTENBERG EBOOK EN SOMMARSAGA FRÅN FINLAND ***
\end{verbatim}

Updated editions will replace the previous one---the old editions will
be renamed.

Creating the works from print editions not protected by U.S. copyright
law means that no one owns a United States copyright in these works, so
the Foundation (and you!) can copy and distribute it in the United
States without permission and without paying copyright royalties.
Special rules, set forth in the General Terms of Use part of this
license, apply to copying and distributing Project Gutenberg™ electronic
works to protect the PROJECT GUTENBERG™ concept and trademark. Project
Gutenberg is a registered trademark, and may not be used if you charge
for an eBook, except by following the terms of the trademark license,
including paying royalties for use of the Project Gutenberg trademark.
If you do not charge anything for copies of this eBook, complying with
the trademark license is very easy. You may use this eBook for nearly
any purpose such as creation of derivative works, reports, performances
and research. Project Gutenberg eBooks may be modified and printed and
given away---you may do practically ANYTHING in the United States with
eBooks not protected by U.S. copyright law. Redistribution is subject to
the trademark license, especially commercial redistribution.

START: FULL LICENSE

THE FULL PROJECT GUTENBERG LICENSE

PLEASE READ THIS BEFORE YOU DISTRIBUTE OR USE THIS WORK

To protect the Project Gutenberg™ mission of promoting the free
distribution of electronic works, by using or distributing this work (or
any other work associated in any way with the phrase ``Project
Gutenberg''), you agree to comply with all the terms of the Full Project
Gutenberg™ License available with this file or online at
www.gutenberg.org/license.

Section 1. General Terms of Use and Redistributing Project Gutenberg™
electronic works

1.A. By reading or using any part of this Project Gutenberg™ electronic
work, you indicate that you have read, understand, agree to and accept
all the terms of this license and intellectual property
(trademark/copyright) agreement. If you do not agree to abide by all the
terms of this agreement, you must cease using and return or destroy all
copies of Project Gutenberg™ electronic works in your possession. If you
paid a fee for obtaining a copy of or access to a Project Gutenberg™
electronic work and you do not agree to be bound by the terms of this
agreement, you may obtain a refund from the person or entity to whom you
paid the fee as set forth in paragraph 1.E.8.

1.B. ``Project Gutenberg'' is a registered trademark. It may only be
used on or associated in any way with an electronic work by people who
agree to be bound by the terms of this agreement. There are a few things
that you can do with most Project Gutenberg™ electronic works even
without complying with the full terms of this agreement. See paragraph
1.C below. There are a lot of things you can do with Project Gutenberg™
electronic works if you follow the terms of this agreement and help
preserve free future access to Project Gutenberg™ electronic works. See
paragraph 1.E below.

1.C. The Project Gutenberg Literary Archive Foundation (``the
Foundation'' or PGLAF), owns a compilation copyright in the collection
of Project Gutenberg™ electronic works. Nearly all the individual works
in the collection are in the public domain in the United States. If an
individual work is unprotected by copyright law in the United States and
you are located in the United States, we do not claim a right to prevent
you from copying, distributing, performing, displaying or creating
derivative works based on the work as long as all references to Project
Gutenberg are removed. Of course, we hope that you will support the
Project Gutenberg™ mission of promoting free access to electronic works
by freely sharing Project Gutenberg™ works in compliance with the terms
of this agreement for keeping the Project Gutenberg™ name associated
with the work. You can easily comply with the terms of this agreement by
keeping this work in the same format with its attached full Project
Gutenberg™ License when you share it without charge with others.

1.D. The copyright laws of the place where you are located also govern
what you can do with this work. Copyright laws in most countries are in
a constant state of change. If you are outside the United States, check
the laws of your country in addition to the terms of this agreement
before downloading, copying, displaying, performing, distributing or
creating derivative works based on this work or any other Project
Gutenberg™ work. The Foundation makes no representations concerning the
copyright status of any work in any country other than the United
States.

1.E. Unless you have removed all references to Project Gutenberg:

1.E.1. The following sentence, with active links to, or other immediate
access to, the full Project Gutenberg™ License must appear prominently
whenever any copy of a Project Gutenberg™ work (any work on which the
phrase ``Project Gutenberg'' appears, or with which the phrase ``Project
Gutenberg'' is associated) is accessed, displayed, performed, viewed,
copied or distributed:

\begin{verbatim}
This eBook is for the use of anyone anywhere in the United States and most
other parts of the world at no cost and with almost no restrictions
whatsoever. You may copy it, give it away or re-use it under the terms
of the Project Gutenberg License included with this eBook or online
at www.gutenberg.org. If you
are not located in the United States, you will have to check the laws
of the country where you are located before using this eBook.
\end{verbatim}

1.E.2. If an individual Project Gutenberg™ electronic work is derived
from texts not protected by U.S. copyright law (does not contain a
notice indicating that it is posted with permission of the copyright
holder), the work can be copied and distributed to anyone in the United
States without paying any fees or charges. If you are redistributing or
providing access to a work with the phrase ``Project Gutenberg''
associated with or appearing on the work, you must comply either with
the requirements of paragraphs 1.E.1 through 1.E.7 or obtain permission
for the use of the work and the Project Gutenberg™ trademark as set
forth in paragraphs 1.E.8 or 1.E.9.

1.E.3. If an individual Project Gutenberg™ electronic work is posted
with the permission of the copyright holder, your use and distribution
must comply with both paragraphs 1.E.1 through 1.E.7 and any additional
terms imposed by the copyright holder. Additional terms will be linked
to the Project Gutenberg™ License for all works posted with the
permission of the copyright holder found at the beginning of this work.

1.E.4. Do not unlink or detach or remove the full Project Gutenberg™
License terms from this work, or any files containing a part of this
work or any other work associated with Project Gutenberg™.

1.E.5. Do not copy, display, perform, distribute or redistribute this
electronic work, or any part of this electronic work, without
prominently displaying the sentence set forth in paragraph 1.E.1 with
active links or immediate access to the full terms of the Project
Gutenberg™ License.

1.E.6. You may convert to and distribute this work in any binary,
compressed, marked up, nonproprietary or proprietary form, including any
word processing or hypertext form. However, if you provide access to or
distribute copies of a Project Gutenberg™ work in a format other than
``Plain Vanilla ASCII'' or other format used in the official version
posted on the official Project Gutenberg™ website (www.gutenberg.org),
you must, at no additional cost, fee or expense to the user, provide a
copy, a means of exporting a copy, or a means of obtaining a copy upon
request, of the work in its original ``Plain Vanilla ASCII'' or other
form. Any alternate format must include the full Project Gutenberg™
License as specified in paragraph 1.E.1.

1.E.7. Do not charge a fee for access to, viewing, displaying,
performing, copying or distributing any Project Gutenberg™ works unless
you comply with paragraph 1.E.8 or 1.E.9.

1.E.8. You may charge a reasonable fee for copies of or providing access
to or distributing Project Gutenberg™ electronic works provided that:

\begin{verbatim}
• You pay a royalty fee of 20% of the gross profits you derive from
    the use of Project Gutenberg™ works calculated using the method
    you already use to calculate your applicable taxes. The fee is owed
    to the owner of the Project Gutenberg™ trademark, but he has
    agreed to donate royalties under this paragraph to the Project
    Gutenberg Literary Archive Foundation. Royalty payments must be paid
    within 60 days following each date on which you prepare (or are
    legally required to prepare) your periodic tax returns. Royalty
    payments should be clearly marked as such and sent to the Project
    Gutenberg Literary Archive Foundation at the address specified in
    Section 4, “Information about donations to the Project Gutenberg
    Literary Archive Foundation.”

• You provide a full refund of any money paid by a user who notifies
    you in writing (or by e-mail) within 30 days of receipt that s/he
    does not agree to the terms of the full Project Gutenberg™
    License. You must require such a user to return or destroy all
    copies of the works possessed in a physical medium and discontinue
    all use of and all access to other copies of Project Gutenberg™
    works.

• You provide, in accordance with paragraph 1.F.3, a full refund of
    any money paid for a work or a replacement copy, if a defect in the
    electronic work is discovered and reported to you within 90 days of
    receipt of the work.

• You comply with all other terms of this agreement for free
    distribution of Project Gutenberg™ works.
\end{verbatim}

1.E.9. If you wish to charge a fee or distribute a Project Gutenberg™
electronic work or group of works on different terms than are set forth
in this agreement, you must obtain permission in writing from the
Project Gutenberg Literary Archive Foundation, the manager of the
Project Gutenberg™ trademark. Contact the Foundation as set forth in
Section 3 below.

1.F.

1.F.1. Project Gutenberg volunteers and employees expend considerable
effort to identify, do copyright research on, transcribe and proofread
works not protected by U.S. copyright law in creating the Project
Gutenberg™ collection. Despite these efforts, Project Gutenberg™
electronic works, and the medium on which they may be stored, may
contain ``Defects,'' such as, but not limited to, incomplete, inaccurate
or corrupt data, transcription errors, a copyright or other intellectual
property infringement, a defective or damaged disk or other medium, a
computer virus, or computer codes that damage or cannot be read by your
equipment.

1.F.2. LIMITED WARRANTY, DISCLAIMER OF DAMAGES - Except for the ``Right
of Replacement or Refund'' described in paragraph 1.F.3, the Project
Gutenberg Literary Archive Foundation, the owner of the Project
Gutenberg™ trademark, and any other party distributing a Project
Gutenberg™ electronic work under this agreement, disclaim all liability
to you for damages, costs and expenses, including legal fees. YOU AGREE
THAT YOU HAVE NO REMEDIES FOR NEGLIGENCE, STRICT LIABILITY, BREACH OF
WARRANTY OR BREACH OF CONTRACT EXCEPT THOSE PROVIDED IN PARAGRAPH 1.F.3.
YOU AGREE THAT THE FOUNDATION, THE TRADEMARK OWNER, AND ANY DISTRIBUTOR
UNDER THIS AGREEMENT WILL NOT BE LIABLE TO YOU FOR ACTUAL, DIRECT,
INDIRECT, CONSEQUENTIAL, PUNITIVE OR INCIDENTAL DAMAGES EVEN IF YOU GIVE
NOTICE OF THE POSSIBILITY OF SUCH DAMAGE.

1.F.3. LIMITED RIGHT OF REPLACEMENT OR REFUND - If you discover a defect
in this electronic work within 90 days of receiving it, you can receive
a refund of the money (if any) you paid for it by sending a written
explanation to the person you received the work from. If you received
the work on a physical medium, you must return the medium with your
written explanation. The person or entity that provided you with the
defective work may elect to provide a replacement copy in lieu of a
refund. If you received the work electronically, the person or entity
providing it to you may choose to give you a second opportunity to
receive the work electronically in lieu of a refund. If the second copy
is also defective, you may demand a refund in writing without further
opportunities to fix the problem.

1.F.4. Except for the limited right of replacement or refund set forth
in paragraph 1.F.3, this work is provided to you `AS-IS', WITH NO OTHER
WARRANTIES OF ANY KIND, EXPRESS OR IMPLIED, INCLUDING BUT NOT LIMITED TO
WARRANTIES OF MERCHANTABILITY OR FITNESS FOR ANY PURPOSE.

1.F.5. Some states do not allow disclaimers of certain implied
warranties or the exclusion or limitation of certain types of damages.
If any disclaimer or limitation set forth in this agreement violates the
law of the state applicable to this agreement, the agreement shall be
interpreted to make the maximum disclaimer or limitation permitted by
the applicable state law. The invalidity or unenforceability of any
provision of this agreement shall not void the remaining provisions.

1.F.6. INDEMNITY - You agree to indemnify and hold the Foundation, the
trademark owner, any agent or employee of the Foundation, anyone
providing copies of Project Gutenberg™ electronic works in accordance
with this agreement, and any volunteers associated with the production,
promotion and distribution of Project Gutenberg™ electronic works,
harmless from all liability, costs and expenses, including legal fees,
that arise directly or indirectly from any of the following which you do
or cause to occur: (a) distribution of this or any Project Gutenberg™
work, (b) alteration, modification, or additions or deletions to any
Project Gutenberg™ work, and (c) any Defect you cause.

Section 2. Information about the Mission of Project Gutenberg™

Project Gutenberg™ is synonymous with the free distribution of
electronic works in formats readable by the widest variety of computers
including obsolete, old, middle-aged and new computers. It exists
because of the efforts of hundreds of volunteers and donations from
people in all walks of life.

Volunteers and financial support to provide volunteers with the
assistance they need are critical to reaching Project Gutenberg™'s goals
and ensuring that the Project Gutenberg™ collection will remain freely
available for generations to come. In 2001, the Project Gutenberg
Literary Archive Foundation was created to provide a secure and
permanent future for Project Gutenberg™ and future generations. To learn
more about the Project Gutenberg Literary Archive Foundation and how
your efforts and donations can help, see Sections 3 and 4 and the
Foundation information page at www.gutenberg.org.

Section 3. Information about the Project Gutenberg Literary Archive
Foundation

The Project Gutenberg Literary Archive Foundation is a non-profit
501(c)(3) educational corporation organized under the laws of the state
of Mississippi and granted tax exempt status by the Internal Revenue
Service. The Foundation's EIN or federal tax identification number is
64-6221541. Contributions to the Project Gutenberg Literary Archive
Foundation are tax deductible to the full extent permitted by U.S.
federal laws and your state's laws.

The Foundation's business office is located at 809 North 1500 West, Salt
Lake City, UT 84116, (801) 596-1887. Email contact links and up to date
contact information can be found at the Foundation's website and
official page at www.gutenberg.org/contact

Section 4. Information about Donations to the Project Gutenberg Literary
Archive Foundation

Project Gutenberg™ depends upon and cannot survive without widespread
public support and donations to carry out its mission of increasing the
number of public domain and licensed works that can be freely
distributed in machine-readable form accessible by the widest array of
equipment including outdated equipment. Many small donations (\$1 to
\$5,000) are particularly important to maintaining tax exempt status
with the IRS.

The Foundation is committed to complying with the laws regulating
charities and charitable donations in all 50 states of the United
States. Compliance requirements are not uniform and it takes a
considerable effort, much paperwork and many fees to meet and keep up
with these requirements. We do not solicit donations in locations where
we have not received written confirmation of compliance. To SEND
DONATIONS or determine the status of compliance for any particular state
visit www.gutenberg.org/donate.

While we cannot and do not solicit contributions from states where we
have not met the solicitation requirements, we know of no prohibition
against accepting unsolicited donations from donors in such states who
approach us with offers to donate.

International donations are gratefully accepted, but we cannot make any
statements concerning tax treatment of donations received from outside
the United States. U.S. laws alone swamp our small staff.

Please check the Project Gutenberg web pages for current donation
methods and addresses. Donations are accepted in a number of other ways
including checks, online payments and credit card donations. To donate,
please visit: www.gutenberg.org/donate.

Section 5. General Information About Project Gutenberg™ electronic works

Professor Michael S. Hart was the originator of the Project Gutenberg™
concept of a library of electronic works that could be freely shared
with anyone. For forty years, he produced and distributed Project
Gutenberg™ eBooks with only a loose network of volunteer support.

Project Gutenberg™ eBooks are often created from several printed
editions, all of which are confirmed as not protected by copyright in
the U.S. unless a copyright notice is included. Thus, we do not
necessarily keep eBooks in compliance with any particular paper edition.

Most people start at our website which has the main PG search facility:
www.gutenberg.org.

This website includes information about Project Gutenberg™, including
how to make donations to the Project Gutenberg Literary Archive
Foundation, how to help produce our new eBooks, and how to subscribe to
our email newsletter to hear about new eBooks.
